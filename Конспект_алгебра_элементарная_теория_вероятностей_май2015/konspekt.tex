\documentclass[a4paper,12pt]{extreport}
 
\usepackage{extsizes}
\usepackage{cmap}
\usepackage[T2A]{fontenc}
\usepackage[utf8]{inputenc}
\usepackage[russian]{babel}
\usepackage{cyrtimes}

%%%%%%%%%%%%%%%%%%%%%%%%%%%%%%%%%%%%%%%%%%%%%%%%%%%%%%%%%%%%%%%%%%%%%%%%%%%%%%%%%%  
\usepackage{graphicx} % для вставки картинок
\graphicspath{{img/}}
\usepackage{amssymb,amsfonts,amsmath,amsthm} % математические дополнения от АМС
\usepackage{indentfirst} % отделять первую строку раздела абзацным отступом тоже
\usepackage[usenames,dvipsnames]{color} % названия цветов
\usepackage{makecell}
\usepackage{multirow} % улучшенное форматирование таблиц
\usepackage{ulem} % подчеркивания
\linespread{1.3} % полуторный интервал
\renewcommand{\rmdefault}{ftm} % Times New Roman
\frenchspacing
\usepackage{geometry}
\geometry{left=3cm,right=1cm,top=2cm,bottom=2cm,bindingoffset=0cm}
\usepackage{titlesec}
% \definecolor{black}{rgb}{0,0,0}
% \usepackage[colorlinks, unicode, pagecolor=black]{hyperref}
\usepackage[unicode]{hyperref}
\usepackage{fancyhdr} %загрузим пакет
\pagestyle{fancy} %применим колонтитул
\fancyhead{} %очистим хидер на всякий случай
\fancyhead[LE,RO]{Сарафанов Ф.Г.} %номер страницы слева сверху на четных и справа на нечетных
\fancyhead[CO, CE]{Конспект по алгебре \today}
\fancyhead[LO,RE]{Элементарный теорвер} 
\fancyfoot{} %футер будет пустой
\fancyfoot[CO,CE]{\thepage}
\renewcommand{\labelenumii}{\theenumii)}
\begin{document}

\textbf{Определение 1}\\
Случайное событие - это событие, которое может произойти некого опыта, а может и не наступить.

\textbf{Определение 2}\\
События, которые нельзя разбить на более простые, именуют \textit{элементарными}.

\textbf{Определение 3}\\
Элементарные события, при которых наступает событие $A$, называют \textit{благоприятствующими} событию $A$.

\subsection*{Формула классической вероятности. Определение.}
Вероятностью $P$ события $A$ --- $P(A)$ называют отношение $m$ благоприятствующих исходов этого события к числу $n$  равновозможных исходов.
$$P(A)=\frac{m}{n}$$

\subsection{Задача}
На клавиатуре телефона 10 цифр ${0..9}$. Какова вероятность того, что случайно набранная цифра будет четной? А будет меньше 4? Или будет четной и больше 3?
\begin{center}
Решение
\end{center}
\begin{itemize}
	\item $m={0; 2; 4; 6; 8} \longrightarrow m=5, n={0..9} \longrightarrow n=10$. $P(A)=\frac{m}{n}=\frac{5}{10}=0.5$
	\item $m={0; 1; 2; 3} \longrightarrow m=4, n={0..9} \longrightarrow n=10$. $P(A)=\frac{m}{n}=\frac{4}{10}=0.4$
	\item $m={4; 6; 8} \longrightarrow m=3, n={0..9} \longrightarrow n=10$. $P(A)=\frac{m}{n}=\frac{3}{10}=0.3$
\end{itemize}


\end{document}