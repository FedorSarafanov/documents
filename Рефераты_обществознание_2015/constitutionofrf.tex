\documentclass[a4paper,12pt]{diss_4}
 
% \usepackage{extsizes}
% \usepackage{cmap}
\usepackage[T2A]{fontenc}
\usepackage[utf8x]{inputenc}
\usepackage[russian]{babel}
% \usepackage{cyrtimes}

%%%%%%%%%%%%%%%%%%%%%%%%%%%%%%%%%%%%%%%%%%%%%%%%%%%%%%%%%%%%%%%%%%%%%%%%%%%%%%%%%%  
\usepackage{graphicx} % для вставки картинок
\graphicspath{{img/}}
\usepackage{amssymb,amsfonts,amsmath,amsthm} % математические дополнения от АМС
\usepackage{indentfirst} % отделять первую строку раздела абзацным отступом тоже
\usepackage[usenames,dvipsnames]{color} % названия цветов
\usepackage{makecell}
\usepackage{multirow} % улучшенное форматирование таблиц
\usepackage{ulem} % подчеркивания
\linespread{1.3} % полуторный интервал
% \renewcommand{\rmdefault}{ftm} % Times New Roman
\frenchspacing
\usepackage{geometry}
% \geometry{left=3cm,right=1cm,top=2cm,bottom=2cm,bindingoffset=0cm}
\usepackage{titlesec}
% \definecolor{black}{rgb}{0,0,0}
% \usepackage[colorlinks, unicode, pagecolor=black]{hyperref}
\usepackage[unicode]{hyperref}
\usepackage{epigraph} %%% to make inspirational quotes.

\begin{document}

% -*- root: constitutionofrf.tex -*-
\begin{titlepage}
\newpage

\begin{center}

Муниципальное автономное образовательное учреждение \\
лицей № 180 \\
г. Нижнего Новгорода \\

% \hrulefill
\end{center}
 
% \flushright{КАФЕДРА № ХХХ}

\vspace{14em}

\begin{center}
\large{Реферат}
\end{center}

% \vspace{.5em}
 
\begin{center}
История конституции РФ
\end{center}

\vspace{4.5em}
 
\begin{flushright}
Выполнил: Сарафанов Фёдор, \\
ученик 10 <<А>> класса \\
Научный руководитель: \\
Тереханова Юлия Николаевна, \\
учитель истории и обществознания \\ 
первой квалификационной категории
\end{flushright}
 
\vspace{\fill}

\begin{center}
Нижний Новгород \\
\the\year
\end{center}

\end{titlepage}
\addtocounter{page}{1}

\tableofcontents
%\large

\chapter*{Введение}
\addcontentsline{toc}{chapter}{Введение}

С 1993 года согласно Конституции Россия является вассалом других стран. Евгений Фёдоров рассказывает о первых шагах по восстановлению суверенитета России. Начинать нужно с права на собственную идеологию и собственную госполитику то есть статей 13 и 15 Конституции. После референдума по этим статьям Конституции уже намного легче будет бороться с иностранным управлением (с пятой колонной) и можно будет менять другие статьи Конституции.
Предлагается в пункте 2 статьи 13 Конституции исключить слова «государственной или».
В итоге этот пункт 2 статьи 13 будет выглядеть так: «Никакая идеология не может устанавливаться в качестве обязательной.» 
Убрав запрет на гос.идеологию Россия обретает право на выработку собственного лица, собственного характера и из искусственного младенчества может активно приступить к гос.строительству. У чиновников появятся чёткие ориентиры государственной политики, появятся понятия что хорошо, а что плохо. С гос.идеологией гражданскому обществу намного проще будет привлечь чиновника к ответственности. Надеюсь, что намного проще будет ставить и реализовывать задачи развития России.
Предлагается пункте 4 статьи 15 Конституции исключить слова «Общепризнанные принципы и нормы международного права и»
В итоге этот пункт 4 статьи 15 будет выглядеть так: «Международные договоры Российской Федерации являются составной частью ее правовой системы. Если международным договором Российской Федерации установлены иные правила, чем предусмотренные законом, то применяются правила международного договора.» 
Вроде опять не существенно, однако это прорыв: Россия сможет самостоятельно формировать свои ценности, строить свою систему координат нравственности и отношения к патриотизму. Ведь чиновник без чувства патриотизма равнозначен священнику делающему обряды, но не имеющего веры в себе.


казывается, самые большие беды в России - не те, о которых вы подумали, а тирания Запада и отсутствие государственной идеологии. Депутат Евгений Фёдоров предлагает внести поправки в две статьи Конституции. Из статьи 13, по мнению единоросса, надо упразднить пункт 2: <<Никакая идеология не может устанавливаться в качестве государственной или обязательной>>. Из статьи 15 необходимо изъять пункт 4: <<Общепризнанные принципы и нормы международного права и международные договоры Российской Федерации являются составной частью ее правовой системы. Если международным договором Российской Федерации установлены иные правила, чем предусмотренные законом, то применяются правила международного договора>>.

\chapter*{Заключение}
\addcontentsline{toc}{chapter}{Заключение}%\tabularnewline


\begin{thebibliography}{99}
\addcontentsline{toc}{chapter}{Литература}

\bibitem{nodejs}

\bibitem{iii}

\end{thebibliography}
\end{document}