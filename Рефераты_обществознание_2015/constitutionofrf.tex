\documentclass[a4paper,12pt]{diss_4}
 
% \usepackage{extsizes}
% \usepackage{cmap}
\usepackage[T2A]{fontenc}
\usepackage[utf8x]{inputenc}
\usepackage[russian]{babel}
% \usepackage{cyrtimes}

%%%%%%%%%%%%%%%%%%%%%%%%%%%%%%%%%%%%%%%%%%%%%%%%%%%%%%%%%%%%%%%%%%%%%%%%%%%%%%%%%%  
\usepackage{graphicx} % для вставки картинок
\graphicspath{{img/}}
\usepackage{amssymb,amsfonts,amsmath,amsthm} % математические дополнения от АМС
\usepackage{indentfirst} % отделять первую строку раздела абзацным отступом тоже
\usepackage[usenames,dvipsnames]{color} % названия цветов
\usepackage{makecell}
\usepackage{multirow} % улучшенное форматирование таблиц
\usepackage{ulem} % подчеркивания
\linespread{1.3} % полуторный интервал
% \renewcommand{\rmdefault}{ftm} % Times New Roman
\frenchspacing
\usepackage{geometry}
% \geometry{left=3cm,right=1cm,top=2cm,bottom=2cm,bindingoffset=0cm}
\usepackage{titlesec}
% \definecolor{black}{rgb}{0,0,0}
% \usepackage[colorlinks, unicode, pagecolor=black]{hyperref}
\usepackage[unicode]{hyperref}
\usepackage{epigraph} %%% to make inspirational quotes.

\begin{document}

% -*- root: constitutionofrf.tex -*-
\begin{titlepage}
\newpage

\begin{center}

Муниципальное автономное образовательное учреждение \\
лицей № 180 \\
г. Нижнего Новгорода \\

% \hrulefill
\end{center}
 
% \flushright{КАФЕДРА № ХХХ}

\vspace{14em}

\begin{center}
\large{Реферат}
\end{center}

% \vspace{.5em}
 
\begin{center}
История конституции РФ
\end{center}

\vspace{4.5em}
 
\begin{flushright}
Выполнил: Сарафанов Фёдор, \\
ученик 10 <<А>> класса \\
Научный руководитель: \\
Тереханова Юлия Николаевна, \\
учитель истории и обществознания \\ 
первой квалификационной категории
\end{flushright}
 
\vspace{\fill}

\begin{center}
Нижний Новгород \\
\the\year
\end{center}

\end{titlepage}
\addtocounter{page}{1}

\tableofcontents
%\large

\chapter{История конституции до 1993 года}

Конституция --- это основной закон государства. Он регулирует общественные отношения, которые формируются в процессе осуществления основоположных мероприятий организации общества и государства. Конституция --- политико-правовой документ. В нём выражены базовые устои общественного и государственного строя. Такие как принадлежность государственной власти, характер собственности, права и свободы граждан, государственный и территориальный уклад, в которых описана система осуществления полновластия народа. Конституция стоит во главе системы законов. Конституция - учредительный документ государства и её институций: парламента, правительства, суда, местного самоуправления и т.д. Также это юридический документ, положения которого ни в коем случае не должны противоречить друг другу. Конституция выражает интересы  гражданского общества, главной ценностью которого является человек.

В России первая попытка создать сословную конституцию, ограничивающую самодержавную власть посредством представительного органа и дающую дворянству сословные права, была предпринята в 1730 году в движении, возбуждённом верховниками. Ряд конституционных функций должно было выполнять планировавшееся Уложение, для разработки которого Екатерина II созвала Уложенную комиссию.

Впоследствии конституционные проекты разрабатывались окружением Александра I и декабристами, известен также проект конституционного характера М. Т. Лорис-Меликова, подписанный Александром II в день его гибели, но так и не вступивший в силу. В 1905--1906 годах были приняты Основные государственные законы Российской империи, фактически ставшие первой конституцией России.

Однако, в Основных государственных законах Российской империи в первую очередь постулировались не права человека, а формальный порядок правления престола и порядок престолонаследия. 

Конституция, принятая после Октябрьской революции 1917-го года в 1918 году (Конституция РСФСР 1918 года) была сильно идиологизированной и, по сути, повторяла партийные документы.

Первой конституцией, в большей степени формирующей гражданское общество, стала конституция 1978 года, просуществовавшая без каких-либо значительных поправок до 1993 года.

\chapter{Конституционный кризис 1993 года}

После <<перестройки>> власть сказалось кардинальное различие Горбачёвского курса обновления власти и советской системы, которые принципиально сосуществовать не могли, вследствие ложных посылов перестройки Горбачёва.

У власти оказалось две ветви: президент Ельцин с Советом министров, с другой -- руководство Верховного Совета и большая часть народных депутатов во главе с Р. И. Хасбулатовым, а также вице-президент России А. В. Руцкой и некоторые другие представители законодательной власти.

Противостояние мотивировалось различиями в представлениях сторон конфликта о реформировании конституционного устройства, о новой Конституции, а также о путях социально-экономического развития России. Президент выступал за скорейшее принятие новой Конституции, усиление президентской власти и либеральные экономические реформы, Верховный Совет и Съезд -- за сохранение всей полноты власти у Съезда народных депутатов (до принятия Конституции), и против излишней поспешности, необдуманности и злоупотреблений («Шоковая терапия») при проведении радикальных экономических реформ. Сторонники Верховного Совета опирались на действовавшую Конституцию, согласно ст. 104 которой высшим органом государственной власти являлся Съезд народных депутатов. По мнению же президента Ельцина, в том, что президент клялся соблюдать Конституцию, но при этом его права были Конституцией ограничены, заключалась «двусмысленность» Конституции.

Надо заметить, что при подобных прецедентах в мировой истории постулировалось первенство конституции. Но Ельцин решил иначе.

Ссылаясь на невозможность продолжения сотрудничества с законодательной властью, ставшей, по мнению президента Бориса Ельцина, в условиях экономического кризиса препятствием на пути экономических реформ, и превращение Верховного совета в «штаб неконструктивной оппозиции», занимающийся политической борьбой, им был издан указ № 1400 «О поэтапной конституционной реформе в Российской Федерации», предписывавший высшему органу государственной власти Российской Федерации -- Съезду народных депутатов, а также постоянно действующему законодательному органу -- Верховному Совету прекратить свою деятельность. Президент предложил депутатам вернуться на работу в те учреждения, где они трудились до своего избрания, и принять участие в выборах в новый законодательный орган -- Федеральное собрание.

Конституционный суд Российской Федерации, собравшись на экстренное заседание, пришёл к заключению, что данный указ в двенадцати местах нарушает российскую Конституцию и является основанием для проведения в отношении президента Ельцина процедуры импичмента согласно статье 121.10 конституции или для немедленного прекращения его полномочий с момента подписания указа № 1400 согласно статье 121.6 конституции. Верховный Совет и Съезд народных депутатов отказались подчиниться данному указу президента, квалифицировали его действия как государственный переворот и на основании статей 121.6 и 121.11 конституции, \textbf{констатировали прекращения полномочий президента Ельцина с момента издания указа № 1400 и переход их к вице-президенту Руцкому}.

До этого, весной 1993 года, Съезд уже предпринимал несколько попыток объявить импичмент, однако в то время они ещё не нашли достаточной поддержки. Съезд народных депутатов также инициировал референдум, который продемонстрировал доверие большинства проголосовавших президенту Российской Федерации Борису Ельцину.

Верховным Советом было принято решение о досрочном созыве X Чрезвычайного Съезда народных депутатов (ранее он был запланирован на 17 ноября 1993 года). Частям милиции (генерал-лейтенант милиции В. Панкратов), подчинившимся Борису Ельцину и мэру Москвы Юрию Лужкову, был отдан приказ о блокаде Дома Советов.

Оборону Дома Советов возглавили и. о. президента России Александр Руцкой, председатель Верховного Совета Руслан Хасбулатов и назначенные Руцким министр обороны Владислав Ачалов и его заместитель Альберт Макашов.

1 октября была предпринята попытка мирных переговоров при посредничестве Патриарха Московского и Всея Руси Алексия Второго, в ходе которых в ночь на 2 октября было подписано соглашение -- протокол № 1 между руководителями палат Верховного Совета с одной стороны и представителями верного Ельцину правительства и его администрации -- с другой -- о проведении учёта и сдачи на хранение всего оружия, находившегося в том числе и в руках у случайных лиц, оборонявших Дом Советов. После подписания протокола № 1 в здание было подано электричество и пропущены несколько сотен журналистов, смягчён пропускной режим и был обеспечен свободный выход для всех желающих. Однако после вмешательства председателя Верховного Совета Руслана Хасбулатова, руководствовавшегося, на взгляд аналитиков, личными амбициями, около полудня 2 октября съезд народных депутатов денонсировал это соглашение и переговоры были прекращены.

3 октября, после многочисленных уличных столкновений с подразделениями ОМОНа, милиции и внутренних войск, демонстранты -- сторонники Верховного Совета прорвали блокаду из сотрудников правоохранительных органов возле Дома Советов. Затем по приказу Руцкого под непосредственным руководством назначенного им заместителя министра обороны генерал-полковника Альберта Макашова захватили здание московской мэрии (бывшее здание СЭВ, из окон которого обстреливались демонстранты). Захват мэрии прошёл без жертв среди штурмующих. У одного из зданий телецентра Останкино -- АСК-3 защитники Верховного Совета после двухчасового митинга перешли к решительным действиям: вход в здание был проломлен имевшимися у демонстрантов военными грузовиками. Затем, не применяя оружие, Макашов и его охрана вошли в это здание с целью провести переговоры о предоставлении эфира. Когда спецназ навел на Макашова лазерный прицел, он вместе с охраной покинул здание телецентра. Выстрелом из АСК-3 был ранен охранник Макашова. После чего, по версии следственной группы Генеральной прокуратуры Российской Федерации, у проломленных дверей АСК-3 произошёл взрыв, ошибочно принятый за выстрел из имевшегося у демонстрантов гранатомёта, а по версии командира спецназа «Витязь» -- был действительно произведён выстрел из гранатомёта сторонниками Верховного Совета, в результате которого погиб один из защитников здания, спецназовец Ситников. После гибели спецназовца бойцы верных Ельцину подразделений МВД открыли по демонстрантам и прохожим огонь на поражение. В 2005 году телеканал НТВ показал видеозапись, из которой следует, что взрыв у входа АСК-3 произошел после того, как был сделан выстрел из гранатомета из противоположного здания телецентра -- АСК-1, куда сторонники Съезда и парламента не проникали и который контролировался военнослужащими внутренних войск, подчинявшиеся Ельцину.

4 октября в результате штурма и танкового обстрела Белый дом был взят под контроль войсками. В ходе событий 3--4 октября погибло, по данным следствия, 123 человека и 384 было ранено. Перестала существовать система Советов, радикально изменилась система власти в России: вместо советской на период до принятия конституции была установлена президентская республика, после вступления в силу новой конституции -- президентско-парламентская. В 1994 году арестованные участники октябрьских событий были амнистированы Государственной Думой Федерального Собрания Российской Федерации, хотя ни один из них не был осуждён.

\chapter{Новая конституция 1993 года}

Конституция 1993 года окончательно ликвидировала систему Советов и провела ряд других изменений, уничтоживших линию власти, противостоящую Ельцину.

\chapter{Недостатки конституции 1993 года}

С 1993 года согласно Конституции Россия является вассалом других стран. Евгений Фёдоров рассказывает о первых шагах по восстановлению суверенитета России. Начинать нужно с права на собственную идеологию и собственную госполитику то есть статей 13 и 15 Конституции. После референдума по этим статьям Конституции уже намного легче будет бороться с иностранным управлением (с пятой колонной) и можно будет менять другие статьи Конституции.
Предлагается в пункте 2 статьи 13 Конституции исключить слова «государственной или».
В итоге этот пункт 2 статьи 13 будет выглядеть так: «Никакая идеология не может устанавливаться в качестве обязательной.» 
Убрав запрет на гос.идеологию Россия обретает право на выработку собственного лица, собственного характера и из искусственного младенчества может активно приступить к гос.строительству. У чиновников появятся чёткие ориентиры государственной политики, появятся понятия что хорошо, а что плохо. С гос.идеологией гражданскому обществу намного проще будет привлечь чиновника к ответственности. Надеюсь, что намного проще будет ставить и реализовывать задачи развития России.
Предлагается пункте 4 статьи 15 Конституции исключить слова «Общепризнанные принципы и нормы международного права и»
В итоге этот пункт 4 статьи 15 будет выглядеть так: «Международные договоры Российской Федерации являются составной частью ее правовой системы. Если международным договором Российской Федерации установлены иные правила, чем предусмотренные законом, то применяются правила международного договора.» 
Вроде опять не существенно, однако это прорыв: Россия сможет самостоятельно формировать свои ценности, строить свою систему координат нравственности и отношения к патриотизму. Ведь чиновник без чувства патриотизма равнозначен священнику делающему обряды, но не имеющего веры в себе.


казывается, самые большие беды в России - не те, о которых вы подумали, а тирания Запада и отсутствие государственной идеологии. Депутат Евгений Фёдоров предлагает внести поправки в две статьи Конституции. Из статьи 13, по мнению единоросса, надо упразднить пункт 2: <<Никакая идеология не может устанавливаться в качестве государственной или обязательной>>. Из статьи 15 необходимо изъять пункт 4: <<Общепризнанные принципы и нормы международного права и международные договоры Российской Федерации являются составной частью ее правовой системы. Если международным договором Российской Федерации установлены иные правила, чем предусмотренные законом, то применяются правила международного договора>>.

\chapter*{Заключение}
\addcontentsline{toc}{chapter}{Заключение}%\tabularnewline


\begin{thebibliography}{99}
\addcontentsline{toc}{chapter}{Литература}

\bibitem{nodejs}

\bibitem{iii}

\end{thebibliography}
\end{document}