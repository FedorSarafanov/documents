\documentclass[a4paper,12pt]{article}
 
\usepackage{extsizes}
\usepackage{cmap}
\usepackage[T2A]{fontenc}
\usepackage[utf8]{inputenc}
\usepackage[russian]{babel}
\usepackage{cyrtimes}

%%%%%%%%%%%%%%%%%%%%%%%%%%%%%%%%%%%%%%%%%%%%%%%%%%%%%%%%%%%%%%%%%%%%%%%%%%%%%%%%%%  
\usepackage{graphicx} % для вставки картинок
\graphicspath{{img/}}
\usepackage{amssymb,amsfonts,amsmath,amsthm} % математические дополнения от АМС
\usepackage{indentfirst} % отделять первую строку раздела абзацным отступом тоже
\usepackage[usenames,dvipsnames]{color} % названия цветов
\usepackage{makecell}
\usepackage{multirow} % улучшенное форматирование таблиц
\usepackage{ulem} % подчеркивания
\linespread{1.3} % полуторный интервал
\renewcommand{\rmdefault}{ftm} % Times New Roman
\frenchspacing
\usepackage{geometry}
\geometry{left=3cm,right=1cm,top=2cm,bottom=2cm,bindingoffset=0cm}
\usepackage{titlesec}
% \definecolor{black}{rgb}{0,0,0}
% \usepackage[colorlinks, unicode, pagecolor=black]{hyperref}
\usepackage[unicode]{hyperref}
\usepackage{misccorr} % Точки после нумерации разделов
\usepackage{fancyhdr} %загрузим пакет
\pagestyle{fancy} %применим колонтитул
\fancyhead{} %очистим хидер на всякий случай
\fancyhead[LE,RO]{Сарафанов Ф.Г.} %номер страницы слева сверху на четных и справа на нечетных
\fancyhead[CO, CE]{Конспект по алгебре \today}
\fancyhead[LO,RE]{Начала теорвера} 
\fancyfoot{} %футер будет пустой
\fancyfoot[CO,CE]{\thepage}
\renewcommand{\labelenumii}{\theenumii)}
\begin{document}

\section{Начала теории вероятностей}
\subsection{Несовместные события}

\textbf{Определение 1}\\
События называют \textit{несовместными}, если они не могут происходить одновременно в одном и том же эксперименте.

\textbf{Пример 1}\\
События двух бросков одной игральной кости несовместны, так как не может одновременно выпасть, например, 1 и 6.

\textbf{Теорема 1}\\
Вероятность суммы двух несовместных событий $A$ и $B$ (Появление хотя бы одного события) равна сумме вероятностей этих событий.
$$P(A+B)=P(A)+P(B)$$

\textbf{Следствие из теоремы 1}\\
Сумма вероятностей противоположных событий равна 1.
$$P(A)+P(\bar{B})=1$$

\subsection*{Задача 1}
Курсант сдаст зачёт по стрельбе, если получит оценку не ниже <<4>>. Какова вероятность сдачи зачёта, если курсант получает за стрельбу оценку <<5>> с вероятностью 0.3, а оценку <<4>>  с вероятностью 0.6?
\begin{center}
Решение
\end{center}

Пусть событие $A$ $\rightarrow$ <<5>>, событие $B$ $\rightarrow$ <<4>>, событие $C$ --- сдача зачёта.

Так как $A$ и  $B$ --- несовместные события, применим теорему 1:
$$P(С)=P(A+B)=P(A)+P(B)=0.3+0.6=0.9$$


\subsection*{Задача 2}
Вероятность того, что новый электрический чайник прослужить больше года, равна 0.97. Вероятность того, что он же прослужит больше двух лет, 0.89. Найдите вероятность того, что чайник прослужит больше года, но меньше двух.

\begin{picture}(400,80)
% Оси координат:
\put(0,20){\vector(1,0){200}}
\put(200,7){$t$}
\put(206,20){\oval(210,40)[tl]}
\put(98,7){2}
\put(78,7){1}
\put(206,20){\oval(250,70)[tl]}
\put(210,34){$P(B)$}
\put(210,53){$P(A)$}
\end{picture}

Вероятность на отрезке [1; 2] есть искомая вероятность $P(C)$. Действительно:
$$P(C)+P(B)=P(A)\Rightarrow{}P(C)=P(A)-P(B)=0.97-0.89=0.08$$

\subsection{Cовместные события}

\section*{Геометрическая вероятность}
Вероятность $P$ события $A$ --- $P(A)$ есть отношение меры $A$: длины, площади, объема к мере $Y$ -- пространства элементарных событий.

\subsection{Задача}
В круг радиуса $R$ случайным образом бросают точку. Найдите вероятность того, что это точка окажется внутри вписанного:
\begin{enumerate}
	\item правильного треугольника
	\item квадрата
	\item правильного шестиугольника
\end{enumerate}
\begin{center}
Решение
\end{center}



\begin{enumerate}
	\item { Пусть $a$-сторона правильного треугольника, $h$ --- его высота, а $B$ - вершина треугольника, противолежащая высоте. Тогда
		\begin{align*}
			n=S_{\text{окр}}=\pi{}R^2\\
			S_{\text{пр.}\vartriangle}&=\frac{1}{2}ah\\
			h=\sqrt{a^2-(\frac{1}{2}a^2)}&=\frac{a\sqrt{3}}{2}\\
			S_{\text{пр.}\vartriangle}=\frac{1}{2}a\frac{a\sqrt{3}}{2}&=\frac{a^2\sqrt{3}}{4}\\
			BO&=R\\
			BO&=\frac{2}{3}h\\
			BO=\frac{2}{3}*\frac{a\sqrt{3}}{2}&=\frac{a\sqrt{3}}{3}\\
			a=\frac{3R}{\sqrt{3}}&=R\sqrt{3}\\
			S_{\vartriangle}=\frac{(3\sqrt{3})^2\sqrt{3}}{4}&=\frac{3\sqrt{3}R^2}{4}\\
			P(A)=\frac{3\sqrt{3}R^2}{4\pi{}R^2}=\frac{3\sqrt{3}}{4\pi}&=0.41
		\end{align*}
	}
	\item { Пусть $a$-сторона квадрата, $h$ --- его высота. Тогда
		\begin{align*}
			n&=S_{\text{окр}}=\pi{}R^2\\
			m&=a^2\\
			R&=\frac{a\sqrt{2}}{2}\\
			a&=R\sqrt{2}\\
			S_\square&=a^2=(R\sqrt{2})^2=2R^2\\
			P(A)&=\frac{m}{n}=\frac{2R^2}{\pi{}R^2}\approx\frac{2}{3}=0.04
		\end{align*}
	}	
	\item { Пусть $a$-сторона правильного шестиугольника. Тогда
		\begin{align*}			
			n&=S_{\text{окр}}=\pi{}R^2\\
			R&=a\\
			m&=S_{\text{шестиугольника}}=\frac{3R^2\sqrt{3}}{2}\\
			P(A)&=\frac{m}{n}=\frac{3R^2\sqrt{3}}{2\pi{}R^2}=\frac{3\sqrt{3}}{2\pi}\approx0.63
		\end{align*}
	}		
\end{enumerate}

\subsection{Задача}
Случайным образом выбирается одно из решений неравенства $x^2\leq9$, найдите вероятность того, что оно является решением неравенства:

\begin{enumerate}
	\item $x^2\leq10$
	\item $2x-3\leq17$
	\item $x^2\geq10$
	\item $x^3+2x\geq0$
\end{enumerate}

\begin{center}
Решение
\end{center}

\begin{enumerate}
	\item { 
		\begin{align*}
			x^2&\leq10\\
			x&\in[-3;3]\\
			x^2&\leq9\\
			x&\in[-\sqrt{10};\sqrt{10}]\longrightarrow\\
			\longrightarrow& P(A)=1
		\end{align*}
	}
	\item { 
		\begin{align*}
			2x-3&\leq17\\
			x&\leq10\\
			x&\in[-3;3]\longrightarrow\\
			\longrightarrow& P(A)=1
		\end{align*}
	}	
	\item { 
		\begin{align*}			
			x^2&\geq10\\
			x&\geq\sqrt{10}\\
			x&\leq-\sqrt{10}\\
			x&\in[-3;3]\longrightarrow\\
			\longrightarrow& P(A)=0
		\end{align*}
	}	
	\item { 
		\begin{align*}			
			x^3+2x&\geq0\\
			x&\geq0\\
			x&\in[-3;3]\longrightarrow\\			
			\longrightarrow& P(A)=\frac{[0; 3]}{[-3; 3]}=\frac{3}{6}=0.5
		\end{align*}
	}			
\end{enumerate}

\end{document}