\documentclass[a4paper,12pt]{article}
 
\usepackage{extsizes}
\usepackage{cmap}
\usepackage[T2A]{fontenc}
\usepackage[utf8]{inputenc}
\usepackage[russian]{babel}
\usepackage{cyrtimes}

%%%%%%%%%%%%%%%%%%%%%%%%%%%%%%%%%%%%%%%%%%%%%%%%%%%%%%%%%%%%%%%%%%%%%%%%%%%%%%%%%%  
\usepackage{graphicx} % для вставки картинок
\graphicspath{{img/}}
\usepackage{amssymb,amsfonts,amsmath,amsthm} % математические дополнения от АМС
\usepackage{indentfirst} % отделять первую строку раздела абзацным отступом тоже
\usepackage[usenames,dvipsnames]{color} % названия цветов
\usepackage{makecell}
\usepackage{multirow} % улучшенное форматирование таблиц
\usepackage{ulem} % подчеркивания
\linespread{1.3} % полуторный интервал
\renewcommand{\rmdefault}{ftm} % Times New Roman
\frenchspacing
\usepackage{geometry}
\geometry{left=3cm,right=1cm,top=2cm,bottom=2cm,bindingoffset=0cm}
\usepackage{titlesec}
% \definecolor{black}{rgb}{0,0,0}
% \usepackage[colorlinks, unicode, pagecolor=black]{hyperref}
\usepackage[unicode]{hyperref}
\usepackage{misccorr} % Точки после нумерации разделов
\usepackage{fancyhdr} %загрузим пакет
\pagestyle{fancy} %применим колонтитул
\fancyhead{} %очистим хидер на всякий случай
\fancyhead[LE,RO]{Сарафанов Ф.Г.} %номер страницы слева сверху на четных и справа на нечетных
\fancyhead[CO, CE]{Конспект по алгебре \today}
\fancyhead[LO,RE]{Начала теорвера} 
\fancyfoot{} %футер будет пустой
\fancyfoot[CO,CE]{\thepage}
\renewcommand{\labelenumii}{\theenumii)}
\begin{document}

\section{Начала теории вероятностей}
\subsection{Несовместные события}

\textbf{Определение 1}\\
События называют \textit{несовместными}, если они не могут происходить одновременно в одном и том же эксперименте.

\textbf{Пример 1}\\
События двух бросков одной игральной кости несовместны, так как не может одновременно выпасть, например, 1 и 6.

\textbf{Теорема 1}\\
Вероятность суммы двух несовместных событий $A$ и $B$ (появление хотя бы одного события) равна сумме вероятностей этих событий.
$$P(A+B)=P(A)+P(B)$$

\textbf{Следствие из теоремы 1}\\
Сумма вероятностей противоположных событий равна 1.
$$P(A)+P(\bar{B})=1$$

\subsection*{Задача 1}
Курсант сдаст зачёт по стрельбе, если получит оценку не ниже <<4>>. Какова вероятность сдачи зачёта, если курсант получает за стрельбу оценку <<5>> с вероятностью 0.3, а оценку <<4>>  с вероятностью 0.6?
\begin{center}
Решение
\end{center}

Пусть событие $A$ $\rightarrow$ <<5>>, событие $B$ $\rightarrow$ <<4>>, событие $C$ --- сдача зачёта.

Так как $A$ и  $B$ --- несовместные события, применим теорему 1:
$$P(С)=P(A+B)=P(A)+P(B)=0.3+0.6=0.9$$


\subsection*{Задача 2}
Вероятность того, что новый электрический чайник прослужить больше года, равна 0.97. Вероятность того, что он же прослужит больше двух лет, 0.89. Найдите вероятность того, что чайник прослужит больше года, но меньше двух.
\begin{center}
Решение
\end{center}
\begin{picture}(400,80)
% Оси координат:
\put(0,20){\vector(1,0){200}}
\put(200,7){$t$}
\put(206,20){\oval(210,40)[tl]}
\put(98,7){2}
\put(78,7){1}
\put(206,20){\oval(250,70)[tl]}
\put(210,34){$P(B)$}
\put(210,53){$P(A)$}
\end{picture}

Вероятность на отрезке [1; 2] есть искомая вероятность $P(C)$. Действительно:
$$P(C)+P(B)=P(A)\Rightarrow{}P(C)=P(A)-P(B)=0.97-0.89=0.08$$

\subsection{Cовместные события}

\textbf{Определение 2}\\
События называют \textit{совместными}, если они могут произойти одновременно.

\textbf{Теорема 2}\\
Вероятность суммы двух совместных событий $A$ и $B$ (появление хотя бы одного события) равна сумме их вероятностей без вероятности их совместного появления.
$$P(A+B)=P(A)+P(B)-P(AB)$$

\subsection*{Задача 3}
В ТЦ два одинаковых кофейных автомата продают кофе. Вероятность того, что к концу дня кончится кофе в автомате, равна 0.3.  Вероятность того, что к концу дня кончится кофе в обоих автоматах, равна 0.12. Найдите вероятность того, что кофе к вечеру останется в обоих автоматах.
\begin{center}
Решение
\end{center}
\begin{align*}
P(\overline{AB})&=0.3+0.3-0.12=0.48\\
P(AB) &= 1-0.48=0.52    
\end{align*}

\end{document}