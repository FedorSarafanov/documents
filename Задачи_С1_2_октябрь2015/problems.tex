\documentclass[a4paper,12pt]{article}
 
\usepackage{extsizes}
\usepackage{cmap}
\usepackage[T2A]{fontenc}
\usepackage[utf8]{inputenc}
\usepackage[russian]{babel}
\usepackage{cyrtimes}
\usepackage{misccorr}

%%%%%%%%%%%%%%%%%%%%%%%%%%%%%%%%%%%%%%%%%%%%%%%%%%%%%%%%%%%%%%%%%%%%%%%%%%%%%%%%%%  
\usepackage{graphicx} % для вставки картинок
\graphicspath{{img/}}
\usepackage{amssymb,amsfonts,amsmath,amsthm} % математические дополнения от АМС
\usepackage{indentfirst} % отделять первую строку раздела абзацным отступом тоже
\usepackage[usenames,dvipsnames]{color} % названия цветов
\usepackage{makecell}
\usepackage{multirow} % улучшенное форматирование таблиц
\usepackage{ulem} % подчеркивания
\linespread{1.3} % полуторный интервал
\renewcommand{\rmdefault}{ftm} % Times New Roman
\frenchspacing
\usepackage{geometry}
\geometry{left=3cm,right=2cm,top=3cm,bottom=3cm,bindingoffset=0cm}
\usepackage{titlesec}
% \definecolor{black}{rgb}{0,0,0}
% \usepackage[colorlinks, unicode, pagecolor=black]{hyperref}
\usepackage[unicode]{hyperref}
\usepackage{fancyhdr} %загрузим пакет
\pagestyle{fancy} %применим колонтитул
\fancyhead{} %очистим хидер на всякий случай
\fancyhead[LE,RO]{Сарафанов Ф.Г.} %номер страницы слева сверху на четных и справа на нечетных
\fancyhead[CO, CE]{С1 на тестовом ЕГЭ по алгебре профильного уровня}
\fancyhead[LO,RE]{\today} 
\fancyfoot{} %футер будет пустой
\fancyfoot[CO,CE]{\thepage}
\renewcommand{\labelenumii}{\theenumii)}

\usepackage{amsthm}
\newtheorem{define}{Определение}
\newtheorem{theorem}{Теорема}
\newtheorem{problem}{Задача}

\begin{document}


\begin{problem}
а) Решите уравнение
$$\sqrt{7-8sin\,x}=-2cos\,x$$
б) Укажите корни этого уравнения, принадлежащие отрезку $\left[-\frac{3\pi}{2};\ \frac{3\pi}{2}\right]$
\end{problem}
\newpage
\begin{proof}[Решение a)]
$$\sqrt{7-8sin\,x}=-2cos\,x$$
\begin{equation*}
	\begin{cases}
		7-8sin\,x=4cos^2\,x\\
		-2cos\,x\ge0
	\end{cases}
	\begin{cases}
		7-8sin\,x=4-4sin^2\,x\\
		-2cos\,x\le0
	\end{cases}	
	\begin{cases}
		4sin^2-8sin\,x+3=0\\
		cos\,x\le0
	\end{cases}		
\end{equation*}
\begin{equation*}
	\begin{cases}
		sin\,x=\frac{3}{2}\text{ и }sin\,x=\frac{1}{2}\\
		cos\,x\le0
	\end{cases}	
	\begin{cases}
		sin\,x=\frac{1}{2}\\
		cos\,x\le0
	\end{cases}	
\end{equation*}
\begin{equation*}
	\begin{cases}
		x=\frac{\pi}{6}+2\pi{}n, n\in{}Z\text{ и }x=\frac{5\pi}{6}+2\pi{}n, n\in{}Z\\
		\frac{pi}{2}+2\pi{}n\le\frac{3\pi}{2}+2\pi{}n, n\in{}Z
	\end{cases}	
	x=\frac{5\pi}{6}+2\pi{}n,n\in{}Z
\end{equation*}
\end{proof}
\begin{proof}[Решение б)]
Укажем все корни этого уравнения, принадлежащие промежутку $\left(-\frac{3\pi}{2};\ \frac{3\pi}{2}\right)$. Решим двойное неравенство:
$$\sqrt{7-8sin\,x}=-2cos\,x$$
$$
-\frac{3\pi}{2}\textless \frac{5\pi}{6} + 2\pi{}n\textless \frac{3\pi}{2}$$$$
-\frac{3}{2}\textless \frac{5}{6} + 2{}n\textless \frac{3}{2}\\
-9\textless5+12n\textless9$$$$
-14\textless12n\textless4$$$$
-\frac{7}{6}\textless{n}\textless\frac{1}{3}$$
При $n=-1 \rightarrow x_1=\frac{5\pi}{6}-2\pi=-\frac{7\pi}{6}$\\
При $n=0 \rightarrow x_2=\frac{5\pi}{6}$\\
\end{proof}
\begin{proof}[Ответ]
а)	$x=\frac{5\pi}{6}+2\pi{}n,n\in{}Z$\\ б)	$-\frac{7\pi}{6}; \frac{5\pi}{6}$
\end{proof}

\newpage
\begin{problem}
а) Решите уравнение
$$ctg^2\,x+2\sqrt{3}ctg\,x+3sin^2\,x=-3sin^2(x-\frac{3\pi}{2})$$
б) Укажите корни этого уравнения, принадлежащие отрезку $\left[-\frac{11\pi}{4};\ -4\pi\right]$
\end{problem}
\newpage
\begin{proof}[Решение a)]
$$ctg^2\,x+2\sqrt{3}ctg\,x+3sin^2\,x=-3sin^2(x-\frac{3\pi}{2})$$
$$ctg^2\,x+2\sqrt{3}ctg\,x+3sin^2\,x=-3cos^2\,x$$
$$ctg^2\,x+2\sqrt{3}ctg\,x+3sin^2\,x-3cos^2\,x=0$$
$$ctg^2\,x+2\sqrt{3}ctg\,x+3(sin^2\,x+cos^2\,x)=0$$
$$ctg^2\,x+2\sqrt{3}ctg\,x+3=0$$
$$(ctg\,x+\sqrt{3})^2=0$$
$$ctg\,x+\sqrt{3}=0$$
$$ctg\,x=-\sqrt{3}$$
$$x=-\frac{\pi}{6}+\pi{}{n},\ n\in{Z}$$
\end{proof}
\begin{proof}[Решение б)]
Найдем все решения уравнения на отрезке $\left[-\frac{11\pi}{4};\ -4\pi\right]$. 
$$-4\pi-\frac{\pi}{6}=-\frac{25\pi}{6}$$
$$-5\pi-\frac{\pi}{6}=-\frac{31\pi}{6}$$
\end{proof}
\begin{proof}[Ответ]
а)	$x=\frac{\pi}{6}+\pi{}n,n\in{}Z$\\ б)	$-\frac{31\pi}{6}; -\frac{25\pi}{6}$
\end{proof}



\end{document}