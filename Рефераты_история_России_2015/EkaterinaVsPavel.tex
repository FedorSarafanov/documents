\documentclass[a4paper,12pt]{diss_4}
 
% \usepackage{extsizes}
% \usepackage{cmap}
\usepackage[T2A]{fontenc}
\usepackage[utf8x]{inputenc}
\usepackage[russian]{babel}
% \usepackage{cyrtimes}

%%%%%%%%%%%%%%%%%%%%%%%%%%%%%%%%%%%%%%%%%%%%%%%%%%%%%%%%%%%%%%%%%%%%%%%%%%%%%%%%%%  
\usepackage{graphicx} % для вставки картинок
\graphicspath{{img/}}
\usepackage{amssymb,amsfonts,amsmath,amsthm} % математические дополнения от АМС
\usepackage{indentfirst} % отделять первую строку раздела абзацным отступом тоже
\usepackage[usenames,dvipsnames]{color} % названия цветов
\usepackage{makecell}
\usepackage{multirow} % улучшенное форматирование таблиц
\usepackage{ulem} % подчеркивания
\linespread{1.3} % полуторный интервал
% \renewcommand{\rmdefault}{ftm} % Times New Roman
\frenchspacing
\usepackage{geometry}
% \geometry{left=3cm,right=1cm,top=2cm,bottom=2cm,bindingoffset=0cm}
\usepackage{titlesec}
% \definecolor{black}{rgb}{0,0,0}
% \usepackage[colorlinks, unicode, pagecolor=black]{hyperref}
\usepackage[unicode]{hyperref}
\usepackage{epigraph} %%% to make inspirational quotes.

\begin{document}

% -*- root: project.tex -*-
\begin{titlepage}
\newpage

\begin{center}

Муниципальное автономное образовательное учреждение \\
лицей № 180 \\
г. Нижнего Новгорода \\

% \hrulefill
\end{center}
 
% \flushright{КАФЕДРА № ХХХ}

\vspace{14em}

\begin{center}
\large{Реферат}
\end{center}

% \vspace{.5em}
 
\begin{center}
Сравнение периодов правления Екатерины II и Павла I
\end{center}

\vspace{4.5em}
 
\begin{flushright}
Выполнил: Сарафанов Фёдор, \\
ученик 10 <<А>> класса \\
Научный руководитель: \\
Тереханова Юлия Николаевна, \\
учитель истории и обществознания \\ 
первой квалификационной категории
\end{flushright}
 
\vspace{\fill}

\begin{center}
Нижний Новгород \\
\the\year
\end{center}

\end{titlepage}
\addtocounter{page}{1}

\tableofcontents
%\large

\chapter*{Введение}
\addcontentsline{toc}{chapter}{Введение}

В начале царствования Павел I изменил многие екатерининские порядки, однако по существу внутренняя политика Павла I продолжала курс Екатерины II. 

Продолжая внешнюю политику Екатерины II, Павел I принял участие в коалиционных войнах против Франции. Под давлением союзников -- австрийцев и англичан -- поставил во главе русской армии А. В. Суворова, под командованием которого были совершены героические Итальянский и Швейцарский походы 1799. Однако распри между Павлом I и его союзниками, надежда Павла I на то, что завоевания французской революции будут сведены на нет самим Наполеоном, привели к сближению с Францией. Мелкая придирчивость Павла I, неуравновешенность характера вызывали недовольство среди придворных. Оно усилилось в связи с изменением внешнеполитического курса, нарушавшего торговые связи с Англией. В среде гвардейских офицеров созрел заговор. В ночь с 11 на 12 марта 1801 в Михайловском замке заговорщики убили Павла I.

Что послужило причиной убийства Павла I? Сильны ли различия между политикой Павла и Екатерины? Эти вопросы автор подробно разобрал в данном реферате.

\chapter{Политика Екатерины II}

Взошедшая на престол, свергнув супруга Петра III --- Екатерина II открыла новую эпоху т.н. <<просвещённого абсолютизма>> во внутренней политике государства. 

Догматически придерживаясь европейского гуманизма и просвещения, на деле правление Екатерины II ознаменовалось максимальным закрепощением крестьян и всесторонним расширением дворянских полномочий и привилегий. 

Новая императрица начала правление с уничтожения деятельности Петра III: приостановила секуляризацию церковных имуществ, отменила мирный договор с Пруссией; согласно указу от 11 февраля 1763 г. служба дворян вновь объявлялась обязательной. 

Но ближайшие события показали, что ни о каком разрыве с прежней политикой речи идти не может. В 1764 г. был заключен новый договор с Фридрихом II, и в том же году Екатерина II провела секуляризацию имений духовного ведомства, распространив ее в 1768 г. на территорию Украины; в 1785 г. подтверждено освобождение дворян от обязательной службы. Историк С. С. Татищев в 1776 г. не без удивления отмечал: <<Как ни велико, на первый взгляд, различие в политических системах Петра III и его преемницы, нужно, однако, сознаться, что в нескольких случаях она служила лишь продолжательницей его начинания>>.

Стремясь создать более реальные гарантии просвещенной монархии, Екатерина II начала работать над т.н. <<жалованными грамотами>>. Однако, если Жалованная грамота дворянства была реализована, проект <<Сельского положения>> так и не был доведен до конца. Французская революция в корне изменила отношение Екатерины II к идеям Просвещения. В 1794 г. в одном из писем она заявляет, что деятельность философов ведет лишь к разрушению: <<что бы они ни говорили и ни делали, - продолжала она, - мир никогда не перестанет нуждаться в повелителе... лучше предпочесть безрассудство одного, чем безумие многих, заражающее бешенством двадцать миллионов людей во имя слова <<свобода>>.

В внешнеевропейской политике Екатерина II играла на противостоянии Англии и Франции. Приведем слова историка А. Т. Мэхэна: <<Императрица могла еще рассчитывать тогда на взаимный антагонизм этих держав, а между тем британский флот и образ его действий в войне представляли для России более серьезную опасность, чем французские армии. Но каковы бы ни были ее соображения, нет сомнения в том, что в то время ее политика клонилась на сторону Франции. Представители последней на Востоке являлись посредниками между императрицей и султаном в непрерывных спорах, возникших из-за Кучук-Кайнарджийского договора. С Францией Екатерина заключила торговый трактат на в высшей степени благоприятных условиях, тогда как торговый трактат с Великобританией не был возобновлен по истечении срока ни тогда, ни даже еще в течение многих лет потом>>


Действия Екатерины, как и результат  внешней политики и внутренней -- выделение дворянского сословия, дальнейшее порабощение крепостных крестьян --- так или иначе, привели к усилению России как империи и укреплению престола. Екатерина II завершила жестокий процесс сколачивания империи, начатый Петром I. К концу её правления Россия прочно заняла место в международной политике. Однако, воспользоваться плодами политики ей уже не было суждено. 6 ноября 1769 года, в возрасте 67 лет Екатерина II скончалась. Желание её возвести на трон своего внука -- Александра не сбылось. Престол занял опальный сын Екатерины -- Павел I. 


\chapter{Восхождение на престол Павла I. Реформы после смерти Екатерины II}
\epigraph{\textit{Этому царствованию принадлежит самый блестящий выход России на европейской сцене}}
{В.О.Ключевский}
\epigraph{\textit{Я желал лучше быть ненавидимым за правое дело, чем любимым за дело неправое.}}
{Император Павел I}

6 ноября 1796 года, в возрасте 42 лет император Павел I вступил на престол. После вступления на престол Павел решительно приступил к ломке порядков, заведённых матерью. У современников осталось впечатление, что многие решения принимались <<назло>> её памяти. Питая глубокое отвращение к революционным идеям Павел, к примеру, вернул свободу радикалам Радищеву, Новикову и Костюшко.

Павел успел провести ряд преобразований, направленных на дальнейшую централизацию государственной власти. В частности, изменились функции Сената, были восстановлены некоторые коллегии, упраздненные Екатериной II. В 1798 году вышел указ о создании департамента водных коммуникаций. 4 декабря 1796 года учреждено Государственное казначейство и должность государственного казначея. Утверждённым в сентябре 1800 года <<Постановлением о коммерц-коллегии>> купечеству было дано право выбрать 13 из 23 её членов из своей среды.

Одним из самых важных преобразований стал новый закон о престолонаследии, который подвёл черту под столетием дворцовых переворотов и женского правления в России. За один этот закон нужно быть благодарным Павлу I последующей гораздо менее кровавой истории России вплоть до 1917 года.


Будучи человеком вспыльчивым, безрассудным в гневе, придирчивым к формальностям, Павел I настроил против себя часть дворянства. Однако, всего Павлом I лично сосланных дворян менее 10. Вряд ли только характер его привел к печальному итогу -- убийству дворянским окружением.

Что же сподвигло дворянство на убийство Павла I?

\chapter{Мальтийский орден и убийство Павла I}

На заре правления Павла основным направлением внешней политики виделась борьба с революционной Францией. В 1798 году Россия вступила в антифранцузскую коалицию c Великобританией, Австрией, Турцией, Королевством Обеих Сицилий. По настоянию союзников главнокомандующим русскими войсками был назначен опальный А. В. Суворов. В его ведение также передавались и австрийские войска.

Под руководством Суворова Северная Италия была освобождена от французского господства. В сентябре 1799 года русская армия совершила знаменитый переход через Альпы. Однако уже в октябре того же года Россия разорвала союз с Австрией из-за невыполнения австрийцами союзнических обязательств, а русские войска были отозваны из Европы. Совместная англо-русская экспедиция в Нидерланды обернулась неудачей, в которой Павел винил английских союзников.

Весьма примечательна роль в истории Российской Мальты. Сейчас историей магистрства Павла I нередко пренебрегают в исследованиях, между тем, Мальтийский орден оказал решающее действие в смене направления борьбы со Францией на борьбу с Англией.

С детства Павел Петрович увлекался рыцарством. 23 февраля 1765 года Порошин, младший наставник будущего императора, записал: <<Читал я Его Высочеству историю об ордене мальтийских кавалеров. Изволил он, потом, забавляться и, привязав к кавалерии своей флаг адмиральский, представлять себя кавалером Мальтийским>>.

Мальта занимает выгодное для флота положение в <<сердце>> Средиземного моря. Владение им -- важный геополитический ключ политики. 

7 августа 1797 года русский император провозгласил себя протектором Ордена. Православный Павел I сделал это, хотя данный титул по уставу Ордена не мог иметь некатолический монарх. Такой титул должен был носить император Священной Римской империи и король обеих Сицилий Франциск II. Но Орден согласился с принятием православным монархом титула защитника католического Ордена; католический король обеих Сицилий не стал отчего-то обращать внимания на такие <<мелочи>>.

6 августа 1798 года с подачи графа Литты Павел подписывает акт <<О поступлении острова Мальта под защиту России>>. Тут же царь повелел президенту Академии наук  в издаваемом от академии календаре означить остров Мальту <<губерниею Российской Империи>>. В ноябре 1798 года мальтийские рыцари преподнесли Павлу корону, регалии Великого магистра и титул гроссмейстера. И тут происходит поистине еще одно чудо -- римский папа Пий VI неожиданно соглашается с просьбой Павла I признать его гроссмейстером Ордена. Правда, только устно. Но зато папа назвал Павла <<другом человечества и защитником угнетенных>>.

Император Павел Петрович добился подписания англо-русского союзного договора (18 декабря 1798 года). Англия, Россия и Неаполь заключили еще одно соглашение о Мальте 20 декабря 1798 года -- через день после подписания уже русско-неаполитанского договора. В нем три державы договорились о порядке введения их гарнизонов на остров. Было оговорено, что именно русский гарнизон займет столицу Мальты. Верховная власть на первых порах будет принадлежать военному совету, во главе которого будет стоять представитель именно русского командования. Павел I был убежден, что Англия уже согласилась на все его планы относительно Мальты. 

Однако далее события начинают развиваться по странному сценарию.

Британцы явно затягивали со взятием Мальты. Однако когда Ушаков, наконец, смог отправиться в сторону острова, имея на своих кораблях 2000 солдат будущего гарнизона, он по пути получил приказание царя возвратиться в Черное море. Какова причина неожиданного решения царя? Официальная версия -- обида Павла на союзников за отвратительное и предательское поведение. Да, австрийцы издевались над русским флагом в Италии, приложили максимум усилий, чтобы погубить армию Суворова в Швейцарии, буквально загнав великого полководца в Альпы. 

Не лучшее отношение встречали наши войска и флот и со стороны англичан. Дело было не только в изворотливости Нельсона, но и в безобразном отношении британцев к нашему 17-тысячному корпусу во главе с генерал-лейтенантом Германом, который оказывал помощь англичанам при высадке в Голландии. Но все это не может быть причиной разрыва союзных отношений и последующего практически моментального перехода России на другую сторону геополитических баррикад. 

За действиями Павла I явно стояли более серьезные резоны, чем просто обида. Ему, вероятно, стало известно, что англичанам удалось добиться от неаполитанского короля Фердинанда, которому Россия фактически отвоевала королевство, обязательства никогда не отдавать Мальту какой-либо иностранной державе без согласия на то Англии. 

Это было нарушением договоренностей, имевшихся и с Англией. Павел не сомневался, что Лондон сам собирается аннексировать Мальту, потому что начался зондаж с их стороны, предпринимались попытки предложить Павлу взамен Мальты Корсику. На фоне этой информации русский царь и решает неожиданно быстро увести свою эскадру домой. Получилось, что царь <<как бы>> без явной видимой причины сначала отзывает корабли и только потом начинается формальный разрыв с Англией и Австрией и сближение с Францией.

 Хронология событий такова: примерно через полгода после отплытия нашей эскадры, в августе 1800-го, французский гарнизон Мальты капитулировал. Завладев островом, британцы подняли на нем только свой флаг. Как мы видим, в то время информация приходила с большим опозданием, поэтому и реакция властей России заставила себя немного подождать. Россия била по Англии сразу в нескольких направлениях, и Павел I, которого почти все историки рисуют взбалмошным сумасбродом, ненавидящим все, что связано с его матерью, совершенно спокойно использовал <<заготовку>> Екатерины Великой. 

 В конце августа 1800 года, то есть сразу после захвата британцами Мальты, Россия выступила с декларацией, в которой резко осудила <<английский морской разбой>> и призвала Данию, Швецию и Пруссию повторить то, что поставило англичан в очень сложное положение во время войны Североамериканских колоний за независимость. 

 То есть возобновить вооруженный нейтралитет 1780 года, когда несколько европейских стран сорвали организованную англичанами блокаду Штатов. Только теперь Павел предложил сорвать блокаду Франции. Не давая британцами мешать чужой торговле, Россия начала <<рубить>> и собственно английскую торговлю. 

 23 октября 1800 года был наложен арест на английские суда в портах России. В тот же день граф Растопчин обратился ко всем членам дипломатического корпуса в Петербурге с нотой, обвинявшей Англию в том, что она нарушила конвенцию от 20 декабря 1798 года о Мальте. Через месяц, 19 ноября 1800 года, английским судам был вообще закрыт доступ в русские порты с одновременным введением запрета на вывоз в Англию стратегических материалов (для кораблестроения). Были приостановлены и платежи английским купцам. Русский царь показывал Лондону, что так подло поступать с собой он не позволит.

 Арестованными в наших портах оказалось около 200 британских судов, их команды выслали во внутренние губернии. Когда экипаж двух английских кораблей, стоявших в Нарве, оказал сопротивление русским войскам и, потопив русское судно, ушел в море, Павел приказал сжечь остальные находившиеся в Нарве английские суда. 4 декабря 1800 года Россия заключила морскую конвенцию о возобновлении вооруженного нейтралитета одновременно с Данией и Швецией. Ну а Пруссия закрыла порты, расположенные в устье рек Эльбы и Везера, для английских товаров и оккупировала немецкий Ганновер.

 В ответ англичане наложили арест на все бывшие в английских портах суда и русские, шведские и датские товары (пруссаков, однако, не трогая). К примеру, около 200 шведских кораблей из 450 имеющихся были задержаны в британских гаванях или насильно туда заведены.

 Ситуация на мировой карте стремительно менялась. Мальта неожиданно выходила англичанам боком. За какие-то полгода добрая половина союзников и нейтралов начала активные враждебные действия против Лондона и фактически поставила под сомнение то, ради чего британцы и затевали все свои интриги, -- морское могущество и монополию морской торговли. Но и на этом неприятности Британии не заканчивались. Раз Англия была врагом и России и Франции, это неизбежно и очень быстро привело к сближению позиций Парижа и Петербурга. Русский царь довольно охотно пошел на сближение с революционной Францией, в которой в это время первую скрипку начал играть Наполеон Бонапарт, очень вовремя убежавший из Египта и взявший власть в стране в свои руки.

 Процитирую адмирала Нельсона:  <<Я смотрю на Северную лигу как на дерево, в котором Павел составляет ствол, а шведы и датчане -- ветви. Если мне удастся добраться до ствола и срубить его, то ветви отпадут сами собою; но я могу испортить ветви и все-таки не быть в состоянии срубить дерево, и при этом мои силы необходимо будут уже ослаблены в момент, когда понадобится наибольшее напряжение их. Получить возможность вырезать русский флот было моею целью>>

 В июле 1800 года, то есть до падения Мальты, Наполеон вошел в контакт с Павлом Петровичем и предложил сдать Мальту русским. В качестве жеста доброй воли Бонапарт освободил около 8 тысяч пленных русских солдат. При этом заново их обмундировал и отпустил в Россию вместе с их офицерами, отдав знамена, сказав при этом, что если русский император сочтет нужным, в качестве ответной любезности он <<может потребовать у англичан освобождения такого же числа пленных французов. Если же это не будет признано удобным, то Первый консул надеется, что Император примет освобождение своих солдат за знак особого уважения с его стороны к храбрым русским войскам>>.

 Павел I принял предложение Наполеона и назначил генерала, который должен был командовать освобожденными пленными. Отправить этот отряд планировалось на Мальту. Отдавая такие распоряжения, русский царь еще не знал, что она уже захвачена англичанами. Наложение ареста на английское имущество и корабли в России как раз было неким залогом выполнения британцами своих обещаний и признания ими тех прав, которые Павел имел на остров Мальту как гроссмейстер ордена. В начале 1801 года царь пишет Наполеону письмо, свидетельствовавшее о его в высшей степени дружественном расположении к Франции (но полное ненависти к Англии), извещавшее о его намерении назначить в Париж посла.

 Превосходство англичан на море было ощутимым, поэтому Россия и Франция решают нанести удар по морской державе на суше. Решение логичное и резонное. Бонапарт разрабатывает операцию по отправке воинских контингентов в Индию. Совместный поход наполеоновских и павловских войск в <<жемчужину британской короны>> призван нанести удар в самую уязвимую и важную часть английской колониальной империи. Ответ Британии не заставил себя ждать. 

 В Балтийское море отправляется флот Нельсона. Нельсон, как уже упоминалось раньше, был готов уничтожить русский флот. Однако, ничего нового: как только флот вошёл в Балтийское море, заговорщиками был убит Павел I. Практически одновременно с этими двумя событиями происходит покушение -- бросок бомбы в Наполеона. Совпадение? Не думаю.


\chapter*{Заключение}
\addcontentsline{toc}{chapter}{Заключение}%\tabularnewline

Если проводить черту между Екатериной II и Павлом I, можно отметить, что расхождения их в политике были далеко не велики. Ключевым расхождением был метод усиления государства: если Екатерина предпочитала усиливать позиции выделенного сословия -- дворянства за счет крестьян, то Павел же начал строить жёсткую структуру власти, не завязанную на дворянских фаворитов и вообще сословия в целом. 

Более того, Павел I несколько облегчил позиции крестьянства: манифестом о трёхдневной барщине запретил помещикам отправление барщины по воскресным дням, праздникам и более трёх дней в неделю; отменил разорительную <<хлебную повинность>>; начал льготную продажу соли; из государственных запасов стали продавать хлеб, чтобы сбить высокие цены. Многие другие указы Павла I обеспечили ему добрую память среди крестьянства.

Парадокс: просвещенная либеральная Екатерина II была куда более жестока для большинства поданных своей империи, нежели вымурштрованный император Павел I.

Однако, нельзя не заметить схожий процесс внешней политики. И Екатерина, и Павел вначале придерживались проанглийской позиции; но оба постепенно сменили вектор политики в сторону Франции. Все время Россия вела освободительные войны и присоединение новых земель. Так присоеденены Новороссия, Крым, заселяется Аляска. 

В целом внешняя политика Екатерины Великой, несмотря на семейные разногласия с Павлом, была достойно им продолжена. Если бы не смерть Павла I, мир бы был совсем иным. Не было бы войны 1812 года; был бы Русско-Французский союз.

Полупьяные заговорщики, которые били табакеркой в висок и душили шарфом русского государя, могли придумать себе любые красивые объяснения, почему они это делали. Сейчас за них это делают историки, пытаясь показать убийство как результат жестокой политики в отношении царского двора. Но большое количество фактов заставляет задуматься о том, что это убийство -- результат действий Туманного Альбиона. 


\begin{thebibliography}{99}
\addcontentsline{toc}{chapter}{Литература}

\bibitem{nodejs} В.О.Ключевский -- Собрание сочинений в 9 томах. М.:Мысль, 1987

\bibitem{ajax} Н.В.Стариков -- Геополитика: как это делается. СПб.:БХВ-Петербург, 2014

\bibitem{igit} \href{https://ru.wikipedia.org/wiki/%D0%95%D0%BA%D0%B0%D1%82%D0%B5%D1%80%D0%B8%D0%BD%D0%B0_II}{https://ru.wikipedia.org/wiki/Екатерина-II}

\bibitem{iii} \href{https://ru.wikipedia.org/wiki/%CF%E0%E2%E5%EB_I}{https://ru.wikipedia.org/wiki/Павел-I}

\end{thebibliography}
\end{document}