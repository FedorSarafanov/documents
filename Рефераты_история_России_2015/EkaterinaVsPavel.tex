\documentclass[a4paper,12pt]{diss_4}
 
% \usepackage{extsizes}
% \usepackage{cmap}
\usepackage[T2A]{fontenc}
\usepackage[utf8x]{inputenc}
\usepackage[russian]{babel}
% \usepackage{cyrtimes}

%%%%%%%%%%%%%%%%%%%%%%%%%%%%%%%%%%%%%%%%%%%%%%%%%%%%%%%%%%%%%%%%%%%%%%%%%%%%%%%%%%  
\usepackage{graphicx} % для вставки картинок
\graphicspath{{img/}}
\usepackage{amssymb,amsfonts,amsmath,amsthm} % математические дополнения от АМС
\usepackage{indentfirst} % отделять первую строку раздела абзацным отступом тоже
\usepackage[usenames,dvipsnames]{color} % названия цветов
\usepackage{makecell}
\usepackage{multirow} % улучшенное форматирование таблиц
\usepackage{ulem} % подчеркивания
\linespread{1.3} % полуторный интервал
% \renewcommand{\rmdefault}{ftm} % Times New Roman
\frenchspacing
\usepackage{geometry}
% \geometry{left=3cm,right=1cm,top=2cm,bottom=2cm,bindingoffset=0cm}
\usepackage{titlesec}
% \definecolor{black}{rgb}{0,0,0}
% \usepackage[colorlinks, unicode, pagecolor=black]{hyperref}
\usepackage[unicode]{hyperref}
\usepackage{epigraph} %%% to make inspirational quotes.

\begin{document}

\include{titlepage_EVsP}
\addtocounter{page}{1}

\tableofcontents
%\large

\chapter*{Введение}
\addcontentsline{toc}{chapter}{Введение}


\epigraph{\textit{Я желал лучше быть ненавидимым за правое дело, чем любимым за дело неправое.}}
{Император Павел I}

\epigraph{\textit{Этому царствованию принадлежит самый блестящий выход России на европейской сцене}}
{В.О.Ключевский}
 

Недаром П.И.Ключевский называет правление Павла I самам блестящим выходом России на междунарондную арену.

В начале царствования П. I изменил многие екатерининские порядки, однако по существу внутренняя политика П. I продолжала курс Екатерины II. Напуганный Великой французской революцией и непрекращающимися крестовыми выступлениями в России, П. I проводил политику крайней реакции. Была введена строжайшая цензура, закрыты частные типографии (1797), запрещен ввоз иностранных книг (1800), введены чрезвычайные полицейские меры для преследования передовой общественной мысли. В условиях обострявшегося кризиса феодальной системы П. I отстаивал интересы помещиков-крепостников, роздал им более 600 тыс. крестьян. В борьбе против крестовых выступлений использовал карательные экспедиции и некоторые законодательные акты, якобы ограничивавшие эксплуатацию крестьянства, такие, как указ 1797 о трёхдневной барщине. Ввёл централизацию и мелочную регламентацию во всех звеньях государственного аппарата. Провёл реформы в армии по прусскому образцу, вызвавшие недовольство многих офицеров и генералов. В своей деятельности П. I опирался на фаворитов-временщиков А. А. Аракчеева и И. П. Кутайсова. 

Продолжая внешнюю политику Екатерины II, П. I принял участие в коалиционных войнах против Франции. Под давлением союзников — австрийцев и англичан — поставил во главе русской армии А. В. Суворова, под командованием которого были совершены героические Итальянский и Швейцарский походы 1799. Однако распри между П. I и его союзниками, надежда П. I на то, что завоевания французской революции будут сведены на нет самим Наполеоном, привели к сближению с Францией. Мелкая придирчивость П. I, неуравновешенность характера вызывали недовольство среди придворных. Оно усилилось в связи с изменением внешнеполитического курса, нарушавшего торговые связи с Англией. В среде гвардейских офицеров созрел заговор. В ночь с 11 на 12 марта 1801 в Михайловском замке заговорщики убили П. I. 
\chapter{Внутренняя политика Екатерины II}
\chapter{Внешняя политика Екатерины II}
\chapter{Внутренняя политика Павла I. Реформы после смерти Екатерины II}
\chapter{Внешняя политика Павла I}
\chapter{Мальтийский орден и убийство Павла I}



\chapter*{Заключение}
\addcontentsline{toc}{chapter}{Заключение}%\tabularnewline


\begin{thebibliography}{99}
\addcontentsline{toc}{chapter}{Литература}

\bibitem{nodejs} Машков А.В. <<Основы лечебной физической культуры>>

\bibitem{ajax} Васильев В.Е. <<Лечебная физическая культура>>

\bibitem{igit} Аксельрод С.Л.  <<Спорт и здоровье>>

\bibitem{iii} \href{http://www.gpsies.com/mapUser.do?username=osabio}{http://www.gpsies.com/mapUser.do?username=osabio}

\end{thebibliography}
\end{document}