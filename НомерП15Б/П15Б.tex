\documentclass[a4paper]{article}
\usepackage{amssymb,amsmath}
\usepackage[active]{srcltx} %SRC Specials for DVI Searching
\usepackage[utf8]{inputenc}
\usepackage{geometry}
%\geometry{a5paper}
%\usepackage{cyrtimes}
%\usepackage{mathtext,afterpage}
\usepackage[russian]{babel}
\usepackage{graphicx}
% \usepackage[pdf]{pstricks}
% \usepackage{auto-pst-pdf}
% \usepackage{pdfsync}
\pagestyle{empty} 

\begin{document}

$${(sin\ x-cos\ x)}^2=1$$
$$sin\ x-cos\ x=1$$
Явно, что это возможно только при $sin\ x=0, cos\ x=-1$
$$sin\ x-cos\ x=-1$$
Явно, что это возможно только при $sin\ x=0, cos\ x=1$.

Отсюда
\begin{equation}
\begin{cases}
	\begin{cases}
		sin\ x=0\\
		cos\ x=-1
	\end{cases}\\
	\begin{cases}
		sin\ x=0\\
		cos\ x=1
	\end{cases}	
\end{cases}	
\end{equation}
или перепишем в виде, учитывая что $cos\ x$ изначально предпологался положительным (иначе начальное уравнение не имеет смысла, сводясь к $sin\ 2x=2$)
\begin{equation}
\begin{cases}
		sin\ x=0\\
		cos\ x=1
\end{cases}	
\end{equation}
Отсюда
\begin{equation}
\begin{cases}
		x=\pi{k}, k\in{Z}\\
		x=2\pi{k}, k\in{Z}
\end{cases}	
\end{equation}
Отсюда и ответ: $x=2\pi{k}, k\in{Z}$.
\end{document}
