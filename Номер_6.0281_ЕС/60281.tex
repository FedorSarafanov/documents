\documentclass[a4paper]{article}
\usepackage{amssymb,amsmath}
\usepackage[active]{srcltx} %SRC Specials for DVI Searching
\usepackage[utf8]{inputenc}
\usepackage{geometry}
%\geometry{a5paper}
%\usepackage{cyrtimes}
%\usepackage{mathtext,afterpage}
\usepackage[russian]{babel}
\usepackage{graphicx}
% \usepackage[pdf]{pstricks}
% \usepackage{auto-pst-pdf}
% \usepackage{pdfsync}
\pagestyle{empty} 

\begin{document}

\begin{equation}
	y(x)=-\frac{3}{\sqrt{2x^2-x-1}}
\end{equation}

Возьмем производную сложной функции $y(x)$. Заметим, что $(\frac{1}{x})'=-\frac{1}{x^2}$
, а $(\sqrt{x})'=\frac{1}{2\sqrt{x}}$. Тогда $(\frac{1}{\sqrt{x}})'=-\frac{1}{\sqrt{x}^2}\cdot\frac{1}{2\sqrt{x}}=\frac{-1}{2x\sqrt{x}}$, и 
\begin{equation}
	y(x)'=\frac{3}{2x\sqrt{x}}\cdot(2x^2-x-1)'=\frac{3}{2x\sqrt{x}}\cdot(4x-1)
\end{equation}

Тогда производная будет равна

\begin{equation}
	y(x)'=\frac{3(4x-1)}{2(2x^2-x-1)\sqrt{2x^2-x-1}}
\end{equation}

Для полученной производной нетрудно найти критические точки, где производная не существует: 

	$$2x^2-x-1=0$$
	$$x=-\frac{1}{2}, x=1$$

В этих точках $2x^2-x-1=0$ и происходит деление на ноль, но и сама функция в этих точках не существует, поэтому рассматривать их не будем.

Кроме того, производная равна нулю, когда числитель равен нулю, а знаменатель нет:

	$$3(4x-1)=0$$
	$$x=\frac{1}{4}$$

Но в этой точке функция также не существует, т.к. $2x^2-x-1$ становится отрицательным, а в действительной области чисел корень из отрицательного числа извлечь нельзя.

Таким образом, необходимо найти только значения на концах графика (точки $x=2, x=3$) $y(x)$.

$$y(2)=\frac{-3}{\sqrt{5}}$$
$$y(3)=\frac{-3}{\sqrt{14}}$$

Выберем наибольшее и наименьшее значение. Наибольшее $y(3)=\frac{-3}{\sqrt{14}}$, наименьшее $y(3)=\frac{-3}{\sqrt{14}}$.

\end{document}
