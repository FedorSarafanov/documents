\documentclass[a4paper,12pt]{extarticle}

\usepackage{cmap}
\usepackage[T2A]{fontenc}
\usepackage[utf8x]{inputenc}
\usepackage[english, russian]{babel}

\usepackage{misccorr} % в заголовках появляется точка, но при ссылке на них ее нет
\usepackage{amssymb,amsfonts,amsmath,amsthm}  
\usepackage{indentfirst}
\usepackage[usenames,dvipsnames]{color} 
\usepackage[unicode, colorlinks, urlcolor=magenta, linkcolor=black, pagecolor=black]{hyperref}
\usepackage{makecell,multirow} 
\usepackage{ulem}
\usepackage{graphicx}
\graphicspath{{img/}}
\usepackage{geometry}
\geometry{left=2cm,right=2cm,top=2cm,bottom=2cm,bindingoffset=0cm}
\usepackage{fancyhdr} 
\linespread{1.3} 
\frenchspacing 
\renewcommand{\labelenumii}{\theenumii)}
\usepackage{subcaption}
%%%%%%%%%%%%%%%%%%%%%%%%%%%%%%%%%%%%%%%%%%%%%%%%%%%%%%%%%%%%%%%%%%%%%%%%%%%%%%%
%%%%%%%%%%%%%%%%%%%%%%%%%%%%%%%%%%%%%%%%%%%%%%%%%%%%%%%%%%%%%%%%%%%%%%%%%%%%%%%

%%%%%%%%%%%%%%%%%%%%%%%%%%%%%%%%%%%%%%%%%%%%%%%%%%%%%%%%%%%%%%%%%%%%%%%%%%%%%%%
\usepackage[outline]{contour}
\usepackage{float}
\usepackage[mode=buildnew]{standalone}
\usepackage{tikz} 

\usepackage{tikz,csvsimple}
\usetikzlibrary{scopes}
\usetikzlibrary{decorations.pathreplacing,decorations.pathmorphing,patterns,calc,scopes,arrows,through, shapes.misc}
\newcommand{\pI}{\vec{p}_i}

% put color to \boxed math command
% \newcommand*{\boxcolor}{orange}
\tikzset{force/.style={>=latex,draw=blue,fill=blue}, axis/.style={densely dashed,gray,font=\small}, acceleration/.style={>=open triangle 60,draw=blue,fill=blue}, inforce/.style={force,double equal sign distance=2pt}, interface/.style={pattern = north east lines, draw    = none, pattern color=gray!60, }, cross/.style={cross out, draw=black, minimum size=2*(#1-\pgflinewidth), inner sep=0pt, outer sep=0pt},    cargo/.style={rectangle, fill=black!70, inner sep=2.5mm, }}

\makeatletter
\renewcommand{\boxed}[2]{
\textcolor{#1}{%
\tikz[baseline={([yshift=-1ex]current bounding box.center)}] 
\node [rectangle, minimum width=1ex,rounded corners,draw]
{\normalcolor\m@th$\displaystyle#2$};}
}\makeatother

\newcommand*\cancel[2][thin]{\tikz[baseline] \node [strike out,draw,anchor=text,inner sep=0pt,text=black,#1]{$#2$};}

\begin{document}

\tableofcontents

\section{Динамика системы материальных точек (СМТ)}

% \begin{figure}[htbp]
% 	\centering
% 	\begin{tikzpicture}
% 		\foreach \x in {0,2,...,6} {
% 			\foreach \y in {0,2,...,4} {
% 				\draw[fill=black] (\x,\y) circle(2pt);
% 			}
% 		}

% 		\draw[->] (0,0) node[below, yshift=-1em] {$m_1$} -- node[above] {$\vec{v}_1$} ++(45:1);
% 		\draw[->] (2,4) node[below, yshift=-1em] {$m_2$} -- node[above] {$\vec{v}_2$} ++(-45:1);
% 		\draw[->] (4,2) node[below, yshift=-1em] {$m_i$} -- node[above] {$\vec{v}_i$} ++(-135:1);
% 		\draw[->] (4,4) node[right, yshift=-1em] {$m_N$} -- node[above] {$\vec{v}_N$} ++(135:1);

% 	\end{tikzpicture}
% 	\caption{Система из $N$ материальных точек}
% 	\label{fig:label}
% \end{figure}



\subsection{Законы сохранения для системы материальных точек} 

\subsubsection{Теорема о изменении импульса СМТ} 

По определению,
\begin{equation}
\vec{p}_i=m_i\vec{v}_i
\end{equation}

Введем:
\begin{equation}
	\vec{p}_c=\sum_{i=1}^N \vec{p}_i\\
\end{equation}

Можем расписать как
\begin{equation}
	\label{dp_i:dt}
	\frac{d\pI}{dt}=\vec{F}_i=\vec{F}_i^\text{внутр}+\vec{F}_i^\text{внеш}\\
\end{equation}

Где
\begin{equation}
	\vec{F}_i^\text{внутр}=\vec{F}_{1,i}+\vec{F}_{2,i}+\ldots+\vec{F}_{i-1,i}+\vec{F}_{i+1,i}+\ldots+\vec{F}_{N,i}\\
\end{equation}

Будем рассматривать $\vec{F}_i^\text{внутр}$ для $\forall i$:
\begin{equation}
	\begin{aligned}
		i=1: \quad& \vec{F}_{2,1}+\ldots\\
		i=2: \quad& \vec{F}_{1,2}+\ldots\\
	\end{aligned}
\end{equation}

По третьему закону Ньютона:
\begin{equation}
	\vec{F}_{i,j}+\vec{F}_{j,i}=0
\end{equation}

Тогда
\begin{equation}
	\sum_{i=1}^N \vec{F}_i^\text{внутр} = 0
\end{equation}
	% \sum_{i=1}^N \frac{d\pI}{dt} = \\
Подействуем оператором суммы на левую и правую части уравнения (\ref{dp_i:dt}):

\begin{equation}
	\sum_{i=1}^N \frac{d\pI}{dt}= \underbrace{\sum_{i=1}^N \vec{F}_i^\text{внутр}}_{\equiv 0}+ \underbrace{\sum_{i=1}^N \vec{F}_i^\text{внеш}}_{\equiv \vec{F}_c^\text{внеш}}\\
\end{equation}

Перепишем левую часть, вытащив дифференцирование из-под суммы:
\begin{equation}
	\frac{d}{dt}\sum_{i=1}^{N}\vec{p}_i=\frac{d\vec{p}_c}{dt}
\end{equation}

И правую:
\begin{equation}
	\sum_{i=1}^{N} \vec{F}_i^\text{внеш}=\vec{F}_c^\text{внеш}
\end{equation}

Тогда получаем \textbf{теорему о изменении импульса СМТ в дифференциальной форме}:
\begin{equation}
	\frac{d\vec{p}_c}{dt}=\vec{F}_c^\text{внеш}
\end{equation}

Интегрируя, получим \textbf{теорему о изменении импульса СМТ в интегральной форме}:
\begin{equation}
	\vec{p}_c(t)-\vec{p}_c(t_0)=\int_{t_0}^t \vec{F}_c^\text{внеш}\cdot dt
\end{equation}

Рассматриваются следующие важные случаи.

\textbf{Случай первый.}  Внешняя сила $\vec{F}_i=1$ равна нулю при любых $i$. Тогда система называется \textit{изолированной}. Такое состояние достигнуть очень сложно: оно, скорее, является гипотетическим.

\begin{equation}
	\vec{F}^\text{внеш}_i=1 \quad \forall i
\end{equation}
Тогда
\begin{equation}
	\vec{F}^\text{внеш}_c=0 \Rightarrow \vec{p}_c=const\\
\end{equation}

\textbf{Случай второй.} Произвольно выбранная внешняя сила может быть не равна нулю, но сумма внешних сил $\vec{F}_c$ равна нулю. Это уже более реальный случай, чем предыдущий.
\begin{equation}
	\vec{F}^\text{внеш}_c=0 \Rightarrow \vec{p}_c=const\\
\end{equation}


\textbf{Случай третий.} Сумма внешних сил $\vec{F}_c^\text{внеш}$ не равна нулю, но сохраняется её направление. 
\begin{figure}[H]
	\centering
	\begin{tikzpicture}[scale=0.5]
	\draw[->] (0,0) -- (3,-1) node[right] {$\vec{F}_c^\text{внеш}$};
	\draw[dashed] (1.5,-0.5) -- ++ (70:1);
	\draw[dashed, ->] (1.5,-0.5) -- ++ (250:2) node[below] {$x$};
	\end{tikzpicture}
\end{figure}

Тогда можно выбрать такую ось $x$, что в проекции на неё
\begin{equation}
	F^\text{внеш}_{cx}=0 \Rightarrow p_{cx}=const\\
\end{equation}


\textbf{Случай четвертый.} Сумма внешних сил $\vec{F}_c^\text{внеш}$ не равна нулю, но если выполняется система условий:
\begin{equation}
	\left\{
	\begin{aligned}
		|\vec{F}_c^\text{внеш}|\ne\infty\\
		\Delta t \rightarrow 0
	\end{aligned}
	\right.
\end{equation}

Тогда
\begin{equation}
	\vec{p}_{c}=const\\
\end{equation}

\subsubsection{Теорема о движении центра масс системы материальных точек} 

Разберемся с геометрическим местом центром масс СМТ. 

Пусть мы имеем систему двух МТ $m_1=m_2$, тогда интуитивно $x_c=\frac{x_1+x_2}{2}$:

\begin{figure}[htbp]
	\centering
	\begin{tikzpicture}
		\draw[fill=black, dashed] (-2,0) -- (0,0) circle (2pt) node [above] {$m_1$} -- (2,0) node {$\times$} node[above, yshift=0.5em] {C} node[below, yshift=-0.5em] {$x_c$} -- (4,0) circle (2pt) node [above] {$m_2$} -- (6,0); 
	\end{tikzpicture}
	\caption{Система из двух МТ с равными массами}
	% \label{fig:label}
\end{figure}

Теперь пусть $m_1\ne m_2$, тогда интуитивно $x_c=\frac{m_1x_1+m_2x_2}{m_1+m_2}$:

\begin{figure}[htbp]
	\centering
	\begin{tikzpicture}
		\draw[fill=black, dashed] (-2,0) -- (0,0) circle (2pt) node [above] {$m_1$} -- (8/3,0) node {$\times$} node[above, yshift=0.5em] {C} node[below, yshift=-0.5em] {$x_c$} -- (4,0) circle (4pt) node [above] {$m_2$} -- (6,0); 
	\end{tikzpicture}
	\caption{Система из двух МТ с разными массами}
	% \label{fig:label}
\end{figure}

Теперь обобщаем на систему из $N$ материальных точек с произвольными массами:

\begin{equation}
	\left\{\begin{aligned}
		x_c=\frac{\sum_{i=1}^{N} m_ix_i}{\sum_{i=1}^{N} m_i}=\frac{\sum_{i=1}^{N} m_ix_i}{m_c}\\
		y_c=\frac{\sum_{i=1}^{N} m_iy_i}{\sum_{i=1}^{N} m_i}=\frac{\sum_{i=1}^{N} m_iy_i}{m_c}\\
		z_c=\frac{\sum_{i=1}^{N} m_iz_i}{\sum_{i=1}^{N} m_i}=\frac{\sum_{i=1}^{N} m_iz_i}{m_c}\\
	\end{aligned}\right.
\end{equation}

\textbf{Определение.} Центр масс - это такая точка, которая задается радиус-вектором
\begin{equation}
	\vec{R}_c=\frac{\sum_{i=1}^N m_i\vec{r}_i}{m_c}
\end{equation}

Нужно задастся вопросом: \textit{сменится ли положение центра масс от смены точки отсчета (полюса) O?}

\begin{figure}[htbp]
	\centering
	\begin{tikzpicture}
		\draw[fill=black] (4,0) node [right] {$m_i$};
		\draw[->, >=latex] (0,0) -- node [above] {$\vec{r}^{\,\,'}_i$} (4,0); 
		\draw[->, >=latex] (0.5,-2) -- node [above] {$\vec{r}_i$} (4,0); 
		\draw[->, >=latex] (0.5,-2) -- node [left] {$\vec{\rho}$} (0,0); 
	\end{tikzpicture}
	% \caption{Система из двух МТ с разными массами}
	% \label{fig:label}
\end{figure}

Геометрически очевидно, что
\begin{equation}
	\vec{r}_i=\vec{r}^{\,\,'}_i+\vec{\rho}
\end{equation}

Тогда
\begin{equation}
	\vec{R}_c=\frac{\sum_{i=1}^N m_i\vec{r}_i}{m_c}+\vec\rho\frac{\sum_{i=1}^N m_i}{m_c}
\end{equation}
	
И получаем что и требовалось найти: 
\begin{equation}
	\vec{R}_c=\vec{R}_c^{\,\,'}+\vec{\rho}
\end{equation}

Положение центра масс \textit{не зависит} от положения полюса.


\textbf{Определение.} Скорость центра масс задается как:
\begin{equation}
	\vec{V}_c=\frac{d \vec{R}_c}{dt}
\end{equation}

\textit{Оговорка:} $v\ll c \Rightarrow m_i=const$. Тогда
\begin{equation}
	\vec{V}_c=\frac{1}{m_c}\sum_{i=1}^N m_i \frac{d\vec{r}_i}{dt}=
	\frac{\sum_{i=1}^N m_i\vec{v}_i}{m_c}
\end{equation}

\begin{equation}
	\vec{a}_c=\frac{d\vec{V}_c}{dt}=\frac{d}{dt}
	\left(\frac{\sum_{i=1}^N m_i\vec{v}_i}{m_c}\right) \bigg|\cdot m_c
\end{equation}

\begin{equation}
	m_c\vec{a}_c=\frac{d}{dt}
	\left(\sum_{i=1}^N m_i\vec{v}_i\right)=\frac{d\vec{p}_c}{dt}=\vec{F}_c^\text{внеш}
\end{equation}

По сути, весь <<размазанный>> по пространству импульс системы мы можем причислить к одной точке -- \textit{центру масс} системы.

\begin{equation}
	m_c\vec{a}_c=\vec{F}_c^\text{внеш}
\end{equation}

Это и есть теорема о движении центра масс системы материальных точек.

Мы получили важный результат: внутренние силы \textit{не могут} создать ускорения СМТ!



\subsubsection{Динамика тел переменной массы. Уравнение Мещерского} 

\begin{figure}[H]
\begin{center}
\begin{minipage}[h]{0.4\linewidth}
	\begin{tikzpicture}
		\begin{scope}
			\clip  (0,0) circle (1.999cm);
			\draw[pattern=north east lines, pattern color=blue] (0,0) ++ (-135:2) circle (1cm);			
		\end{scope}

		\coordinate (C) at (1,1);

			\draw[pattern=north west lines, pattern color=blue] (C) ++ (45:2) circle (1cm);			
		\begin{scope}
			\clip  (C) circle (2cm);
			\draw[color=white, fill=white] (C) ++ (45:2) circle (1.1cm);			
		\end{scope}
			\draw ([shift=(16:2cm)]C) arc (16:74:2cm);


		 \contourlength{1mm};

		\node  at ($(C)+(45:2.5)$) {\contour{white}{$\Delta m_2$}};
		
		\node  at ($(0,0)+(-135:1.5)$) {\contour{white}{$\Delta m_1$}};

		\node  at ($(0,0)$) {\contour{white}{$M$}};

		\draw (0,0) circle (2cm);

		\draw[->] ([shift=(45:2cm)]C) -- node[below, yshift=-0.5em] {$\vec{v}_2$} ++(-135:1);

		\draw[->] ([shift=(-45:2cm)]0,0) -- node[above, pos=0.7] {$\vec{v}$} ++(-45:1);

		\draw[color=white] (-3.7,-3.7) rectangle  (3.5,3.5);
	\end{tikzpicture}
	\subcaption{Момент времени $t_0$}	
\end{minipage}
\hfill 
\begin{minipage}[h]{0.4\linewidth}
	\begin{tikzpicture}

		\draw (0,0) circle (2cm);
		\coordinate (C1) at (-0.02,-0.02);

		\begin{scope}
			\clip  (C1) circle (1.999cm);
			\draw[fill=white] (C1) ++ (-135:2) circle (1cm);			
		\end{scope}
			% \draw ([shift=(16+180:2cm)]C1) arc (16+180:74+180:2cm);

		\coordinate (C1) at (-1,-1);

		\begin{scope}
			\clip  (C1) circle (1.999cm);
			\draw[pattern=north east lines, pattern color=blue] (C1) ++ (-135:2) circle (1cm);			
		\end{scope}
			\draw ([shift=(16+180:2cm)]C1) arc (16+180:74+180:2cm);


		\coordinate (C) at (0,0);

			\draw[pattern=north west lines, pattern color=blue] (C) ++ (45:2) circle (1cm);			
		\begin{scope}
			\clip  (C) circle (2cm);
			\draw[color=white, fill=white] (C) ++ (45:2) circle (1.1cm);			
		\end{scope}


		 \contourlength{1mm};

		\node  at ($(C)+(45:2.5)$) {\contour{white}{$\Delta m_2$}};
		
		\node  at ($(C1)+(-135:1.5)$) {\contour{white}{$\Delta m_1$}};

		\node  at ($(0,0)$) {\contour{white}{$M$}};


		\draw[->] ([shift=(-135:2cm)]C1) --  ++(-135:1)
					node[below] {$\vec{v}_1$};

		\draw[->] ([shift=(-30:2cm)]0,0) -- node[below, pos=1, yshift=0em] {$\vec{v}+\Delta\vec{v}$} ++(-30:1);
		\draw[color=white] (-3.7,-3.7) rectangle  (3.5,3.5);

	\end{tikzpicture}
	\subcaption{Момент времени $t_0+\Delta t$}	
\end{minipage}
\end{center}
\end{figure}

Пусть $M$ -- масса <<основного>> тела, $\Delta m_1$ -- то, что 
<<отвалится>> ($\Delta m_1>0$), $\Delta m_2$ -- то, что присоединится. 

Запишем импульс системы до и после изменения конфигурации:
\begin{equation}
	\vec{p}_0=\vec{p}(t_0)=M\vec{v}+\Delta m_2\vec{v}_2=M\vec{v}+\Delta m_2(\vec{v}+\vec{u}_2)
\end{equation}
\begin{equation}
	\vec{p}(t)=(M-\Delta m_1 +\Delta m_2)[\vec{v}+\Delta\vec{v}]+\Delta m_1 ([\vec{v}+\Delta\vec{v}]+\vec{u}_1)
\end{equation}
Тогда изменение импульса будет
\begin{equation}
	\Delta\vec{p}=
	\vec{p}(t)-\vec{p}_0=(M-\Delta m_1 +\Delta m_2)[\vec{v}+\Delta\vec{v}]+\Delta m_1 ([\vec{v}+\Delta\vec{v}]+\vec{u}_1)-
	M\vec{v}-\Delta m_2(\vec{v}+\vec{u}_2)
\end{equation}
Теперь нужно аккуратно раскрыть скобки:
\begin{gather}
	\begin{aligned}
		\Delta\vec{p}=
		\vec{p}(t)-\vec{p}_0=
		M\vec{v}-\Delta m_1\vec{v} +\Delta m_2\vec{v}+\\+
		M\Delta\vec{v}-\boxed{red}{\Delta m_1\Delta\vec{v}} +\boxed{red}{\Delta m_2\Delta\vec{v}}+\\+
		\Delta m_1 \vec{v}+\boxed{red}{\Delta m_1\Delta\vec{v}}+\Delta m_1\vec{u}_1-\\-
		M\vec{v}-
		\Delta m_2\vec{v}-\Delta m_2\vec{u}_2
	\end{aligned}
\end{gather}
Величинами вида $\boxed{red}{\Delta \cdot\Delta}$ пренебрежем, как величинами более высокого порядка малости, чем $\Delta$. Приведем подобные:
\begin{gather}
	% \begin{aligned}
		\Delta\vec{p}=
		\vec{p}(t)-\vec{p}_0=
			\cancel[thick,draw=green]{M\vec{v}}-\cancel[thick,draw=red]{\Delta m_1\vec{v}} +\cancel[thick,draw=blue]{\Delta m_2\vec{v}}+\\+
		M\Delta\vec{v}+
		\cancel[thick,draw=red]{\Delta m_1\vec{v}}+\Delta m_1\vec{u}_1-\\-
		\cancel[thick,draw=green]{M\vec{v}}-
		\cancel[thick,draw=blue]{\Delta m_2\vec{v}}-\Delta m_2\vec{u}_2
	% \end{aligned}
\end{gather}
Тогда можем, наконец, окончательно записать изменение импульса:
\begin{equation}
	\Delta\vec{p}\backsimeq M\Delta\vec{v}+\Delta m_1\vec{u}_1-\Delta m_2\vec{u}_2
\end{equation}
Физика оперирует не бесконечно малыми величинами, а дискретными. Отсюда следует
\begin{equation}
	\Delta t \to 0 \quad \Rightarrow \quad \frac{\Delta{\vec{p}}}{\Delta{t}}\to \frac{d\vec{p}}{dt}
\end{equation}
Переходя к дифференциалам, равенство запишем строгим:
\begin{equation}
	\frac{d\vec{p}}{dt}= M\frac{d\vec{v}}{dt}+\frac{dm_1}{dt}\vec{u}_1-\frac{dm_2}{dt}\vec{u}_2
\end{equation}
Под импульсом мы понимали импульс системы, поэтому его производная - равнодействующая внешних сил. Перепишем формулу:
\begin{equation}
	M\frac{d\vec{v}}{dt}=\vec{F}^\text{внеш}-\frac{dm_1}{dt}\vec{u}_1+\frac{dm_2}{dt}\vec{u}_2
\end{equation}
Это то, к чему мы стремились - уравнение Мещерского.
\begin{equation}
	\vec{F}^\text{реакт}=-\frac{dm_1}{dt}\vec{u}_1+\frac{dm_2}{dt}\vec{u}_2
\end{equation}

\subsubsection{Теорема о изменении момента импульса СМТ}

\textbf{Определение.} Момент импульса $i$-й материальной точки СМТ
\begin{equation}
	\vec{N}_i=[\vec{r}_i\times m_i\vec{v}_i]
\end{equation}

\textbf{Определение.} Момент импульса СМТ
\begin{equation}
	\vec{N}_c=\sum_{i=1}^{N}[\vec{r}_i\times m_i\vec{v}_i]
\end{equation}    

Ранее мы доказали, что $\frac{d\vec{N}}{dt}=\vec{M}$. Зададимся вопросом, так ли это для СМТ?

Для $i$-й материальной точки СМТ
\begin{equation}
	\frac{d\vec{N}_i}{dt}=\vec{M}_i=[\vec{r}_i\times \vec{F}_i] \quad \forall i
\end{equation}
Тогда
\begin{equation}
	\sum_{i=1}^{N}\frac{d\vec{N}_i}{dt}=\sum_{i=1}^{N}[\vec{r}_i\times \vec{F}_i]=
	\sum_{i=1}^{N}[\vec{r}_i\times (\vec{F}^\text{внеш}_i+\vec{F}^\text{внутр}_i)]
\end{equation}
Распишем:
\begin{equation}
	\vec{F}^\text{внутр}_i=\vec{F}_{1,i}+\vec{F}_{2,i}+\ldots+\vec{F}_{j-1,i}+\vec{F}_{j+1,i}+\ldots+\vec{F}_{N,i}
\end{equation}
\begin{equation}
	\vec{F}_{i,j}=-\vec{F}_{j,i}
\end{equation}
Для взаимодействия $i$-й и $j$-й точек:
\begin{equation}
	[\vec{r}_i\times\vec{F}_{j,i}]+[\vec{r}_j\times\vec{F}_{i,j}]=[(\vec{r}_i-\vec{r}_j)\times\vec{F}_{j,i}]=
	[\vec{r}_{i,j}\times\vec{F}_{j,i}]\equiv0
\end{equation}
Это значит, что все $\vec{M}_i^\text{внутр}=0$.
\begin{equation}
	\sum_{i=1}^{N}\frac{d\vec{N}_i}{dt}=\sum_{i=1}^{N}[\vec{r}_i\times \vec{F}^\text{внеш}_i]
\end{equation}
Получили \textbf{теорему}: \textit{Момент импульса СМТ меняется  за счет \textbf{только} момента внешних сил.}
\begin{equation}
	\frac{d\vec{N}_c}{dt}=\vec{M}^\text{внеш}_c
\end{equation}

\subsubsection{Закон сохранения момента импульса}


\textbf{Случай первый.}  Внешняя сила $\vec{F}_i=1$ равна нулю при любых $i$. 

\begin{equation}
	\vec{F}^\text{внеш}_i=1 \quad \forall i \Rightarrow \vec{M}_c^\text{внеш}=0
\end{equation}
Тогда
\begin{equation}
	\vec{N}_c=const\\
\end{equation}

\textbf{Случай второй.} Произвольно выбранная внешняя сила может быть не равна нулю, но момент внешних сил $\vec{M}^\text{внеш}_c$ равен нулю. 
\begin{equation}
	\vec{M}^\text{внеш}_c=0 \Rightarrow \vec{N}_c=const\\
\end{equation}

\textbf{Случай третий.} Произвольно выбранный момент внешней силы может быть не равна нулю, но момент внешних сил $\vec{M}^\text{внеш}_c$ равен нулю. 
\begin{equation}
	\vec{M}^\text{внеш}_c=0 \Rightarrow \vec{N}_c=const\\
\end{equation}


\textbf{Случай четвертый.} Момент внешних сил $\vec{M}_c^\text{внеш}$ не равен нулю, но сохраняет направление. 
\begin{figure}[H]
	\centering
	\begin{tikzpicture}[scale=0.5]
	\draw[->] (0,0) -- (3,-1) node[right] {$\vec{M}_c^\text{внеш}$};
	\draw[dashed] (1.5,-0.5) -- ++ (70:1);
	\draw[dashed, ->] (1.5,-0.5) -- ++ (250:2) node[below] {$x$};
	\end{tikzpicture}
\end{figure}

Тогда можно выбрать такую ось $x$, что в проекции на неё
\begin{equation}
	M^\text{внеш}_{cx}=0 \Rightarrow N_{cx}=const\\
\end{equation}


\textbf{Случай пятый.} Запишем теорему о изменении момента импульса в интегральной форме.
\begin{equation}
	\Delta \vec{N}_c=\vec{N}_c(t)-\vec{N}_c(t_0)=\int_{t_0}^t \vec{M}_c^\text{внеш}
\end{equation}
Тогда если выполняется система условий:
\begin{equation}
	\left\{
	\begin{aligned}
		|\vec{M}_c^\text{внеш}|\ne\infty\\
		\Delta t=t-t_0 \rightarrow 0
	\end{aligned}
	\right.
\end{equation}
То
\begin{equation}
	\vec{N}_{c}=const\\
\end{equation}

Нужно задастся вопросом: \textit{что зависит от смены точки отсчета (полюса) O?}

\begin{figure}[htbp]
	\centering
	\begin{tikzpicture}
		\draw[fill=black] (4,0) node [right] {$m_i$};
		\draw[->, >=latex] (0,0) circle (1pt) node[left] {$O'$}  -- node [above] {$\vec{r}^{\,\,'}_i$} (4,0); 
		\draw[->, >=latex] (0.5,-2) circle (1pt) node[left] {$O$} -- node [above] {$\vec{r}_i$} (4,0); 
		\draw[->, >=latex] (0.5,-2) -- node [left] {$\vec{\rho}$} (0,0); 
	\end{tikzpicture}
	% \caption{Система из двух МТ с разными массами}
	% \label{fig:label}
\end{figure}
\begin{gather}
	\vec\rho=\vec{OO'}\\
	\vec{r}_i=\vec\rho+\vec{r'}_i\\
	\vec{N}_{O'i}=[\vec{r'}_i\times\vec{p}_i]\\
	\vec{N}_{Oi}=[\vec{r}_i\times\vec{p}_i]=[\vec{r'}_i\times\vec{p}_i]+[\vec\rho\times\vec{p}_i]\\
	\vec{N}_{Oi}=\vec{N}_{O'i}+[\vec\rho\times\vec{p}_i]\\
	\vec{N}_{O}=\vec{N}_{O'}+\sum_{i=1}^{N}[\vec\rho\times\vec{p}_i]\\
	\vec{N}_{O}=\vec{N}_{O'}+[\vec\rho\times\vec{p}_c]
\end{gather}
Если мы найдем такую СО, где импульс системы равен нулю, то в ней момент импульса не зависит от выбора точки отсчета.

Такая СО - \textbf{центромассовая}, и если мы в ней -- сопровождающая.
Будем обозначать величины в ЦМСО звездочкой:
\begin{equation}
	\vec{p}_c^{\,*}=0
\end{equation}

\subsubsection{Связь моментов импульса в лабораторной и центромассовой системах отсчета}
\newcommand\zi{^{\,*}_i}
\newcommand\sumn{\sum_{i=1}^{N}}
$O$ -- начало отсчета в ЛСО (лабораторной системе отсчета), $O'$ -- в ЦМСО (центромассовой системе отсчета).
\begin{equation}
	\vec{r}_i=\vec{r}\zi+\vec\rho
\end{equation}
\begin{figure}[htbp]
	\centering
	\begin{tikzpicture}
		\coordinate (A) at (0,0);
		\coordinate (B) at (0.5,-2);
		\coordinate (C) at (4,0);
		\draw[fill=black] (C) circle (1pt) node [right] {$m_i$};
		\draw[fill=black] (0,0) circle (1pt) node[left] {$O'$}; 
		\draw[fill=black] (0.5,-2) circle (1pt) node[left] {$O$}; 

		\draw[->, >=latex] (A) -- node [above] {$\vec{r}\zi$} (C);
		\draw[->, >=latex] (B) -- node [above] {$\vec{r}_i$}  (C);
		\draw[->, >=latex] (B) -- node [left] {$\vec{\rho}$}  (A);
	\end{tikzpicture}
	% \caption{Система из двух МТ с разными массами}
	% \label{fig:label}
\end{figure}
\begin{gather}
	\vec{v}_i=\vec{v}\zi+\vec{v}_c\\
	\vec{N}_i=[\vec{r}_i\times m_i\vec{v}_i]\\
	\vec{N}_c=\sumn[\vec{r}_i\times m_i\vec{v}_i]=\sumn[(\vec{r}\zi+\vec\rho)\times m_i(\vec{v}\zi+\vec{v}_c)]=\\=
	\underbrace{\sumn[\vec{r}\zi\times m_i\vec{v}\zi]}_{\displaystyle \equiv {\vec{N}_c^{\,*}}}+
	\underbrace{\sumn[\vec{r}\zi\times m_i\vec{v}_c]}_{\displaystyle \equiv [\vec{r}_c^{\,*}\times\vec{p}_c]}+
	\underbrace{\sumn[\vec\rho\times m_i\vec{v}\zi]}_{\displaystyle \equiv [\vec\rho\times\vec{p}_c^{\,*}]=0}+
	\underbrace{\sumn[\vec\rho\times m_i\vec{v}_c]}_{\displaystyle \equiv [\vec\rho\times\vec{p}_c]}=\\=
	{\vec{N}_c^{\,*}}+[\vec{r}_c^{\,*}\times\vec{p}_c]+[\vec\rho\times\vec{p}_c]=\\=
	{\vec{N}_c^{\,*}}+[(\vec{r}_c^{\,*}+\vec\rho)\times\vec{p}_c]=\\=
	\vec{N}_c^{\,*}+[\vec{r}_c\times\vec{p}_c]
\end{gather}
Причем здесь первое слагаемое отвечает за вращение СМТ относительно центра масс, а второе - за вращение центра масс относительно ЛСО.
% \newpage
\subsubsection{Уравнение моментов относительно оси}

Введем сферическую систему координат с центром $O$, ортонормированным базисом $\{\vec{e}_r,\vec{e}_\tau,\vec{e}_\phi\}$

\begin{figure}[H]
\begin{center}
\begin{minipage}[h]{0.49\linewidth}
	\centering
	\begin{tikzpicture}
		\draw[black!50] (0,0) circle (2cm);
		\draw[black,fill=black] (0,0) circle (1pt);

		\draw[magenta!70, dashed] (0,0) (180:2.5) -- (0:2.5);
		\draw[magenta!70, dashed] (0,0) (90:2.5) -- (-90:2.5);
		\draw[blue, ->, >=latex] (0,0) (90:2.5) -- node[left] {$\vec\omega$} (90:3);
		 \contourlength{1.5mm};
		\draw[magenta!70, dashed] (-90:1) node[rotate=90] {\contour{white}{Ось $z$}};
		\draw[->, >=latex] (0,0) -- node[midway,fill=white!20, opacity=0.9] {$\vec{r}$} (45:2) coordinate (A);
		\draw[->, >=latex] (0,1.414) -- node[midway,fill=white!20, opacity=0.9] {$\vec{r}_\perp$} (A);
		\draw[->, >=latex] (0,0) -- node[midway,fill=white!20, opacity=0.9] {$\vec{r}_\parallel$} (0,1.414);

		\draw[magenta,->, >=latex] (A) -- ++(45:0.7) node[right, above] {$\vec{e}_r$};
		\draw[magenta,->, >=latex] (A) -- ++(135:0.7) node[right, above] {$\vec{e}_\tau$};
	\end{tikzpicture}
	\subcaption{Вид в разрезе}	
\end{minipage}
\hfill 
\begin{minipage}[h]{0.49\linewidth}
	\centering
	\begin{tikzpicture}
		\draw[black!50] (0,0) circle (2cm);
		\draw[black,fill=black] (0,0) circle (1pt);

		\draw[magenta!70, dashed] (0,0) (180:2.5) -- (0:2.5);
		\draw[magenta!70, dashed] (0,0) (90:2.5) -- (-90:2.5);
		\draw[blue, ->, >=latex, opacity=0] (0,0) (90:2.5) -- node[left] {$\vec\omega$} (90:3);


		\draw[blue] (110:1.2) node[left] {$\bigodot\vec\omega$};

		\draw[->, >=latex] (0,0) -- (0:2) coordinate (A);

		\draw[magenta,->, >=latex] (A) -- ++(0:0.7) node[right, above] {$\vec{e}_r$};
		\draw[magenta,->, >=latex] (A) -- ++(90:0.7) node[right, above] {$\vec{e}_\phi$};
	\end{tikzpicture}
	\subcaption{Вид сверху}	
\end{minipage}
\end{center}
\end{figure}
\begin{equation}
	\vec{v}=\vec{v}_\tau+\vec{v}_r+\vec{v}_\phi
\end{equation}
Можем записать
\begin{equation}
	\vec{v}_\phi=[\omega\times\vec{r}\,]=[\omega\times\vec{r}_\perp]
\end{equation}    
Запишем момент импульса по определению:
\begin{equation}
	\vec{N}=[\vec{r}\times m\vec{v}\,]=m([\vec{r}\times\vec{v}_\tau]+[\vec{r}\times\vec{v}_r]+[\vec{r}\times\vec{v}_\phi])
\end{equation}
Формально запишем его проекцию на $z$:
\begin{equation}
	[\vec{N}]_z=[\vec{r}\times m\vec{v}_\phi]=m[\vec{r}\times[\vec\omega\times\vec{r}_\perp]]
\end{equation}
Откуда по бессмертному <<бац минус цаб>>
\begin{equation}
	N_z=m\omega_z(\vec{r},\vec{r}_\perp)-m\underbrace{(\vec{r}_\perp)_z}_{\equiv 0}(\vec{r},\omega)=m\omega_zr_\perp^2
\end{equation}
Все вышеизложенные выкладки были для \textit{одной} материальной точки. Для СМТ:
\begin{equation}
	N_{cz}=\sumn m_i\omega_{zi}r_{\perp i}^2
\end{equation}
В частном случае, когда все точки тела вращаются с одной угловой скоростью (твердое тело),
\begin{equation}
	\omega_{zi}=\omega_z \quad\forall i \Rightarrow N_{cz}=\omega_z\sumn m_i r_{\perp i}^2
\end{equation}
\textbf{Определение.} Момент инерции - это мера инертности вращательного движения, выражающаяся как
\begin{equation}
	I=\sumn m_i r_{\perp i}^2
\end{equation}
\begin{equation}
	N_{cz}=I\omega_z
\end{equation}
Можем записать закон сохранения момента импульса в таком виде:
\begin{equation}
	I_1\cdot\omega_{1z}=I_2\cdot\omega_{2z}
\end{equation}

\subsection{Энергетические соотношения для СМТ} 

Для точки нам известно:
\begin{equation}
	W_k=\frac{mv^2}{2}
\end{equation}
\begin{equation}
	\Delta W_k=W_k^\text{кон}-W_k^\text{нач}=A_{1-2}^\text{всех}=\int_{(1)}^{(2)}(\vec{F},d\vec{l})
	=\int_{(1)}^{(2)}F_l\cdot{dl}
\end{equation}
% \textbf{Определение.} 
\begin{gather}
	W_{kc}=\sumn\frac{m_iv_i^2}{2}\\
	\Delta W_{ki}=A_i^\text{всех},\quad\forall i\\
	\Delta W_{kc}=\sumn\Delta W_{ki}=\sumn A_i^\text{всех}
\end{gather}

Поставим себе задачу: \textit{попробовать избавиться (по аналогии с импульсом и моментом импульса) от внутренних сил в <<$A^\text{всех}$>>}

\begin{figure}[H]
\begin{center}
\begin{minipage}[h]{0.49\linewidth}
	\centering
	\begin{tikzpicture}
		\coordinate (A) at (0,0) ;
		\coordinate (B) at (3,-3) ;
		\draw[dashed] (A) node [above] {$m_i$} -- (B) node [below] {$m_j$};
		\draw[fill=black] (A) circle (2pt) (B) circle (2pt);

		\draw[->, >=latex,blue] (A) -- node[midway,fill=white!20, opacity=0.9] {$\vec{F}_{ji}$} ++(-45:1.5);
		\draw[->, >=latex,blue] (B) -- node[midway,fill=white!20, opacity=0.9] {$\vec{F}_{ij}$} ++(135:1.5);
	\end{tikzpicture}\end{minipage}
\hfill 
\begin{minipage}[h]{0.49\linewidth}
\begin{gather}
	\vec{F}_{ij}=-\vec{F}_{ji}\\
	dA_i=\vec{F}_{ji}\cdot d\vec{l}_i\\
	dA_j=\vec{F}_{ij}\cdot d\vec{l}_j\\
	dA_i+dA_j=\vec{F}_{ji}(d\vec{l}_i-d\vec{l}_j)
\end{gather}
\end{minipage}
\end{center}
\end{figure}
Печально, но это -- \textbf{не ноль в общем случае}. Избавиться от внутренних сил в <<$A^\text{всех}$>> не удалось.

\subsubsection{Связь между $W_{kc}$ в различных системах отсчета. Теорема Кёнига}
    
Как всегда, нас будет интересовать выделенная система отсчета. Обозначим $K$ -- лабораторную систему отсчета, $K'$ -- движущуюся относительно ЛСО со скоростью $\vec{u}$.

\begin{figure}[H]
    \centering
\begin{tikzpicture}
    \draw[axis,->] (-1,0) -- (3,0) node[right, black] {$x$};

    \draw[axis,->] (0,1) -- ++(0,1) node[right, black] {$K'$};С
    \draw[axis,->] (-1,0) -- ++(0,3) node[right, black] {$K$};

    \draw[force,->] (0,1.3) -- node[above, black] {$\vec{u}$} ++(1,0);
    \draw[axis,->] (0,1) --  ++(2,0);
    % \draw[force,->] (4,1.2) -- node[above, black] {$-\vec{v}$} ++(-1,0);

    % \draw[line width=2pt] (0,1) -- ++(1,0);
    % \draw[line width=2pt] (5,1.2) -- ++(-1,0);

\end{tikzpicture}
\end{figure}

\begin{equation}
	\vec{v}=\vec{v}{\,'}+\vec{u}
\end{equation}
\begin{equation}
	W_k=\frac{mv^2}{2}=\frac{mv'^{2}}{2}+
	\frac{mu^{2}}{2}+
	m(\vec{v}{\,'},\vec{u})
\end{equation}
Для системы материальных точек
\begin{equation}
	W_{kc}=\sumn W_{ki}=\underbrace{\sumn\frac{m_iv'^{2}_i}{2}}_{\displaystyle \equiv W'_{kc}}+
	\underbrace{\sumn\frac{m_iu^{2}}{2}}_{\displaystyle \equiv \frac{m_c u^2}{2}}+
	\underbrace{\sumn m_i(\vec{v_i}{\,'},\vec{u})}_{\displaystyle \equiv (\vec{p}{\,'}_c,\vec{u})}
\end{equation}

\textbf{Если} движущаяся система  -- ЦМСО, тогда $\vec{p'}_c=\vec{p}^{\,*}_c=0$, $u=v_c$ и выполняется \textbf{теорема Кёнига:}
\begin{equation}
	W_{kc}=W_{kc}^{*}+\frac{m_cv_c^2}{2}
\end{equation}

\subsubsection{Потенциальная энергия СМТ. Закон сохранения механической энергии для СМТ}
Для материальной точки нам известно:
\begin{equation}
	\underbrace{W_\text{п}^\text{нач}-W_\text{п}^\text{кон}}_{\displaystyle -\Delta W_\text{п}}=A_{1-2}^\text{конс}
\end{equation}
\begin{equation}
	W_\text{мех}=W_\text{п}+W_\text{к}
\end{equation}
По нашему определению, 
\begin{gather}
	W_{\text{п}c}=\sumn W_{\text{п}i}\\
	W_{\text{мех}_c}=\sumn W_{\text{мех}_i}
\end{gather}
\begin{equation}
	-\Delta W_{\text{п}c}=\sumn A_{1-2}^\text{конс}=A_c=A_c^\text{внеш,конс}+A_c^\text{внутр,конс}
\end{equation}

\textbf{Пусть} все силы будут консервативными. \textbf{Тогда} 
\begin{equation}
	A_c=A_c^\text{внеш,конс}+A_c^\text{внутр,конс}=-\Delta W_{\text{п}c}
\end{equation}
\begin{equation}
	A_c=\Delta W_{kc}
\end{equation}
{Иначе} говоря,
\begin{gather}
	-\Delta W_{\text{п}c}=\Delta W_{kc}\\
	\Delta(W_{\text{п}c}+\Delta W_{kc})=0
\end{gather}
Получили \textbf{закон сохранения механической энергии для СМТ}:
\begin{equation}
	W_{\text{мех}_c}=\textrm{const}
\end{equation}
\end{document}
