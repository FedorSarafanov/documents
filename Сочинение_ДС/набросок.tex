\documentclass[a4paper,14pt]{extreport}
 
\usepackage{extsizes}
\usepackage{cmap}
\usepackage[T2A]{fontenc}
\usepackage[utf8x]{inputenc}
\usepackage[russian]{babel}
\usepackage{cyrtimes}
\usepackage[unicode]{hyperref}


\begin{document}

Понятие патриотизм точно определил Н.М.Карамзин: ''Патриотизм есть любовь ко благу и славе Отечества и желание способствовать им во всех отношениях''. Похожее определение дает В.Соловьев: ''Ясное сознание своих обязанностей по отношению к отечеству и верное их исполнение образуют добродетель патриотизма''   Исходя из этих определений, суть любви к Отечеству состоит в понимании главных задач, стоящих перед обществом и государством, в неустанной борьбе за их решение. Патриотизм в русском национальном самосознании всегда был связан с жертвенностью. 

Призыв ''положить жизнь за Отечество'' звучал в стихах Н.М.Карамзина, С.Н.Глинки, А.И.Тургенева. В то же время патриотизм чаще всего сопряжен в общественном сознании с военной деятельностью, но не захватнической.

История нашего государства - это история войн в его защиту. Поэтому стержнем государственного патриотизма становится военно-патриотическое воспитание, получив-шее заметное развитие в трудах и деяниях П.А.Румянцева, А.В.Суворова, М.И.Кутузова, П.С.Нахимова, М.И.Драгомирова, С.О.Макарова, М.Д.Скобелева и других.

За последние полтора десятилетия российской истории, пожалуй, ни одна идеологическая ценность не подвергалась таким переосмыслениям, а следовательно, испытаниям, как патриотизм.

Характерной чертой перестроечного времени стало крушение различных догм и постулатов. Так была брошена в оборот и подхвачена интиллегенцией фраза: «Патриотизм – это чувство примитивное, оно есть даже у кошки».  Предполагалось, что патриотизм - отжившая ценность, мешающая строить новое демократическое общество, свободное от прежних стереотипов.

Прошло некоторое время и в канун 50-летия Победы над фашизмом тем же интеллектуалам пришлось реанимировать патриотизм, как неотъемлемую составную часть российского менталитета. Оказалось, что без патриотизма невозможно построить новое сильное государство, привить людям понимание их гражданского долга и уважения к закону. Без ясного, определенного акцента на защиту интересов России невозможно выработать сколько-нибудь плодотворную и самостоятельную внешнюю и внутреннюю политику. Без заботы о собственной национальной экономике, национальном рынке, росте отечественных производителей, опоры на собственные силы невозможно улучшить жизнь людей. Без уважения к собственной истории, к делам и традициям старших поколений невозможно вырастить морально здоровую молодежь. Без возрождения национальной гордости, национального достоинства невозможно вдохновить людей на высокие дела.

Вместе с истинным проявлением высокого чувства встречается и спекуляция патриотизмом. Князь Вяземский в своих «Письмах из Парижа» пишет:

	Многие признают за патриотизм безусловную похвалу всему, что своё. Тюрго называл это лакейским патриотизмом, du patriotisme d'antichambre. У нас можно бы его назвать квасным патриотизмом. Я полагаю, что любовь к отечеству должна быть слепа в пожертвованиях ему, но не в тщеславном самодовольстве; в эту любовь может входить и ненависть. Какой патриот, какому народу ни принадлежал бы он, не хотел бы выдрать несколько страниц из истории отечественной, и не кипел негодованием, видя предрассудки и пороки, свойственные его согражданам? Истинная любовь ревнива и взыскательна.

Действительно, нередко можно встретить подобных вяземским ``ура-патриотов''. Они восхваляют и 

Перед Россией встала важнейшая задача - реализовать огромный духовно-нравственный потенциал, накопленный за всю историю существования государства, для решения проблем в различных сферах жизни общества. Государственная стратегия России должна постоянно опираться на историческое и духовное наследие народа, поэтому в последнее десятилетие остро встал вопрос выработки национальной идеи, которая смогла бы объединить российский народ в новых исторических условиях. По мнению А.Кивы ''Национальная идея - это обруч нации. Как только он лопается, нация либо впадает в глубокую депрессию, либо распадается, либо становится жертвой какой-то реакционной идеи и даже человеконенавистнической идеологии''.


\end{document}