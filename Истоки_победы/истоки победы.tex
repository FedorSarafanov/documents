\documentclass[a4paper,14pt]{extarticle}

\usepackage{cmap}					% поиск в PDF
\usepackage[T2A]{fontenc}			% кодировка
\usepackage[utf8]{inputenc}			% кодировка исходного текста
\usepackage[english,russian]{babel}	% локализация и переносы
\usepackage[left=2cm,right=2cm]{geometry}
\author{Сарафанов Ф.Г.}
\title{Источники и цена Великой Победы}
\date{\today}

\begin{document}

\maketitle

\textbf{Довоенный СССР. Индустриализация довоенных пятилеток. }

\textbf{Пакт Молотова-Риббентроппа}

\textbf{Нападение. Священная война}

В самом начале войны 2 миллиона добровольцев было зачислено тогда в Армию и ещё два – в народное ополчение.

Всего за годы войны в партию вступило более 8 миллионов человек, а составляла она к концу войны 5 миллионов 340 тысяч. Причём около 60\% состава ВКП (б) находилось в армии.

Г.К.ЖУКОВ: 

    Мне не раз приходилось разговаривать с направлявшимися в войска политбойцами. Эти люди несли в себе какую-то особую, непоколебимую уверенность в нашей победе. «Выстоим!» - говорили они. И я чувствовал, что это не просто слова, это образ мышления, это настоящий советский патриотизм. Своим великолепным оптимизмом политбойцы возвращали уверенность людям, начинавшим терять присутствие духа

\textbf{Вооружение}    Самолеты - ПЕтляков, ЯКовлев, ТУполев (+СУхой), ЛАвочкин, ИЛьюшин, МИкоян, Поликарпов 

\textbf{Пушка Грабина. Противник - Тухачев с реактивной артиллерилией.}    Грабин 1933 год, завод № 92 (Горький) - ЗИС-3
    Сталин: \textit{<<Ваша пушка спасла Россию>>}

\textbf{Моральный дух. Подвиг Гастелло. }

Важнейшее условие Победы – единство партии и народа. Это единство
убедительно проявлялось не только на фронте, но и в тылу. Численность членов партии в промышленности, на транспорте, в сельском хозяйстве возросла за время войны в полтора раза.

Огромный размах приобрело партизанское движение, которое охватило вместе с подпольщиками около двух миллионов человек. А вот немцам, когда война шла уже на их территории, ничего подобного организовать не удалось.

\textbf{Наше дело правое. Враг будет разбит. Победа будет за нами}

\textbf{Трансформация идеологии}

«Содержательная» эволюция идеологического оформления войны.

Классовые лозунги вытеснялись из пропагандистского лексикона государства, заменяясь патриотическими. Не случайным после тяжелых поражений начала войны было обращение Сталина к национальным чувствам русского народа, ранее попиравшимся идеологическими догматами: духовные силы были призваны спасти положение там, где оказались недостаточными силы материальные. 

Так, весьма необычным оказалось соединение в одной речи Верховного Главнокомандующего на параде Красной Армии 7 ноября 1941 г. старых русских и революционных советских традиций и символов: 

\textit{«Война, которую вы ведете, есть война освободительная, война справедливая. Пусть вдохновляет вас в этой войне мужественный образ наших великих предков — Александра Невского, Димитрия Донского, Кузьмы Минина, Димитрия Пожарского, Александра Суворова, Михаила Кутузова! Пусть осенит вас победоносное знамя великого Ленина!»}

\textbf{Оценки потерь.}

\textbf{Чёрный миф - непомерная цена Победы. <<Немцев трупами завалили>>, <<Не жалели людей>>}
Безвозвратные потери армий СССР и Германии с сателлитами (включая военнопленных) — 11,5 млн и 8,6 млн чел. соответственно. Соотношение безвозвратных потерь армий Германии с сателлитами и СССР составляет: 1:1,3. 

\textbf{Оценки: Кривошеев: 24млн (11млн+13млн).}

Начальник управления Минобороны России по увековечиванию памяти погибших при защите Отечества Владимир Попов,13 ноября 2015 года: Современная оценка 26.6млн (12+14)
\end{document} % Конец текста.