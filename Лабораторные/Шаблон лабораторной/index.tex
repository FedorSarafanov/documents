%%%%%%%%%%%%%%%%%%%%%%%%%%%%%%%%%%%%%%%%%%%%%%%%%%%%%%%%%%%%%%%%%%%%%%%%%%%%%%%
\documentclass[a4paper,12pt]{article}

%%%%%%%%%%%%%%%%%%%%%%%%%%%%%%%%%%%%%%%%%%%%%%%%%%%%%%%%%%%%%%%%%%%%%%%%%%%%%%%

\usepackage{extsizes}
\usepackage{cmap}
\usepackage[T2A]{fontenc}
\usepackage[utf8x]{inputenc}
\usepackage[english, russian]{babel}
\usepackage{misccorr}
\usepackage{amssymb,amsfonts,amsmath,amsthm}  % математические дополнения от АМС
\usepackage{envmath}  % многострочные формулы EqSystem
\usepackage{indentfirst} % Включение отступа первой строки раздела
\usepackage[usenames,dvipsnames]{color} % названия цветов

%%%%%%%%%%%%%%%%%%%%%%%%%%%%%%%%%%%%%%%%%%%%%%%%%%%%%%%%%%%%%%%%%%%%%%%%%%%%%%%

	% выбрать цвета:
% \definecolor{BlueGreen}{RGB}{49,152,255}

%%%%%%%%%%%%%%%%%%%%%%%%%%%%%%%%%%%%%%%%%%%%%%%%%%%%%%%%%%%%%%%%%%%%%%%%%%%%%%%

	% назначить цвета при подключении hyperref
\usepackage[unicode, colorlinks, urlcolor=magenta, linkcolor=black, pagecolor=black]{hyperref}
	% linkcolor=
	% цвет гиперссылок внутри документа, по-умолчанию red
	% pagecolor=
	% цвет гиперссылок на другие страницы внутри документа, по-умолчанию red
	% filecolor=
	% цвет гиперссылок, открывающих локальные файлы, по-умолчанию cyan
	% anchorcolor=
	% цвет текста мишени, по-умолчанию black
	% citecolor=
	% цвет библиографических ссылок, по-умолчанию green
	% urlcolor=
	% цвет гиперссылок на сетевые ресурсы, по-умолчанию magenta

%%%%%%%%%%%%%%%%%%%%%%%%%%%%%%%%%%%%%%%%%%%%%%%%%%%%%%%%%%%%%%%%%%%%%%%%%%%%%%%

	% улучшенное форматирование таблиц
\usepackage{makecell} 
\usepackage{multirow} 

%%%%%%%%%%%%%%%%%%%%%%%%%%%%%%%%%%%%%%%%%%%%%%%%%%%%%%%%%%%%%%%%%%%%%%%%%%%%%%%

	% подчеркивания
\usepackage{ulem}

%%%%%%%%%%%%%%%%%%%%%%%%%%%%%%%%%%%%%%%%%%%%%%%%%%%%%%%%%%%%%%%%%%%%%%%%%%%%%%%

	% для вставки изображений
\usepackage{graphicx} 
	%где искать изображения
\graphicspath{{img/}}

%%%%%%%%%%%%%%%%%%%%%%%%%%%%%%%%%%%%%%%%%%%%%%%%%%%%%%%%%%%%%%%%%%%%%%%%%%%%%%%
	
	% поля страницы 
\usepackage{geometry}
\geometry{left=3cm,right=2cm,top=3cm,bottom=3cm,bindingoffset=0cm}

%%%%%%%%%%%%%%%%%%%%%%%%%%%%%%%%%%%%%%%%%%%%%%%%%%%%%%%%%%%%%%%%%%%%%%%%%%%%%%%

	% Cтиль оформления chapter
% \usepackage[Glenn]{fncychap} 
	% Всего имеется семь возможных стилей: 
	% Sonny, Lenny, Glenn, Conny, Rejne, Bjarne, Bjornstrup.

%%%%%%%%%%%%%%%%%%%%%%%%%%%%%%%%%%%%%%%%%%%%%%%%%%%%%%%%%%%%%%%%%%%%%%%%%%%%%%%

	%Колинтулы страниц
\usepackage{fancyhdr} 

%%%%%%%%%%%%%%%%%%%%%%%%%%%%%%%%%%%%%%%%%%%%%%%%%%%%%%%%%%%%%%%%%%%%%%%%%%%%%%%

	% полуторный интервал
\linespread{1.3} 

	% стиль пробелов: французский - все пробелы примерно одинаковые
\frenchspacing 

	% Элементы списка второго уровня с скобочкой вместо точки
\renewcommand{\labelenumii}{\theenumii)} 

%%%%%%%%%%%%%%%%%%%%%%%%%%%%%%%%%%%%%%%%%%%%%%%%%%%%%%%%%%%%%%%%%%%%%%%%%%%%%%%  % преамбула
%%%%%%%%%%%%%%%%%%%%%%%%%%%%%%%%%%%%%%%%%%%%%%%%%%%%%%%%%%%%%%%%%%%%%%%%%%%%%%%

\newcommand{\labauthors}{Сарафанов Ф.Г., Сидоров Д.А.}
\newcommand{\labnumber}{27}
\newcommand{\labtheme}{тема}

\newcommand{\ddt}{$\ \pm\ 0.2\ \text{с}$}
\newcommand{\ddtv}{$\ \pm\ 0.8\ \text{с}$}
\newcommand{\ddh}{$\ \pm\ 0.1\ \text{см}$}
\newcommand{\dm}{\Delta{}m}
\newcommand{\Dh}{\Delta{}x}
\newcommand{\Dl}{\Delta{}(\lambda)}
\newcommand{\dmsr}{<\Delta{}m>}
\newcommand{\el}{\varepsilon(\lambda)}

%%%%%%%%%%%%%%%%%%%%%%%%%%%%%%%%%%%%%%%%%%%%%%%%%%%%%%%%%%%%%%%%%%%%%%%%%%%%%%%
%%%%%%%%%%%%%%%%%%%%%%%%%%%%%%%%%%%%%%%%%%%%%%%%%%%%%%%%%%%%%%%%%%%%%%%%%%%%%%%
	%применим колонтитул к стилю страницы
\pagestyle{fancy} 
	%очистим "шапку" страницы
\fancyhead{} 
	%слева сверху на четных и справа на нечетных
\fancyhead[LE,RO]{\labauthors} 
	%справа сверху на четных и слева на нечетных
\fancyhead[LO, RE]{Отчёт по лабораторной работе №\labnumber} 
	%очистим "подвал" страницы
\fancyfoot{} 
	% номер страницы в нижнем колинтуле в центре
\fancyfoot[CO,CE]{\thepage} 

%%%%%%%%%%%%%%%%%%%%%%%%%%%%%%%%%%%%%%%%%%%%%%%%%%%%%%%%%%%%%%%%%%%%%%%%%%%%%%% % колинтулы на страницах
%%%%%%%%%%%%%%%%%%%%%%%%%%%%%%%%%%%%%%%%%%%%%%%%%%%%%%%%%%%%%%%%%%%%%%%%%%%%%%%

\begin{document}

\section{Отчёт по лабораторной работе №\labnumber \\ <<\labtheme>>}

% \subsection{Физические основы лабораторной работы}

В лабораторной работе исследуется равноускоренное движение на установке <<машина Атвуда>>.

Погрешности, используемые в работе: погрешность секундомера ---  $\Delta\,t=0.01\ c$, погрешность измерения длины ---  $\Delta\,h=0.5\ \text{см}$, погрешность известной массы грузов $M$ ---  $\Delta\,M=0.5\  \text{г}$, погрешность масс перегрузков $m_1, m_2$ --- $\Delta\,m=0.05\  \text{г}$.

Запишем 2 закон Ньютона для грузов $M+m_1$ (слева) и $M+m_1$ (справа):

\begin{EqSystem}
	(M+m_1)\vec{a_1}=(M+m_1)\vec{g}+\vec{T_1}\\
	(M+m_2)\vec{a_2}=(M+m_2)\vec{g}+\vec{T_2}
\end{EqSystem}

Спроецируем на ось X, направленную вертикально вниз:

\begin{EqSystem}
	\label{eq:ax}
	(M+m_1){{a_1}_x}=(M+m_1){g}-{T_1}\\
	(M+m_2){{a_2}_x}=(M+m_2){g}-{T_2}
\end{EqSystem}

% \begin{equation}
% X(\omega) = 
%  \begin{cases}

%  \end{cases}
% \end{equation}

Нить предполагается невесомой. Тогда можно записать 2 закон Ньютона для участка нити длиной $\Delta\,L\rightarrow0$. На участок цепи действуют силы натяжения нити и тормозящая сила %(см. рис. \ref{}):

\begin{gather}
	\label{eq:dl}
	F=ma\\
	m\vec{a_{\Delta\,L}}=\vec{F_\text{т}}+\vec{T_1}+\vec{T_2}
\end{gather}

Из условия невесомости масса участка равна нулю. Учитывая это, запишем проекцию (\ref{eq:dl}) на ось X:

\begin{gather}
\label{eq:TTF}
	T_2-T_1=F_\text{т}
\end{gather}

Однако, из третьего закона Ньютона можно обобщить это равенство на произвольную длину нити, так как на каждом участке силы будут транзитивно равны силе, приложенной от предыдущего участка нити.

Рассмотрим нерастяжимую нить. Сдвинем без ускорения нить на $\Delta\,x$ за время $\Delta\,t$. Из условия нерастяжимости грузы пройдут равное расстояние по модулю, но противоположное по направлению. Запишем скорость этих точек по определению: 

\begin{gather}
	\label{eq:dx}
	v_{1x}=\frac{\Delta\,x}{\Delta\,t},	v_{2x}=\frac{-\Delta\,x}{\Delta\,t}\Rightarrow\\
	v_{1x}=-v_{2x}
\end{gather}

Возьмем производную по времени от скорости (\ref{eq:dx}), по определению это будет проекция ускорения грузов на ось X:

\begin{gather}
	v_{1x}=-v_{2x}\\
	\frac{d}{dt}{v_{1x}}=-\frac{d}{dt}{v_{2x}}\\
	a_{1x}=-a_{2x}\label{eq:dv}
\end{gather}

Перепишем систему уравнений (\ref{eq:ax}) с учетом невесомости (\ref{eq:TTF}) и нерастяжимости (\ref{eq:dv}) нити:

\begin{equation}
\begin{cases}
	(M+m_1){-{a_2}_x}=(M+m_1){g}-{T_1}\\
	(M+m_2){{a_2}_x}=(M+m_2){g}-{T_1+F_\text{т}}\label{eru2}
 \end{cases}
\end{equation}

Выразим отсюда ускорение, вычитая уравнения в системе (\ref{eru2}):

\begin{gather}
	\label{eq:a2x}
	a_{2x}=\frac{(m_2-m_1)g-F_\text{т}}{2M+m_1+m_2}
\end{gather}

Как видно из уравнения (\ref{eq:a2x}), ускорение блоков зависит от тормозящей силы. Для того, чтобы применить это уравнение, необходимо найти физический смысл этой силы и её зависимость от известных величин.

% Можно предположить, что в силе трения есть свободный член $F_0$, неизменный во времени. Неизменно во времени сухое трение. 

% Итак, член $F_0$ -- это  сухое трение в установке.

Можно выдвинуть несколько гипотез о тормозящей силе : $F_\text{т}=F_0+?$

\subsection{Гипотеза первая. $F_\text{т}=F(v)$}
Тормозящая сила зависит от скорости, где-то возникает вязкое трение. Это можно проверить, сняв зависимость $h(t^2)$ для разных перегрузков. 

Рассчитаем прямоугольники погрешностей измерений.

\begin{gather*}
	\Delta\,h=0.5\ \text{cm}\\
	\Delta\,(t^2)=2t\Delta\,t
\end{gather*}

% Максимальная абсолютная погрешность времени составляет для максимального замеренного времени $\tau=4.49$ секунды $\Delta\,(\tau^2)=2\cdot4.49\cdot0.01=0.08$ секунды, откуда следует, что изобразить прямоугольники погрешностей на графике (\ref{fig1}) на данном масштабе нельзя. 

\begin{figure}[h]
\begin{center}
\includegraphics*[width=1\textwidth]{img/ex1.png}
\caption{\label{fig1}Эскиз графика зависимости $h(t^2)$}
\end{center}
\end{figure}

Как видно из графика, все три груза двигались с постоянным ускорением --- следовательно, гипотеза $F_\text{т}=F(v)$ неверна. 

% На графике видно небольшое отклонение от прямой больше размера прямоугольника погрешностей. Это опыты, в которые была внесена ошибка измерения. Предположительно --- из-за магнита, который отрывал груз в разное, отличное от начального, время.

\subsection{Гипотеза вторая. $F_\text{т}=F_0+F(a)=F_0+\lambda{}a$}

Перепишем уравнение (\ref{eq:a2x}) с учетом $F_\text{т}=F_0+F(a)=F_0+\lambda{}a$. 

\begin{gather}
	\label{eq:a-g}
	a_{2x}=\frac{(m_2-m_1)g-F_0}{2M+m_1+m_2+\lambda}
\end{gather}

Пусть $m_2-m_1$ будет $\Delta\,m$, а $m_1+m_2$  в опытах будем брать постоянной. Тогда уравнение (\ref{eq:a-g}) можно записать в виде:

\begin{gather}
	\label{eq:a-dm}
	a_{2x}=\Delta\,m{}\frac{g}{2M+m_1+m_2+\lambda}-\frac{F0}{2M+m_1+m_2+\lambda}
\end{gather}

Это ничто иное, как уравнение прямой. Таким образом, сняв зависимость $a(\Delta\,m)$, и убедившись в том, что это прямая, мы можем рассчитать уравнение регрессионной прямой, соответствующей зависимости $a(\Delta\,m)$, вычислить её угловой коэффициент и вычислить $\lambda$, а затем вычислить из неё же сдвиг графика от нуля и подставив $\lambda$  в свободный член найти $F_0$.

Снимать зависимость $a(\Delta\,m)$ можно следующим образом: набрав массу перегрузков на левом грузе, менять $\Delta\,m$ перекладыванием части перегрузков с левого груза на правый. Таким образом суммарная масса перегрузков будет постоянной, а $\Delta\,m$ уменьшаться. Будем измерять время падения груза и вычислять ускорение по формуле (\ref{eq:a-h})

\begin{gather}
	\label{eq:a-h}
	a=\frac{2h}{t^2}
\end{gather}

Рассчитаем погрешности для косвенно измеряемого ускорения(\ref{eq:a-err}):

\begin{gather}
	\label{eq:a-err}
	\varepsilon\,(a)=\frac{2\Delta\,h}{h}+\frac{\Delta\,(t^2)}{t^2}=\\
	=\frac{2\Delta\,h}{h}+\frac{\Delta\,t}{t}\\
	\Delta\,(a)=\varepsilon\,(a)\cdot\,a=\varepsilon\,(a)\cdot\frac{2h}{t^2}=\\
	=\frac{4t\Delta\,h+2\Delta\,t\,h}{t^3}
\end{gather}

Таблица экспериментальных результатов доступна в протоколе лабораторной работы. Построим график зависимости (рис. \ref{fig:a-m}, стр. \pageref{fig:a-m}).

Так как масштаб не позволяет отобразить прямоугольники погрешностей, сделаем выносные чертежи (рис. \ref{fig:a-m-2}, стр. \pageref{fig:a-m-2}) с такими же осями и единицами измерения, как и на (рис. \ref{fig:a-m}, стр. \pageref{fig:a-m}) для каждой из пяти точек в таком масштабе,чтобы отображаемая область графика была в 25 раз больше прямоугольника погрешностей в данной точке.


\begin{figure}[h]
\begin{minipage}[h]{1\linewidth}
	% \begin{figure}[h]
	\begin{center}
	\includegraphics*[width=1\textwidth]{img/ex_22.png}
	\caption{\label{fig:a-m}Эскиз графика зависимости $a(\Delta\,m)$}
	\end{center}
	% \end{figure}
\end{minipage}
\vfill
\begin{minipage}[h]{1\linewidth}
	% \begin{figure}[h]
	\begin{center}
	\includegraphics*[width=1\textwidth]{img/ex_2-5.png}
	\caption{\label{fig:a-m-2}Прямоугольники погрешностей с графика $a(\Delta\,m)$}
	\end{center}
	% \end{figure}
\end{minipage}

\end{figure}
\subsection{Теория лабораторной работы}

В лабораторной работе исследуется .

Погрешности, используемые в работе: 

Запишем :
\begin{EqSystem}
\end{EqSystem}

Спроецируем на ось X, направленную :
\begin{EqSystem}
	\label{eq:}
\end{EqSystem}

\subsection{Вывод}

В результате проделанной работы были выполнены следующие пункты.

Опровергнута гипотеза 

Снята линейная зависимость  откуда сделан вывод о .

Снята зависимость ,
для которой расчитана соответствующая погрешность (\ref{})

Оценены коэффициенты $\lambda$  и $F_0$ методом .

Изучено уравнение динамики вращательного движения (ОУДВД) и физический смысл момента инерции, а также методы его вычисления.

Рассчитано значение коэффициента 

Определена правильность определения

Сравнение , полученного разными способами, показывает: в пределах погрешностей измерений можно утверждать следующее: 


В пределах погрешностей измерений были построены графики зависимостей.

В работе рассчитаны погрешности для всех косвенных измерений, размеры прямоугольников ошибок. 

Все точки на графиках укладываются на  теоретические графики в пределах размеров их прямоугольников ошибок.

Подтверждена 

\newpage
\section*{Приложение 1. Графики зависимостей} % (fold)
\label{sec:figures}

% section figures (end)

\end{document}
