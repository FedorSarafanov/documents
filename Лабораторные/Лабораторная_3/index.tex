%%%%%%%%%%%%%%%%%%%%%%%%%%%%%%%%%%%%%%%%%%%%%%%%%%%%%%%%%%%%%%%%%%%%%%%%%%%%%%%
\input{addons/lab-head}  % преамбула
%%%%%%%%%%%%%%%%%%%%%%%%%%%%%%%%%%%%%%%%%%%%%%%%%%%%%%%%%%%%%%%%%%%%%%%%%%%%%%%

\newcommand{\labauthors}{Сарафанов Ф.\,Г}%., Сидоров Д.\,А.}
\newcommand{\labauthor}{Сарафанов~Ф.\,Г.}
\newcommand{\labnumber}{27}
\newcommand{\labtheme}{Определение отношения заряда электрона к его массе}

\newcommand{\ddt}{$\ \pm\ 0.2\ \text{с}$}
\newcommand{\ddtv}{$\ \pm\ 0.8\ \text{с}$}
\newcommand{\ddh}{$\ \pm\ 0.1\ \text{см}$}
\newcommand{\dm}{\Delta{}m}
\newcommand{\Dh}{\Delta{}x}
\newcommand{\Dl}{\Delta{}(\lambda)}
\newcommand{\dmsr}{<\Delta{}m>}
\newcommand{\el}{\varepsilon(\lambda)}

%%%%%%%%%%%%%%%%%%%%%%%%%%%%%%%%%%%%%%%%%%%%%%%%%%%%%%%%%%%%%%%%%%%%%%%%%%%%%%%
\input{addons/lab-kol} % колинтулы на страницах
%%%%%%%%%%%%%%%%%%%%%%%%%%%%%%%%%%%%%%%%%%%%%%%%%%%%%%%%%%%%%%%%%%%%%%%%%%%%%%%

\begin{document}

	\begin{titlepage}

	\begin{center}

	{\small\textsc{Нижегородский государственный университет имени Н.\,И. Лобачевского}}
	\vskip 1pt \hrule \vskip 3pt
	{\small\textsc{Радиофизический факультет}}

	\vfill

	{\Large Отчет по лабораторной работе №\labnumber\vskip 12pt\bfseries \labtheme}
		
	\end{center}

	\vfill
		
	\begin{flushright}
		{Выполнил студент 410 группы\\\labauthor\vskip 12pt Принял:\\ Менсов С.\,Н.}
	\end{flushright}
		
	\vfill
		
	\begin{center}
		Нижний Новгород, 2016
	\end{center}

	\end{titlepage}

\section{Отчёт по лабораторной работе №\labnumber \\ <<\labtheme>>}

% \input{ch_0_phy}

\textbf{Цель работы:} изучение характера движения заряженных частиц в однородном магнитном поле и определение удельного заряда электрона методом магнитной фокусировки и методом отклонения в известных полях.

\textbf{Оборудование:}
экспериментальная установка (ЭЛТ и блок питания), коммутатор, амперметр постоянного тока, источник питания постоянного тока 

\textbf{Приборные погрешности:} $\Delta{U}=62.6\ \text{В}$, $\Delta{I}=0.015\ \text{А}$, $\Delta{K}=0.01\ \text{м}$. 

\subsection{Измерение удельного заряда электрона методом отклонения земным магнитным полем}

В данном эксперименте ЭЛТ выставлялась так, чтобы её продольная ось была сонаправлена с линиями магнитного поля земли: известна горизонтальная составляющая поля  $B_\text{з}=0.186\ \text{гаусса}\ =1.86\cdot10^{-5}\ \text{Тл}$ и наклонение $\alpha=70^{\circ}$. Отсюда напряженность магнитного поля вдоль ЭЛТ составляет $B=\frac{B_\text{з}}{cos(\alpha)}=6.36\cdot10^{-6}\ \text{Тл}$.

Выберем декартову систему координат так, чтобы начало координат лежало на конце второго анода, ось $Z$ совпадала с продольной осью ЭЛТ.

Если на анод ЭЛТ подавать напряжение $U_a$, то из него электроны будут вылетать с кинетической энергией 
$$\frac{mv_{0z}^2}{2}=U_a\cdot{}e$$
и скоростью
$$v_{0z}=\sqrt{\frac{2\cdot{}U_a\cdot{}e}{m}}$$

Обозначим удельный заряд $\frac{e}{m}=\eta$. Тогда $v_{0z}=\sqrt{2U_a\eta}$.

На заряд будет действовать сила Лоренца, направленная перпендикулярно скорости электрона, закручивающая его по окружности радиуса $R$.

$$F_\text{л}=ma$$
$$eVB=\frac{mv_{0z}^2}{R}$$

Отсюда

$$\eta=\frac{v_{0z}}{RB}$$

Радиус можно найти по отклонению проецируемого на экран ЭЛТ пятна $K$, формула приближенно упрощается для $K<<L_1$:

\begin{equation}
	\eta=\frac{8K^2U_a}{B^2L^4}	
	\label{eta1}
\end{equation}

Рассчитаем погрешность формулы (\ref{eta1}):

\begin{equation}
	\varepsilon{(\eta)}=\frac{2\Delta{K}}{K}+\frac{\Delta{U}}{U_a}
\end{equation}

Проведем несколько опытов, используя противоположные направления магнитного поля и разные значения $U_a$.
% 178402485704.8718 0.28619582417582423
% 131071213987.25276 0.3338148717948718
% 119028346035.43253 0.4003682352941177
% 201157904799.881 0.3080605429864253

% 51058046431.32149
% 43753520493.152985
% 47655168852.18371
% 61968813378.662994

\begin{table}[h]
\begin{center}
\begin{tabular}{|c|c|c|c|c|c|}

\hline
$U_a$, В & $\alpha$, град & $K$, м & $\frac{e}{m}$ (СИ) & $\varepsilon{(\frac{e}{m})}$ & $\Delta{(\frac{e}{m})}$\\
\hline
\multirow{2}{*}{$1300$} & $+90$ & $0.007$ & $1.78\cdot10^{11}$ & $0.28$ & $5.10\cdot10^{10}$ \\ 

\cline{2-6}
						& $-90$ & $0.006$ & $1.31\cdot10^{11}$ & $0.33$ & $4.37\cdot10^{10}$ \\ \hline
\multirow{2}{*}{$1700$} & $+90$ & $0.005$ & $1.19\cdot10^{11}$ & $0.40$ & $4.76\cdot10^{10}$ \\
\cline{2-6}
						& $-90$ &$0.0065$ & $2.01\cdot10^{11}$ & $0.30$ & $6.19\cdot10^{10}$ \\ \hline

\end{tabular}
\end{center}
% \caption{\label{tab:t-phi}Зависимость периода математического маятника от угла отклонения, $T(\phi)$}
\end{table} 

По данным таблицы среднеквадратичное отклонение будет составлять $5.11\cdot10^{10}$, среднее значение $\frac{e}{m}=1.57\cdot10^{11}$.

Интервал ошибок будет составлять от $1.06\cdot10^{11}$ до $2.08\cdot10^{11}$. Этим методом нашли $\frac{e}{m}$ с точностью до порядка.

\begin{figure}[h!]
	\centering
	\includegraphics[width=\textwidth]{figure_1.png}
	\caption{Теоретическая зависимость отклонения пятна от напряжения и реальные результаты с прямоугольниками ошибок}
	\label{fig:figure1}
\end{figure}

\newpage

\subsection{Измерение удельного заряда электрона методом фокусировки пучка продольным магнитным полем селеноида}

Если вдоль оси трубки создать постоянное магнитное поле $B_z$, которое можно рассчитать по формуле
\begin{equation}
	B=\mu_0\cdot{}n_0\cdot{}I=4\pi\cdot10^{-7}\cdot8400\cdot{}I
\end{equation}
то электронный пучок, вылетевший из электронной пушки с начальной скоростью $v_0=\sqrt{2\cdot{U_a}\cdot\eta}$, пройдет в ЭЛТ в сонаправленном продольной оси (oZ) ЭЛТ магнитном поле напряженностью $B=\mu_0nI$. 

Для того, чтобы поле Bx действовало на электрон, необходимо, чтобы его скорость имела поперечную составляющую , перпендикулярную $v_z$. В этом случае в плоскости $XoY$ электрон под действием силы  будет равномерно двигаться по окружности, радиус которой определится из второго закона динамики:

\begin{equation}
	m\frac{v_0^2}{R}=ev_0B_z
\end{equation}
\begin{equation}
	R=\frac{mv_0}{eB_z}
\end{equation}

На неком начальном участке пути, меньшем на порядок относительно длины трубки, пучок проходит сквозь отклоняющие пластины с напряженностью электрического поля $E=\frac{U_\text{откл}}{d}$, где $d$ --- расстояние между пластинами.

Если пренебречь тем, что поля скрещенные, можно предположить, что в пластинах пучок приобретает произвольную скорость, лежащую в плоскость (XoY), перпендикулярной oZ. Ясно, что он будет закручиваться магнитным полем, а так как проекция начальной скорости $v_0z$ не изменилась, то
его траектория будет представлять собой винтовую линию, нанесенную на цилиндр радиуса $R$, и электроны пучка будут двигаться по спирали и через время $nT=\tau$ пересекать ось oZ.

Обозначим $\eta=\frac{e}{m}$ и $\omega=\eta{}B_z$ (циклотронная частота). Тогда можем переписать формулу как

\begin{equation}
	R=\frac{v_0}{\omega}
\end{equation}

Решив несколько систем уравнений:

$$
\begin{cases}
nT_1v_{0z}=l\\
(n+1)T_2v_{0z}=l
\end{cases}
$$

$$
\begin{cases}
T_1=\frac{2\pi}{\eta{}B_1}\\
T_2=\frac{2\pi}{\eta{}B_2}
\end{cases}
$$

Число фокусировок находится как $$n=\frac{T_2}{T_1-T_2}=\frac{B_1}{B_2-B_1}=\frac{I_1}{I_2-I_1}$$

А удельный заряд определяется как $$\eta=\frac{8\cdot\pi^2\cdot{}U_a}{l^2(B_2-B_1)^2}$$  через два опыта с последовательными n-й и n+1 фокусировками.

Нужно найти первое такое пересечение на экране ЭЛТ (n фокусировок) и увеличивать ток на соленоиде (увеличивать напряженность магнитного поля) до тех пор, пока не получим на экране (n+1 фокусировку).

$$\eta=\frac{8\cdot\pi^2\cdot{}U_a}{(l\cdot\mu_0\cdot{n_0})^2(I_2-I_1)^2}$$

Рассчитаем погрешность формулы:

$$\Delta{(\eta)}=\frac{8 \pi^{2} \left(4 U_{a} \Delta{I} \left(I_{2} - I_{1}\right) + \Delta{U} \left(I_{2} - I_{1}\right)^{2}\right)}{\mu_{0}^{2} d^{2} l^{2} n_{0}^{2} \left(I_{2} - I_{1}\right)^{4}}$$

Где $d$ --- коэффициент перевода в систему СИ единиц тока.

Провели ряд экспериментов: 

\begin{table}[h]
\begin{center}
\begin{tabular}{|c|c|c|c|c|c|}

\hline
$U_a$, В & $I_1$, А & $I_2$, А & $\frac{e}{m}$ (СИ) & $\Delta{(\frac{e}{m})}$ & $n$\\
\hline
$1200$ & $0.6$ & $1.14$ & $1.78\cdot10^{11}$ & $0.96\cdot10^{10}$  & $1.11$ \\ \hline
$1000$ & $0.54$ & $1.04$ & $1.73\cdot10^{11}$ & $1.12\cdot10^{10}$ & $1.08$ \\ \hline
$1000$ & $0.5$ & $1.08$ & $1.81\cdot10^{11}$ & $1.16\cdot10^{10}$  & $0.86$ \\ \hline
$1100$ & $0.46$ & $1.08$ & $1.74\cdot10^{11}$ & $1.02\cdot10^{10}$ & $0.74$ \\ \hline

\end{tabular}
\end{center}
% \caption{\label{tab:t-phi}Зависимость периода математического маятника от угла отклонения, $T(\phi)$}
\end{table} 


По данным таблицы среднеквадратичное отклонение будет составлять $0.061\cdot10^{11}$, среднее значение $\frac{e}{m}=1.76\cdot10^{11}$.

Доверительный интервал будет составлять от $1.70\cdot10^{11}$ до $1.82\cdot10^{11}$. Этим методом нашли $\frac{e}{m}$ так, что известное табличное значение попадает в доверительный интервал, а экспериментальное значение лежит еще ближе к известному.

\end{document}
