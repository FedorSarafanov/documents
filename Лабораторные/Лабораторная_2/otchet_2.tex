\documentclass[a4paper,12pt]{report}
 
\usepackage{extsizes}
\usepackage{cmap}
\usepackage[T2A]{fontenc}
\usepackage[utf8x]{inputenc}
% \usepackage[russian]{babel}
\usepackage[english, russian]{babel}
% \usepackage{newtx}
% \usepackage{cyrtimes}
\usepackage{misccorr}

%%%%%%%%%%%%%%%%%%%%%%%%%%%%%%%%%%%%%%%%%%%%%%%%%%%%%%%%%%%%%%%%%%%%%%%%%%%%%%%%%%  
\usepackage{graphicx} % для вставки картинок
\graphicspath{{img/}}
\usepackage{amssymb,amsfonts,amsmath,amsthm} % математические дополнения от АМС

% \usepackage{fontspec}
% \usepackage{unicode-math}

\usepackage{indentfirst} % отделять первую строку раздела абзацным отступом тоже
\usepackage[usenames,dvipsnames]{color} % названия цветов
\usepackage{makecell}
\usepackage{multirow} % улучшенное форматирование таблиц
\usepackage{ulem} % подчеркивания
\linespread{1.3} % полуторный интервал
% \renewcommand{\rmdefault}{ftm} % Times New Roman (не работает)
\frenchspacing
\usepackage{geometry}
\geometry{left=3cm,right=2cm,top=3cm,bottom=3cm,bindingoffset=0cm}
\usepackage{titlesec}

% \definecolor{black}{rgb}{0,0,0}
\usepackage[colorlinks, unicode, pagecolor=black]{hyperref}
% \usepackage[unicode]{hyperref} %ссылки
\usepackage{fancyhdr} %загрузим пакет
\pagestyle{fancy} %применим колонтитул
\fancyhead{} %очистим хидер на всякий случай
\fancyhead[LE,RO]{Сарафанов Ф.Г., Сидоров Д.А.} %номер страницы слева сверху на четных и справа на нечетных
\fancyhead[LO, RE]{Отчёт по лабораторной работе №27}
% \fancyhead[LO,RE]{Машина Атвуда} 
\fancyfoot{} %футер будет пустой
\fancyfoot[CO,CE]{\thepage}
\renewcommand{\labelenumii}{\theenumii)}
\newcommand{\ddt}{$\ \pm\ 0.2\ \text{с}$}
\newcommand{\ddtv}{$\ \pm\ 0.8\ \text{с}$}
\newcommand{\ddh}{$\ \pm\ 0.1\ \text{см}$}
\newcommand{\dm}{\Delta{}m}
\newcommand{\Dh}{\Delta{}x}
\newcommand{\Dl}{\Delta{}(\lambda)}
\newcommand{\dmsr}{<\Delta{}m>}
\newcommand{\el}{\varepsilon(\lambda)}
% \usepackage[Glenn]{fncychap} % выбираем стиль Glenn
% \usepackage{amsthm}
\usepackage{envmath}
% \newtheorem{define}{Определение}
% \newtheorem{theorem}{Теорема}
% \newtheorem{problem}{Задача}

\begin{document}

% \section{}
% \chapter{1}
\section{Отчёт по лабораторной работе №27 \\ <<Изучение равноускоренного движения при помощи машины Атвуда>>}

% \subsection{Физические основы лабораторной работы}

В лабораторной работе исследуется равноускоренное движение на установке <<машина Атвуда>>.

Погрешности, используемые в работе: погрешность секундомера ---  $\Delta\,t=0.01\ c$, погрешность измерения длины ---  $\Delta\,h=0.5\ \text{см}$, погрешность известной массы грузов $M$ ---  $\Delta\,M=0.5\  \text{г}$, погрешность масс перегрузков $m_1, m_2$ --- $\Delta\,m=0.05\  \text{г}$.

Запишем 2 закон Ньютона для грузов $M+m_1$ (слева) и $M+m_1$ (справа):

\begin{EqSystem}
	(M+m_1)\vec{a_1}=(M+m_1)\vec{g}+\vec{T_1}\\
	(M+m_2)\vec{a_2}=(M+m_2)\vec{g}+\vec{T_2}
\end{EqSystem}

Спроецируем на ось X, направленную вертикально вниз:

\begin{EqSystem}
	\label{eq:ax}
	(M+m_1){{a_1}_x}=(M+m_1){g}-{T_1}\\
	(M+m_2){{a_2}_x}=(M+m_2){g}-{T_2}
\end{EqSystem}

% \begin{equation}
% X(\omega) = 
%  \begin{cases}

%  \end{cases}
% \end{equation}

Нить предполагается невесомой. Тогда можно записать 2 закон Ньютона для участка нити длиной $\Delta\,L\rightarrow0$. На участок цепи действуют силы натяжения нити и тормозящая сила %(см. рис. \ref{}):

\begin{gather}
	\label{eq:dl}
	F=ma\\
	m\vec{a_{\Delta\,L}}=\vec{F_\text{т}}+\vec{T_1}+\vec{T_2}
\end{gather}

Из условия невесомости масса участка равна нулю. Учитывая это, запишем проекцию (\ref{eq:dl}) на ось X:

\begin{gather}
\label{eq:TTF}
	T_2-T_1=F_\text{т}
\end{gather}

Однако, из третьего закона Ньютона можно обобщить это равенство на произвольную длину нити, так как на каждом участке силы будут транзитивно равны силе, приложенной от предыдущего участка нити.

Рассмотрим нерастяжимую нить. Сдвинем без ускорения нить на $\Delta\,x$ за время $\Delta\,t$. Из условия нерастяжимости грузы пройдут равное расстояние по модулю, но противоположное по направлению. Запишем скорость этих точек по определению: 

\begin{gather}
	\label{eq:dx}
	v_{1x}=\frac{\Delta\,x}{\Delta\,t},	v_{2x}=\frac{-\Delta\,x}{\Delta\,t}\Rightarrow\\
	v_{1x}=-v_{2x}
\end{gather}

Возьмем производную по времени от скорости (\ref{eq:dx}), по определению это будет проекция ускорения грузов на ось X:

\begin{gather}
	v_{1x}=-v_{2x}\\
	\frac{d}{dt}{v_{1x}}=-\frac{d}{dt}{v_{2x}}\\
	a_{1x}=-a_{2x}\label{eq:dv}
\end{gather}

Перепишем систему уравнений (\ref{eq:ax}) с учетом невесомости (\ref{eq:TTF}) и нерастяжимости (\ref{eq:dv}) нити:

\begin{equation}
\begin{cases}
	(M+m_1){-{a_2}_x}=(M+m_1){g}-{T_1}\\
	(M+m_2){{a_2}_x}=(M+m_2){g}-{T_1+F_\text{т}}\label{eru2}
 \end{cases}
\end{equation}

Выразим отсюда ускорение, вычитая уравнения в системе (\ref{eru2}):

\begin{gather}
	\label{eq:a2x}
	a_{2x}=\frac{(m_2-m_1)g-F_\text{т}}{2M+m_1+m_2}
\end{gather}

Как видно из уравнения (\ref{eq:a2x}), ускорение блоков зависит от тормозящей силы. Для того, чтобы применить это уравнение, необходимо найти физический смысл этой силы и её зависимость от известных величин.

% Можно предположить, что в силе трения есть свободный член $F_0$, неизменный во времени. Неизменно во времени сухое трение. 

% Итак, член $F_0$ -- это  сухое трение в установке.

Можно выдвинуть несколько гипотез о тормозящей силе : $F_\text{т}=F_0+?$

\subsection{Гипотеза первая. $F_\text{т}=F(v)$}
Тормозящая сила зависит от скорости, где-то возникает вязкое трение. Это можно проверить, сняв зависимость $h(t^2)$ для разных перегрузков. 

Рассчитаем прямоугольники погрешностей измерений.

\begin{gather*}
	\Delta\,h=0.5\ \text{cm}\\
	\Delta\,(t^2)=2t\Delta\,t
\end{gather*}

% Максимальная абсолютная погрешность времени составляет для максимального замеренного времени $\tau=4.49$ секунды $\Delta\,(\tau^2)=2\cdot4.49\cdot0.01=0.08$ секунды, откуда следует, что изобразить прямоугольники погрешностей на графике (\ref{fig1}) на данном масштабе нельзя. 

\begin{figure}[h]
\begin{center}
\includegraphics*[width=1\textwidth]{img/ex1.png}
\caption{\label{fig1}Эскиз графика зависимости $h(t^2)$}
\end{center}
\end{figure}

Как видно из графика, все три груза двигались с постоянным ускорением --- следовательно, гипотеза $F_\text{т}=F(v)$ неверна. 

% На графике видно небольшое отклонение от прямой больше размера прямоугольника погрешностей. Это опыты, в которые была внесена ошибка измерения. Предположительно --- из-за магнита, который отрывал груз в разное, отличное от начального, время.

\subsection{Гипотеза вторая. $F_\text{т}=F_0+F(a)=F_0+\lambda{}a$}

Перепишем уравнение (\ref{eq:a2x}) с учетом $F_\text{т}=F_0+F(a)=F_0+\lambda{}a$. 

\begin{gather}
	\label{eq:a-g}
	a_{2x}=\frac{(m_2-m_1)g-F_0}{2M+m_1+m_2+\lambda}
\end{gather}

Пусть $m_2-m_1$ будет $\Delta\,m$, а $m_1+m_2$  в опытах будем брать постоянной. Тогда уравнение (\ref{eq:a-g}) можно записать в виде:

\begin{gather}
	\label{eq:a-dm}
	a_{2x}=\Delta\,m{}\frac{g}{2M+m_1+m_2+\lambda}-\frac{F0}{2M+m_1+m_2+\lambda}
\end{gather}

Это ничто иное, как уравнение прямой. Таким образом, сняв зависимость $a(\Delta\,m)$, и убедившись в том, что это прямая, мы можем рассчитать уравнение регрессионной прямой, соответствующей зависимости $a(\Delta\,m)$, вычислить её угловой коэффициент и вычислить $\lambda$, а затем вычислить из неё же сдвиг графика от нуля и подставив $\lambda$  в свободный член найти $F_0$.

Снимать зависимость $a(\Delta\,m)$ можно следующим образом: набрав массу перегрузков на левом грузе, менять $\Delta\,m$ перекладыванием части перегрузков с левого груза на правый. Таким образом суммарная масса перегрузков будет постоянной, а $\Delta\,m$ уменьшаться. Будем измерять время падения груза и вычислять ускорение по формуле (\ref{eq:a-h})

\begin{gather}
	\label{eq:a-h}
	a=\frac{2h}{t^2}
\end{gather}

Рассчитаем погрешности для косвенно измеряемого ускорения(\ref{eq:a-err}):

\begin{gather}
	\label{eq:a-err}
	\varepsilon\,(a)=\frac{2\Delta\,h}{h}+\frac{\Delta\,(t^2)}{t^2}=\\
	=\frac{2\Delta\,h}{h}+\frac{\Delta\,t}{t}\\
	\Delta\,(a)=\varepsilon\,(a)\cdot\,a=\varepsilon\,(a)\cdot\frac{2h}{t^2}=\\
	=\frac{4t\Delta\,h+2\Delta\,t\,h}{t^3}
\end{gather}

Таблица экспериментальных результатов доступна в протоколе лабораторной работы. Построим график зависимости (рис. \ref{fig:a-m}, стр. \pageref{fig:a-m}).

Так как масштаб не позволяет отобразить прямоугольники погрешностей, сделаем выносные чертежи (рис. \ref{fig:a-m-2}, стр. \pageref{fig:a-m-2}) с такими же осями и единицами измерения, как и на (рис. \ref{fig:a-m}, стр. \pageref{fig:a-m}) для каждой из пяти точек в таком масштабе,чтобы отображаемая область графика была в 25 раз больше прямоугольника погрешностей в данной точке.


\begin{figure}[h]
\begin{minipage}[h]{1\linewidth}
	% \begin{figure}[h]
	\begin{center}
	\includegraphics*[width=1\textwidth]{img/ex_22.png}
	\caption{\label{fig:a-m}Эскиз графика зависимости $a(\Delta\,m)$}
	\end{center}
	% \end{figure}
\end{minipage}
\vfill
\begin{minipage}[h]{1\linewidth}
	% \begin{figure}[h]
	\begin{center}
	\includegraphics*[width=1\textwidth]{img/ex_2-5.png}
	\caption{\label{fig:a-m-2}Прямоугольники погрешностей с графика $a(\Delta\,m)$}
	\end{center}
	% \end{figure}
\end{minipage}

\end{figure}
\subsection{Теория лабораторной работы}

В лабораторной работе исследуется равноускоренное движение на установке <<машина Атвуда>>.

Погрешности, используемые в работе: погрешность секундомера ---  $\Delta\,t=0.01\ c$, погрешность измерения длины ---  $\Delta\,h=0.5\ \text{см}$, погрешность известной массы грузов $M$ ---  $\Delta\,M=0.5\  \text{г}$, погрешность масс перегрузков $m_1, m_2$ --- $\Delta\,m=0.05\  \text{г}$.

Запишем 2 закон Ньютона для грузов $M+m_1$ (слева) и $M+m_1$ (справа):
\begin{EqSystem}
	(M+m_1)\vec{a_1}=(M+m_1)\vec{g}+\vec{T_1}\\
	(M+m_2)\vec{a_2}=(M+m_2)\vec{g}+\vec{T_2}
\end{EqSystem}

Спроецируем на ось X, направленную вертикально вниз:
\begin{EqSystem}
	\label{eq:ax}
	(M+m_1){{a_1}_x}=(M+m_1){g}-{T_1}\\
	(M+m_2){{a_2}_x}=(M+m_2){g}-{T_2}
\end{EqSystem}

% \begin{equation}
% X(\omega) = 
%  \begin{cases}

%  \end{cases}
% \end{equation}
\textbf{Предоставим решение контрольного вопроса №1.}

Нить предполагается невесомой. Тогда можно записать 2 закон Ньютона для участка нити длиной $\Delta\,L\rightarrow0$. На участок цепи действуют силы натяжения нити и тормозящая сила %(см. рис. \ref{}):
\begin{gather}
	\label{eq:dl}
	F=ma\\
	m\vec{a_{\Delta\,L}}=\vec{F_\text{т}}+\vec{T_1}+\vec{T_2}
\end{gather}

Из условия невесомости масса участка равна нулю. Учитывая это, запишем проекцию (\ref{eq:dl}) на ось X:
\begin{gather}
\label{eq:TTF}
	T_2-T_1=F_\text{т}
\end{gather}

Однако, из третьего закона Ньютона можно обобщить это равенство на произвольную длину нити, так как на каждом участке силы будут транзитивно равны силе, приложенной от предыдущего участка нити.

Рассмотрим нерастяжимую нить. Сдвинем без ускорения нить на $\Delta\,x$ за время $\Delta\,t$. Из условия нерастяжимости грузы пройдут равное расстояние по модулю, но противоположное по направлению. Запишем скорость этих точек по определению: 
\begin{gather}
	\label{eq:dx}
	v_{1x}=\frac{\Delta\,x}{\Delta\,t},	v_{2x}=\frac{-\Delta\,x}{\Delta\,t}\Rightarrow\\
	v_{1x}=-v_{2x}
\end{gather}

Возьмем производную по времени от скорости (\ref{eq:dx}), по определению это будет проекция ускорения грузов на ось X:
\begin{gather}
	v_{1x}=-v_{2x}\\
	\frac{d}{dt}{v_{1x}}=-\frac{d}{dt}{v_{2x}}\\
	a_{1x}=-a_{2x}\label{eq:dv}
\end{gather}

Перепишем систему уравнений (\ref{eq:ax}) с учетом невесомости (\ref{eq:TTF}) и нерастяжимости (\ref{eq:dv}) нити:
\begin{equation}
\begin{cases}
	(M+m_1){-{a_2}_x}=(M+m_1){g}-{T_1}\\
	(M+m_2){{a_2}_x}=(M+m_2){g}-{T_1+F_\text{т}}\label{eru2}
 \end{cases}
\end{equation}

Выразим отсюда ускорение, вычитая уравнения в системе (\ref{eru2}):
\begin{gather}
	\label{eq:a2x}
	a_{2x}=\frac{(m_2-m_1)g-F_\text{т}}{2M+m_1+m_2}
\end{gather}

Как видно из уравнения (\ref{eq:a2x}), ускорение блоков зависит от тормозящей силы. Для того, чтобы применить это уравнение, необходимо найти физический смысл этой силы и её зависимость от известных величин.

% Можно предположить, что в силе трения есть свободный член $F_0$, неизменный во времени. Неизменно во времени сухое трение. 

% Итак, член $F_0$ -- это  сухое трение в установке.

Можно выдвинуть несколько гипотез о тормозящей силе : $F_\text{т}=F_0+?$

\subsection{Гипотеза первая. $F_\text{т}=F(v)$}
Тормозящая сила зависит от скорости, где-то возникает вязкое трение. Это можно проверить, сняв зависимость $h(t^2)$ для разных перегрузков. 

Рассчитаем прямоугольники погрешностей измерений.
\begin{gather*}
	\Delta\,h=0.5\ \text{cm}\\
	\Delta\,(t^2)=2t\Delta\,t
\end{gather*}

% Максимальная абсолютная погрешность времени составляет для максимального замеренного времени $\tau=4.49$ секунды $\Delta\,(\tau^2)=2\cdot4.49\cdot0.01=0.08$ секунды, откуда следует, что изобразить прямоугольники погрешностей на графике (\ref{fig1}) на данном масштабе нельзя. 

Как видно из графика (см. приложение 1, стр. \pageref{fig:htt}), все три груза двигались с постоянным ускорением --- следовательно, гипотеза $F_\text{т}=F(v)$ неверна. 

% На графике видно небольшое отклонение от прямой больше размера прямоугольника погрешностей. Это опыты, в которые была внесена ошибка измерения. Предположительно --- из-за магнита, который отрывал груз в разное, отличное от начального, время.

\subsection{Гипотеза вторая. $F_\text{т}=F_0+F(a)=F_0+\lambda{}a$}

Перепишем уравнение (\ref{eq:a2x}) с учетом $F_\text{т}=F_0+F(a)=F_0+\lambda{}a$. 
\begin{gather}
	\label{eq:a-g}
	a_{2x}=\frac{(m_2-m_1)g-F_0}{2M+m_1+m_2+\lambda}
\end{gather}

Пусть $m_2-m_1$ будет $\Delta\,m$, а $m_1+m_2$  в опытах будем брать постоянной. Тогда уравнение (\ref{eq:a-g}) можно записать в виде:
\begin{gather}
	\label{eq:a-dm}
	a_{2x}=\Delta\,m{}\frac{g}{2M+m_1+m_2+\lambda}-\frac{F0}{2M+m_1+m_2+\lambda}
\end{gather}

Это ничто иное, как уравнение прямой. Таким образом, сняв зависимость $a(\Delta\,m)$, и убедившись в том, что это прямая, мы можем рассчитать уравнение регрессионной прямой, соответствующей зависимости $a(\Delta\,m)$, вычислить её угловой коэффициент и вычислить $\lambda$, а затем вычислить из неё же сдвиг графика от нуля и подставив $\lambda$  в свободный член найти $F_0$.

Снимать зависимость $a(\Delta\,m)$ можно следующим образом: набрав массу перегрузков на левом грузе, менять $\Delta\,m$ перекладыванием части перегрузков с левого груза на правый. Таким образом суммарная масса перегрузков будет постоянной, а $\Delta\,m$ уменьшаться. Будем измерять время падения груза и вычислять ускорение по следующей формуле:
\begin{gather}
	\label{eq:a-h}
	a=\frac{2h}{t^2}
\end{gather}

Рассчитаем погрешности для косвенно измеряемого ускорения(\ref{eq:a-err}):
\begin{gather}
	\label{eq:a-err}
	\varepsilon\,(a)=\frac{2\Delta\,h}{h}+\frac{\Delta\,(t^2)}{t^2}=\\
	=\frac{2\Delta\,h}{h}+\frac{\Delta\,t}{t}\\
	\Delta\,(a)=\varepsilon\,(a)\cdot\,a=\varepsilon\,(a)\cdot\frac{2h}{t^2}=\\
	=\frac{4t\Delta\,h+2\Delta\,t\,h}{t^3}
\end{gather}

Таблица экспериментальных результатов доступна в протоколе лабораторной работы. Построим график зависимости (см. приложение 1, стр. \pageref{fig:a-m})
% (рис. \ref{fig:a-m}, стр. \pageref{fig:a-m}).

Так как масштаб не позволяет качественно отобразить прямоугольники погрешностей, сделаем выносные чертежи (см. приложение 1, стр. \pageref{fig:a-m-2}) с такими же осями и единицами измерения, как и на (см. приложение 1, стр. \pageref{fig:a-m}) для каждой из пяти точек в таком масштабе,чтобы отображаемая область графика была в 25 раз больше прямоугольника погрешностей в данной точке.

\textbf{Предоставим решение контрольного вопроса №2.}

При $(m_2-m_1)g<F_0$ ускорение по формуле (\ref{eq:a-g}) будет отрицательным: следовательно, тормозящая сила $F_\text{т}=F_0+\lambda{a}\Longrightarrow F_\text{т}=F_0-\lambda{|a|}$.

\textbf{Предоставим решение контрольного вопроса №3.}

Согласно формуле $F_\text{т}=F_0+\lambda{a}$ при $F_0\ne0$: $F_\text{т}$ может равняться нулю при $\lambda{a}=-F_0$. Так как $\lambda$ положительно, то такое возможно при $a=\frac{-F_0}{\lambda}$

\subsection{Расчет погрешностей $\lambda, F_0$}
% При помощи графика (рис. \ref{fig:a-m}) получили
Оценку коэффициентов $\lambda,\ F_0$ из уравнения (\ref{eq:a-dm}) рассчитаем методом наименьших квадратов (МНК) Гаусса.

В параметрическом линейном регрессионном анализе в качестве \textit{математической зависимости пары переменных ($\Delta{m}, a$) рассматривается линейная зависимость $k\Delta{m}-b$ со случайной нормальной ошибкой}, а именно:

\begin{gather}
	% \label{ex:}	
	a_i=k\Delta{m}_i-b+\sigma_i,\hspace{0.5cm}   i=1,2,\ldots,n,
	% a_i=k\Delta{m}_i-b+\sigma\xi_i,\hspace{0.5cm}   i=1,2,\ldots,n,	
\end{gather}

где случайные величины $\xi_i,\ \ i=1,2,\ldots,n$ имеют стандартное нормальное
распределение и независимы.

Найдем математические ожидания экспериментальных данных:

\begin{gather}
	% \label{ex:}
	<\Delta{}m>=\frac{1}{n}\sum\limits_{i=1}^{n}\Delta\,m_i=32.088\\\nonumber
	<a>=\frac{1}{n}\sum\limits_{i=1}^{n}a_i=30.3377
\end{gather}
% Можно выразить оценочные значения коэффициентов следующим образом:
% Доверительные интервалы для параметров регрессии

Запишем оценку выборочной дисперсии $Var(\dm)$ на конечном наборе результатов измерений:
\begin{equation}
	Var(\dm)=\frac{1}{n}\sum\limits_{i=1}^{n}(\dm_i-<\dm>)^2=284.73
\end{equation}

Оценим \textit{кажущуюся ошибку $\Delta^2$ }
\begin{equation}
	\Delta^2=\frac{1}{n}\sum\limits_{i=1}^{n}(a_i-\hat{b}-\hat{k}\Delta{}m_i)=0.0152
\end{equation}

Запишем точечную оценку $\hat{\sigma}$ среднеквадратичного отклонения $\sigma$ (стандартная ошибка) при помощи $\Delta^2$ --- кажущейся ошибки:
\begin{gather}
	% \label{ex:}	
	\hat{\sigma}=\sqrt{\frac{\Delta^2}{n-2}}=\sqrt{\frac{0.015246}{3}}=0.07128
\end{gather}

Оценим ковариацию величин $\dm$ и $a$ (среднее арифметическое произведений отклонений значений этих величин от своих выборочных средних):
\begin{gather}
	% \label{ex:}
	Cov(\dm,a)=\frac{1}{n}\sum\limits_{i=1}^{n}(\dm_i-\dmsr)\cdot{}a_i=319.81
\end{gather}
Тогда можем рассчитать оценки значений коэффициентов:
\begin{gather}
	\hat{k}=\frac{Cov(\dm,a)}{Var(\dm)}=1.123\\
	\hat{b}=\dmsr-\hat{k}<a>=-5.70
\end{gather}
% Полученные значения, где первое вычислили как угловой коэффициент, а второе как сдвиг графика по оси $Y$ в точке $x=0$:

% \begin{gather}
% 	\label{ex:koeff}
% 	\hat{k}=\frac{g}{2M+m_1+m_2+\lambda}=1.123\\\nonumber
% 	\text{и}\\
% 	\hat{b}=\frac{F0}{2M+m_1+m_2+\lambda}=5.70
% \end{gather}

% Произведя нехитрые арифметические действия для измеряемых величин, получили

% \begin{gather}
% 	\label{ex:koeff}
% 	\lambda=99.15\ \text{г}\\\nonumber
% 	\text{и}\\
% 	F0=4979.25\ \text{дин}
% \end{gather}

% Погрешности для параметров регрессионной прямой $a=1.123\Delta{m}-5.70$ реально рассчитать.

Оценим значимость коэффициента регрессии.

Для этого используется t-критерий Стьюдента. Выдвигается гипотеза $H_0$ об отсутствии 
влияния фактора на отклик. Если фактическое значение  $t_f$ t-критерия превышает табличное, то гипотеза отклоняется: влияние фактора на отклик обнаружено. Найдем $t_f$: 

\begin{equation}
	t_f=\frac{|\hat{k}|}{\hat{\sigma}}=\frac{|1.123|}{0.07128}=15.75
\end{equation}

Табличное значение t-критерия Стьюдента на уровне значимости $\alpha=0.05$ (коэффициент доверия $1-\alpha=99.5\%$) и числе степеней свободы $n-2$ составляет $t_\alpha=5.8409$.

$5.8409<15.75 \Longrightarrow t_\alpha<t_f$, гипотеза $H_0$ отклонена, влияние фактора на отклик обнаружено.

% Запишем радиусы доверительных интервалов:
Найдем оценки стандартных отклонений (корень из оценки теоретических дисперсий, умноженный на квантиль распределения Стьюдента) для $k$ и $b$:

\begin{gather}
	% \label{ex:}	
	\epsilon_\alpha^b=\frac{\hat{\sigma}\cdot{}t_\alpha}{\sqrt{n}}\cdot\sqrt{1+\frac{<\Delta{m}>^2}{n\cdot{}Var(\dm)}},
	\hspace{0.5cm}
	\epsilon_\alpha^k=\frac{\hat{\sigma}\cdot{}t_\alpha}{\sqrt{n\cdot{}Var(\dm)}},	
	% \hat{\sigma}=\sqrt{\frac{\Delta^2}{n-2}}
\end{gather}

Где $t_\alpha=5.8409$ --- верхняя двусторонняя квантиль распределения Стьюдента для $n-2=3$ степеней свободы по уровню значимости $\alpha$ при коэффициенте доверия $1-\alpha=99.5\%$.

% r=b0.792
Найдем тесноту связи отклика и фактора линейным коэффициентом корреляции Пирсона $r$, который можно вычислить по следующей формуле: 

\begin{equation}
	r=\hat{k}\frac{\sum\limits_n(\dm_i-\dmsr)^2}{\sum\limits_n(a_i-<a>)^2}=0.889
\end{equation}

Качественная оценка тесноты связи выявлена по шкале Чеддока - коэффициент Пирсона лежит в интервале $[0.7\ldots0.9]$, теснота связи - высокая.
 
Тогда можем записать относительную погрешность коэффициентов $k, b$ при коэффициенте доверия $1-\alpha=99.5\%$:

\begin{gather*}
\epsilon_{0.01}^k=\frac{0.07128\cdot{}5.8409}{\sqrt{5\cdot284.73}}=0.0110
\Longrightarrow \varepsilon(k)=\frac{\epsilon_{0.01}^k}{\hat{k}}=\frac{0.0110}{1.123}=0.0097\\
\epsilon_{0.01}^b=\frac{0.07128\cdot{}5.8409}{\sqrt{5}}\cdot\sqrt{1+\frac{{32.088^2}}{5\cdot284.73}}=0.2444
\Longrightarrow \varepsilon(b)=\frac{\epsilon_{0.01}^b}{\hat{b}}=\frac{0.2444}{5.70}=0.0428
\end{gather*}

Но 
\begin{gather}
	% \label{ex:}
	\hat{k}=\frac{g}{2M+m_1+m_2+\lambda}=1.123\\\nonumber
	\text{тогда}\\
	\varepsilon(k)=\frac{\Delta(2M+m_1+m_2+\lambda)}{2M+m_1+m_2+\lambda}=
	\frac{4\Delta(m)+\Delta(\lambda)}{2M+m_1+m_2+\lambda}\Longrightarrow\\
	\Delta\lambda=\varepsilon(k)\cdot(2M+m_1+m_2+\lambda)-4\Delta(m)\\
	\Delta\lambda=0.0097(2\cdot363+48.4+99.15)-4\cdot0.5=6.4734\\
	\label{interval_lambda1}\lambda\in\[92.67\ldots105.62\] \text{(c вероятностью 99.5\%)}
\end{gather}

Тогда
\begin{gather}
	% \label{ex:}
	\varepsilon(b)=\varepsilon(F_0\cdot{}k)=\frac{\Delta{F_0}}{F_0}+\varepsilon(k)\Longrightarrow\\
	\Delta{F_0}=F_0\cdot(\varepsilon(b)-\varepsilon(k))\\
	\Delta{F_0}=4979.25\cdot(0.0428-0.0097)=164.8131
	% 0.0097(2\cdot363+48.4+99.15)-4\cdot0.5=6.4734
\end{gather}

\textbf{Предоставим решение контрольного вопроса №4.}
Из (\ref{eq:a-g}) легко выводится 
$$F_0=(m_2-m_1)g-a(2M+m_1+m_2+\lambda)$$
Тогда оценим интервал $F_0$ для 5 экспериментальных точек:
\begin{gather*}
F_0=4940\ (\text{дин})\\
F_0=4944\ (\text{дин})\\
F_0=5053\ (\text{дин})\\
F_0=4931\ (\text{дин})\\
F_0=5013\ (\text{дин})\\	
\end{gather*}
Получили интервал изменения для используемых в установке грузов и перегрузов $F_0\in\[4940\ldots5053\]$
Отметим, что все эти значения лежат в рассчитанном доверительном интервале $F_0$: 
$F_0\in\[4814.43\ldots5144.06\]$. 

% Постановку размерностей можно объяснить: $\lambda$  суммируется с массой, следовательно, должно иметь размерность массы, а так как эксперимент измерялся в системе СГС, то размерность $\lambda$ будет масса (в граммах).

% Сила же в СГС измеряется в динах, где $1$ дин = $10^{-5}$ Н.

% Итак, гипотеза о том, что $F_\text{тр}=F_0+F(a)=F_0+\lambda{}a$, подтвердилась: график ускорения от изменения массы перегрузков (рис. \ref{fig:a-m}) действительно представляет собой прямую, что доказывает эту гипотезу.

Для использования этих констант необходимо доказать, что (\ref{eq:a-dm}) верно. Обратим внимание, что мы можем скомпенсировать $F_0$, положив на  правый груз перегрузок $mg=F_0$. Если (\ref{eq:a-dm}) верно,  то зависимость $h(\Delta\,t)$ должна быть линейной (ускорение будет равно нулю).

Таблица экспериментальных результатов доступна в протоколе лабораторной работы. Построим график зависимости (рис. \ref{fig:h-t}, см. приложение 1, стр. \pageref{fig:h-t}).

% Абсолютная погрешность времени составляет для $0.2$ секунды, а длины - 0.5 см, откуда следует, что изобразить прямоугольники погрешностей на графике (рис. \ref{fig:h-t}, стр.\pageref{fig:h-t}) на данном масштабе нельзя (аналогично (рис. \ref{fig1})). Размеры прямоугольника погрешностей много меньше размеров графика.

График логически соотвествует ожиданиям: для большей массы перегрузков получали большую постоянную скорость после снятия перегрузков типа Б. $h(\Delta\,t)$ меняется линейно, то есть движение равномерное: следственно, $F_0$ действительно является сухим трением, которое можно скомпенсировать перегрузком. 

\subsection{Физический смысл константы $\lambda$} 

Остается определить физический смысл константы $\lambda$.

Запишем основное уравнение динамики вращательного движения (ОУДВД):

\begin{gather}
	\label{ex:oudvd}
	\sum\limits_i M_i=I\cdot\beta,
\end{gather}

где $\beta$ - угловое ускорение, равное $\frac{a}{R}$, $R$ --- радиус блока, I --- момент инерции (мера инертности) блока.

Запишем ОУДВД (\ref{ex:oudvd}) для машины Атвуда.

\begin{gather}
	\nonumber F_0\cdot{}r+\lambda\cdot{}a\cdot{}R=I_\text{блока}\cdot\frac{a}{R}\\\nonumber
	\text{пренебрежём слагаемым } F_0\cdot{}r, \text{ тогда}\\
	\label{ex:I}\lambda=\frac{I_\text{блока}}{R^2}
\end{gather}

Итак, константа $\lambda$ выражает меру инертности блока. 

% Остается сравнить $\lambda$ с графика и $\lambda$, рассчитанного через момент инерции. 

Будем рассматривать блок как три диска, сумма их моментов инерции будет равна моменту инерции блока. Введем плотность блока (блок сделан из дюраля).

Тогда:

\begin{gather}
	\nonumber
	\lambda=\frac{I_\text{блока}}{R^2}=\\\nonumber
	=\frac{1}{R_1^2}\cdot(2\cdot0.5M_1R_1^2+0.5M_2R_1^2)=\\\nonumber
	=\frac{1}{R_1^2}\cdot((\pi{}{R_1}^2{d_1}\rho)^2+0.5(\pi{}{R_2}^2{d_2}\rho)^2)=\\\nonumber
	=\frac{1}{4.5^2}\cdot((\pi\cdot{4.5}^2\cdot{0.5}\cdot2.69)^2+0.5(\pi\cdot{4.35}^2\cdot{0.02}\cdot2.69)^2)=\\
	\label{ex:I}=100.74 \text{ г}
\end{gather}

Найдем абсолютную погрешность $\lambda$, учитывая, что приборная погрешность штангенциркуля $\Dh=0.01$см.

Запишем функцию без констант, так как сначала будем искать относительную погрешность:
\begin{gather}
	\lambda=\frac{R_1^4 d_1^2}{R_1^2}+\frac{R_2^4 d_2^2}{R_1^2}\\
	\Dl=\Delta(\frac{R_1^4 d_1^2}{R_1^2})+\Delta(\frac{R_2^4 d_2^2}{R_1^2})\\
	\Delta(\frac{R_1^4 d_1^2}{R_1^2})=\frac{R_1^4 d_1^2}{R_1^2}\cdot\left(\frac{2\Dh}{R_1}+\frac{2\Dh}{d_1}\right)=2\Dh(d_1^2R_1+d_1R_1^2)\\
	\Delta(\frac{R_2^4 d_2^2}{R_1^2})=\frac{R_2^4 d_2^2}{R_1^2}\cdot\left(\frac{4\Dh}{R_2}+\frac{2\Dh}{d_2}+\frac{2\Dh}{R_1}\right)=2\Dh\left(\frac{2R_2^3d_2^2}{R_1^2}+\frac{R_2^4d_2}{R_1^2}+\frac{R_2^4d_2^2}{R_1^3}\right)\\
	\el=\frac{4\Dh}{R_1}+\frac{2\Dh}{d_1}+\frac{2\Dh}{d_2}+\frac{4\Dh}{R_2}\\
	\Dl=2\Dh\left(\frac{2R_2^3d_2^2}{R_1^2}+\frac{R_2^4d_2}{R_1^2}+\frac{R_2^4d_2^2}{R_1^3}+d_1^2R_1+d_1R_1^2\right)=0.8834
\end{gather}

% \begin{gather}
% 	\el=\frac{}{}
% \end{gather}

% Итак, предположение, что $\lambda$ отражает инерционные свойства блока, подтвердилось с высокой точностью. Расхождение значений составляет всего 1.1\%.

Следовательно,

\begin{equation}
	\label{interval_lambda2}\lambda\in\[99.85\ldots101.62\]
\end{equation}

Отметим, что этот доверительный интервал (\ref{interval_lambda2}) полностью лежит в доверительном интервале (\ref{interval_lambda1}) $\lambda$, найденного другим способом. Следовательно, в пределах погрешностей измерений при коэффициенте доверия, равном 99.5\%, можно утверждать следующее: $\lambda$, полученное исследованием функции, совпадает с $\lambda$, полученным при помощи косвенного измерения момента инерции.

\subsection{Вывод}

В  результате проделанной работы были выполнены следующие пункты.

Опровергнута гипотеза о зависимости ускорения груза от мгновенной скорости.

Снята линейная зависимость $S(t^2)$ для трех значений $m_2-m_1$, откуда сделан вывод о равноускоренном движении грузов в машине Атвуда.

Снята зависимость ускорения грузов от разности масс перегрузков,
для которой расчитана соответствующая погрешность ускорения (\ref{eq:a-err})

Оценены коэффициенты $\lambda$  и $F_0$ методом наименьших квадратов Гаусса.
Найдена их абсолютная погрешность через t-критерий Стьюдента (при коэффициенте доверия 99.5\%).
$$\lambda=99.15\ \text{г} \pm 6.47\ \text{г}$$
$$F_0=4979.25\ \text{дин} \pm 164.81\ \text{дин}$$

Изучено уравнение динамики вращательного движения (ОУДВД) и физический смысл момента инерции, а также методы его вычисления.

Рассчитано значение коэффициента $\lambda$ через ОУДВД и его абсолютная погрешность:
$$\lambda=100.74\ \text{г} \pm 0.88\ \text{г}$$

Определена правильность определения $F_0$: на правый груз был добавлен постоянный перегрузок, равный $\frac{F_0}{g}=5.07 \text{г}$, после разгона и снятия разгонных перегрузов грузы двигались с постоянной скоростью, что доказывается показано на графике (рис. \ref{fig:h-t}).

Сравнение коэффициента $\lambda$, полученного разными способами, показывает: в пределах погрешностей измерений при коэффициенте доверия, равном 99.5\%, можно утверждать следующее: $\lambda$, полученное исследованием зависимости $a(\dm)$, совпадает с $\lambda$, полученным при помощи косвенного измерения момента инерции.

Для эксперементальных данных, укладывающихся на график уравнения (\ref{eq:a-dm}) оценен линейный коэффициент корреляции Пирсона $$r=0.889,$$ что по шкале Чеддока означает высокую тесноту связи: отсюда можно сказать, что построенная математическая модель подходит для описания движения грузов.

В пределах погрешностей измерений были построены линеаризованные графики зависимостей.

В работе рассчитаны погрешности для всех косвенных измерений, размеры прямоугольников ошибок. 

Все точки на графиках укладываются на линеаризованные функции в пределах размеров их прямоугольников ошибок.

Подтверждена справедливость закона равномерного прямолинейного движения тела при равнодействующей сил, равной нулю (1 закон Ньютона) с помощью машины Атвуда (рис. \ref{fig:h-t})

\newpage
\section*{Приложение 1. Графики зависимостей} % (fold)
\label{sec:figures}

\begin{figure}[h]
\begin{center}
\includegraphics*[width=1\textwidth]{img/ex1.eps}
\caption{\label{fig:htt}График зависимости $h(t^2)$}
\end{center}
\end{figure}

\begin{figure}[h]
\begin{minipage}[h]{1\linewidth}
	% \begin{figure}[h]
	\begin{center}
	\includegraphics*[width=1\textwidth]{img/ex_22.png}
	\caption{\label{fig:a-m}График зависимости $a(\Delta\,m)$}
	\end{center}
	% \end{figure}
\end{minipage}
\vfill
\begin{minipage}[h]{1\linewidth}
	% \begin{figure}[h]
	\begin{center}
	\includegraphics*[width=1\textwidth]{img/ex_2-5.png}
	\caption{\label{fig:a-m-2}Прямоугольники погрешностей с графика $a(\Delta\,m)$}
	\end{center}
	% \end{figure}
\end{minipage}

\end{figure} 

\begin{figure}[h]
	\begin{center}
		\includegraphics*[width=1\textwidth]{img/ex4-d.eps}
		\caption{\label{fig:h-t}График зависимости $h(\Delta\,t)$}
	\end{center}
\end{figure}

% section figures (end)

\end{document}
