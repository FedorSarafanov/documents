\documentclass[a4paper,14pt]{extreport}
 
\usepackage{extsizes}
\usepackage{cmap}
\usepackage[T2A]{fontenc}
\usepackage[utf8]{inputenc}
\usepackage[russian]{babel}
\usepackage{cyrtimes}

%%%%%%%%%%%%%%%%%%%%%%%%%%%%%%%%%%%%%%%%%%%%%%%%%%%%%%%%%%%%%%%%%%%%%%%%%%%%%%%%%%  
\usepackage{graphicx} % для вставки картинок
\graphicspath{{img/}}
\usepackage{amssymb,amsfonts,amsmath,amsthm} % математические дополнения от АМС
\usepackage{indentfirst} % отделять первую строку раздела абзацным отступом тоже
\usepackage[usenames,dvipsnames]{color} % названия цветов
\usepackage{makecell}
\usepackage{multirow} % улучшенное форматирование таблиц
\usepackage{ulem} % подчеркивания
\linespread{1.3} % полуторный интервал
\renewcommand{\rmdefault}{ftm} % Times New Roman
\frenchspacing
\usepackage{geometry}
\geometry{left=3cm,right=1cm,top=2cm,bottom=2cm,bindingoffset=0cm}
\usepackage{titlesec}
% \definecolor{black}{rgb}{0,0,0}
% \usepackage[colorlinks, unicode, pagecolor=black]{hyperref}
\usepackage[unicode]{hyperref}


\begin{document}

% -*- root: project.tex -*-
\begin{titlepage}
\newpage

\begin{center}

Муниципальное автономное образовательное учреждение \\
лицей № 180 \\
г. Нижнего Новгорода \\

% \hrulefill
\end{center}
 
% \flushright{КАФЕДРА № ХХХ}

\vspace{14em}

\begin{center}
\large{Реферат}
\end{center}

% \vspace{.5em}
 
\begin{center}
Велосипедный спорт
\end{center}

\vspace{4.5em}
 
\begin{flushright}
Выполнил: Сарафанов Фёдор, \\
ученик 10 <<А>> класса \\
Научный руководитель: \\
Яшков Александр Николаевич, \\
учитель физической культуры \\ 
первой квалификационной категории
\end{flushright}
 
\vspace{\fill}

\begin{center}
Нижний Новгород \\
2015
\end{center}

\end{titlepage}
\addtocounter{page}{1}

\tableofcontents
%\large

\chapter*{Введение}
\addcontentsline{toc}{chapter}{Введение}
Кислород  находится  в  окружающем  нас  воздухе. Он  может проникнуть  сквозь  кожу,  но  лишь  в  небольших количествах,  совершенно  недостаточных  для  поддержания  жизни. Поступление  в  организм  кислорода  и  удаление  углекислого  газа  обеспечивает  дыхательная  система.  Транспорт  газов  и  других  необходимых  организму    веществ  осуществляется  с  помощью  кровеносной  системы.  Функция  дыхательной системы  сводится  лишь  к  тому,  чтобы  снабжать  кровь  достаточным  количеством  кислорода и  удалять  из  нее  углекислый  газ. 

Химическое восстановление молекулярного кислорода с образованием воды служит для млекопитающих основным источником  энергии. Без нее жизнь не может продолжаться дольше нескольких секунд. 

Восстановлению кислорода сопутствует образование $CO_2$. Кислород входящий в $CO_2$ не происходит непосредственно из молекулярного кислорода. Использование $O_2$  и образование $CO_2$  связаны  между собой промежуточными метаболическими реакциями; теоретически каждая из них длятся некоторое время. 

Обмен  $O_2$  и $CO_2$  между организмом и средой называется дыханием. У высших животных процесс дыхания осуществляется благодаря ряду последовательных процессов:

\begin{itemize}
    \item Обмен газов между средой и легкими, что обычно обозначают как <<легочную вентиляцию>>
    \item Обмен газов между альвеолами легких и кровью  (легочное дыхание)
    \item Обмен газов между кровью и тканями.
    \item Наконец, газы переходят внутри ткани к местам потребления (для $O_2$) и от мест образования (для $CO_2$) (клеточное дыхание).  Выпадение любого из этих четырех процессов приводят к нарушениям дыхания и создает опасность для жизни человека
\end{itemize}

\chapter{Анатомия дыхательной системы человека}

Дыхательная  система   человека  состоит  из  тканей  и  органов,  обеспечивающих  легочную вентиляцию  и  легочное  дыхание.  К воздухоносным путям относятся: нос, полость носа, носоглотка, гортань, трахея, бронхи и бронхиолы. Легкие состоят из бронхиол и альвеолярных мешочков, а также из артерий, капилляров и вен легочного круга кровообращения. К элементам костно-мышечной системы, связанным с дыханием, относятся ребра, межреберные мышцы, диафрагма и вспомогательные дыхательные мышцы.

\section{Воздухоносные пути}
Нос и полость носа служат проводящими каналами для воздуха, в которых он нагревается, увлажняется и фильтруется.  В полости носа заключены также обонятельные  рецепторы.

Наружная  часть  носа  образована  треугольным  костно-хрящевым остовом, который покрыт кожей; два овальных  отверстия на нижней поверхности-ноздри открываются каждое в клиновидную полость носа. Эти полости разделены перегородкой. 
Три легких губчатых завитка (раковины) выдаются из боковых стенок ноздрей, частично разделяя полости на четыре  незамкнутых прохода (носовые ходы).

Полость носа выстлана  богато васкуляризованной слизистой оболочкой. Многочисленные жесткие волоски,  а также снабженные  ресничками эпителиальные и бокаловидные клетки служат для  очистки вдыхаемого воздуха от твердых частиц. В верхней части полости  лежат  обонятельные  клетки.  

Гортань  лежит  между  трахеей  и  корнем  языка.  Полость  гортани  разделена  двумя  складками  слизистой  оболочки,  не  полностью  сходящимися  по  средней  линии.  Пространство  между  этими складками - голосовая  щель  защищено  пластинкой  волокнистого  хряща  -  надгортанником. По краям голосовой щели в слизистой оболочке лежат фиброзные эластичные связки, которые называются нижними, или истинными, голосовыми складками (связками). Над ними находятся ложные голосовые складки,  которые защищают истинные  голосовые складки и сохраняют их влажными; они помогают также задерживать дыхание, а при глотании препятствуют попаданию пищи  в гортань. 

Специализированные мышцы натягивают и расслабляют истинные и ложные голосовые складки. Эти мышцы играют важную роль при фонации, а также препятствуют попаданию каких-либо частиц в дыхательные пути.

Трахея начинается у нижнего конца гортани и спускается в грудную полость, где делится на правый и левый бронхи; стенка ее образована соединительной тканью и хрящом. У большинства млекопитающих хрящи образуют неполные кольца. Части, примыкающие  к пищеводу,  замещены  фиброзной связкой. Правый бронх обычно короче и шире левого. 

Войдя в легкие, главные бронхи постепенно делятся на все более мелкие трубки (бронхиолы), самые мелкие из которых-конечные бронхиолы  являются  последним  элементом  воздухоносных путей. От гортани до конечных бронхиол трубки выстланы мерцательным эпителием.


\section{Легкие}
В целом легкие имеют вид губчатых, пористых конусовидных образований, лежащих о обеих половинах грудной полости.
Наименьший структурный элемент легкого - долька состоит из конечной бронхиолы, ведущей в легочную бронхиолу и альвеолярный мешок. Стенки легочной бронхиолы и альвеолярного мешка образуют углубления-альвеолы.  Такая  структура  легких  увеличивает  их  дыхательную  поверхность, которая  в  50-100  раз  превышает  поверхность  тела.    Относительная  величина  поверхности,  через  которую  в  легких  происходит  газообмен,  больше  у  животных  с  высокой  активностью  и  подвижностью.

Стенки альвеол состоят из одного слоя эпителиальных клеток и окружены легочными  капиллярами.  Внутренняя  поверхность альвеолы покрыта поверхностно активным  веществом  -- сурфактантом. 

Как полагают, сурфактант является продуктом секреции гранулярных клеток.  Отдельная альвеола, тесно соприкасающаяся с  соседними  структурами,  имеет форму неправильного многогранника и приблизительные размеры до 250 мкм. Принято считать, что общая поверхность альвеол,  через которую осуществляется газообмен, экспоненциально  зависит от веса тела. С возрастом отмечается уменьшение площади поверхности альвеол.

\subsection{Плевра}
Каждое легкое окружено мешком  -плеврой. Наружный  (париетальный) листок плевры примыкает к внутренней поверхности грудной стенки и диафрагме, внутренний (висцеральный) покрывает легкое. Щель между листками называется плевральной полостью. При движении грудной клетки внутренний листок обычно легко скользит по наружному. Давление в плевральной полости всегда меньше атмосферного (отрицательное). В условиях покоя внутриплевральное давление у человека в среднем на 4,5 торр ниже атмосферного (-4,5 торр). Межплевральное пространство между легкими называется средостением; в нем находятся трахея,  зобная железа (тимус) и сердце с большими сосудами, лимфатические узлы и пищевод.

\subsection{Кровеносные сосуды легких}
Легочная артерия несет кровь от правого желудочка сердца, она делится на правую и левую ветви, которые направляются к легким. Эти артерии ветвятся, следуя за бронхами, снабжают крупные структуры легкого и образуют капилляры, оплетающие стенки альвеол.

Воздух в альвеоле отделен от крови в капилляре:

\begin{itemize}
    \item стенкой  альвеолы
    \item стенкой капилляра и в некоторых случаях промежуточным слоем между ними
\end{itemize}    

Из капилляров кровь поступает в мелкие вены, которые в конце концов соединяются и образуют легочные вены, доставляющие кровь в левое предсердие. 

Бронхиальные артерии большого круга тоже приносят кровь к легким, а именно снабжают бронхи и бронхиолы, лимфатические  узлы, стенки кровеносных сосудов и плевру. Большая часть  этой крови оттекает в бронхиальные вены, а оттуда-в непарную (справа) и в полунепарную (слева). Очень небольшое количество  артериальной бронхиальной крови поступает в легочные вены.              


\subsection{Дыхательные мышцы}
Дыхательные мышцы - это те мышцы, сокращения которых изменяют объем грудной клетки. Мышцы, направляющиеся от головы, шеи, рук и некоторых верхних грудных и нижних шейных позвонков, а также наружные межреберные мышцы, соединяющие ребро с ребром, приподнимают ребра и увеличивают  объем  грудной  клетки.  Диафрагма-мышечно-сухожильная  пластина, прикрепленная к позвонкам, ребрам и грудине,отделяет грудную полость от брюшной. Это главная мышца, участвующая в нормальном вдохе. При усиленном вдохе сокращаются дополнительные группы мышц. При усиленном выдохе действуют  мышцы,  прикрепленные  между ребрами (внутренние межреберные мышцы),  к ребрам и нижним грудным и верхним поясничным позвонкам, а также мышцы брюшной полости; они опускают ребра и прижимают брюшные органы к расслабившейся диафрагме, уменьшая таким образом емкость грудной клетки.


\subsection{Легочная вентиляция}
Пока внутриплевральное давление остается ниже атмосферного, размеры легких точно следуют за размерами грудной полости. Движения легких совершаются в результате сокращения  дыхательных мышц в сочетании с движением частей грудной  стенки и диафрагмы.


\subsection{Дыхательные движения}
Расслабление всех связанных с дыханием мышц придает грудной клетке положение пассивного выдоха. Соответствующая мышечная  активность может перевести это положение во вдох или же усилить выдох.
Вдох создается расширением грудной полости и всегда является активным процессом. Благодаря своему сочленению с  позвонками ребра движутся вверх и наружу, увеличивая расстояние от позвоночника до грудины, а также боковые размеры  грудной полости (реберный или грудной тип дыхания). 

Сокращение диафрагмы меняет ее форму из куполообразной в более плоскую, что увеличивает размеры грудной полости в продольном направлении (диафрагмальный или брюшной тип дыхания). Обычно главную роль во вдохе играет диафрагмальное дыхание. Поскольку люди-существа двуногие, при каждом движении  ребер и грудины меняется центр тяжести тела и возникает необходимость приспособить к этому разные мышцы.

При спокойном дыхании у человека обычно достаточно эластических свойств и веса переместившихся тканей, чтобы  вернуть их в положение, предшествующее вдоху.

Таким образом, выдох в покое происходит  пассивно  вследствие  постепенного снижения активности мышц, создающих  условие  для  вдоха. Активный выдох может возникнуть вследствие сокращения внутренних  межреберных мышц в дополнение к другим мышечным группам, которые опускают ребра, уменьшают поперечные размеры грудной полости и расстояние между грудиной и позвоночником. Активный выдох может также произойти вследствие сокращения брюшных мышц, которое прижимает внутренности к расслабленной диафрагме  и  уменьшает  продольный  размер грудной полости. 

Расширение легкого снижает (на время) общее внутрилегочное (альвеолярное) давление. Оно равно атмосферному, когда  воздух не движется, а голосовая щель открыта. Оно ниже атмосферного, пока легкие не наполнятся при вдохе, и выше атмосферного при выдохе. Внутриплевральное давление тоже меняется на протяжении дыхательного движения; но оно всегда ниже атмосферного (т. е. всегда отрицательное).

\subsection{Изменения объема легких}

У человека легкие  занимают  около  6\%  объема  тела   независимо  от  его  веса.  Объем легкого меняется при вдохе не всюду одинаково. Для  этого имеются три главные причины, во-первых, грудная полость  увеличивается неравномерно во всех направлениях, во-вторых, не асе части легкого одинаково растяжимы. В-третьих, предполагается существование гравитационного эффекта, который способствует смещению легкого книзу.

Объем воздуха, вдыхаемый при обычном (неусиленном) вдохе и выдыхаемой при обычном (неусиленном) выдохе, называется  дыхательным воздухом. Объем максимального выдоха после  предшествовавшего максимального вдоха называется жизненной емкостью. Она не равна всему объему воздуха в легком (общему объему легкого), поскольку легкие полностью не спадаются. Объем воздуха, который остается в наспавшихся легких, называется остаточным воздухом. 

 Имеется дополнительный объем,  который можно вдохнуть при максимальном усилии после нормального вдоха. 
А тот воздух, который выдыхается максимальным усилием после нормального выдоха, это резервный объем  выдоха. Функциональная остаточная емкость состоит из резервного объема выдоха и остаточного объема. Это тот находящийся в легких воздух, в котором разбавляется нормальный дыхательный воздух. Вследствие этого состав газа в легких после одного дыхательного движения обычно резко не меняется.

Минутный объем $V$-это воздух, вдыхаемый за одну минуту. Его можно вычислить, умножив средний  дыхательный объем $Vt$ на число дыханий в минуту $f$, или $V=fVt$. 

Часть$ Vt$, например, воздух в трахее и бронхах до конечных бронхиол и в некоторых  альвеолах, не участвует в газообмене, так как  не  приходит  в  соприкосновение  с  активным  легочным кроватоком  -  это  так  называемое  <<мертвое>> пространство $Vd$.  Часть  $Vt$,  которая  участвует  в  газообмене  с легочной  кровью,  называется  альвеолярным  объемом $VA$.

 С физиологической  точки  зрения  альвеолярная   вентиляция  $VA$ - наиболее существенная  часть  наружного дыхания   $VA=f(Vt-Vd)$,  так   как  она  является  тем  объемом  вдыхаемого  за  минуту  воздуха,  который  обменивается  газами  с  кровью  легочных  капилляров. 

\subsection{Легочное  дыхание}

Газ является таким состоянием вещества, при котором оно равномерно распределяется по ограниченному объему. В газовой фазе взаимодействие молекул между собой незначительно.

Когда они сталкиваются со стенками замкнутого пространства, их движение создает определенную силу; эта сила, приложенная к единице площади, называется давлением газа и выражается в миллиметрах ртутного столба, или торрах; давление газа пропорционально числу молекул и их средней скорости. При комнатной температуре давление какого-либо вида молекул; например, $O_2$ или $N_2$, не зависит от присутствия молекул другого газа. 

Общее измеряемое давление газа равно сумме давлений отдельных видов молекул (так называемых парциальных давлений) или $РB=РN_2+Ро_2+Рн_2o+РB$, где $РB$ - барометрическое давление. 

Долю $F$ данного газа $x$ в сухой газовой смеси мощно вычислить по следующему уравнению:
               $$ Fx=Px/PB-PH2O $$

И наоборот, парциальное давление давнего газа $x$ можно вычислить из его доли: $Рx-Fx(РB-Рн_2o)$. Сухой атмосферный воздух содержит 20,94\% $O_2*РO_2=20,94/100*760$ торр  (на  уровне моря) $=159,1$ торр.

Газообмен в легких между альвеолами и кровью происходит путем диффузии. Диффузия возникает в силу постоянного движения молекул газа к обеспечивает перенос молекул из области более высокой их концентрации в область, где их концентрация ниже.

\subsection{Транспорт дыхательных газов}

Около О,3\% $O_2$, содержащегося в артериальной крови большого круга при нормальном $Ро_2$, растворено в плазме. Все остальное количество находится в непрочном химическом соединении с гемоглобином $Hb$ эритроцитов. Гемоглобин представляет собой белок с присоединенной к нему железосодержащей группой. $Fе $+ каждой молекулы гемоглобина соединяется непрочно и обратимо с одной молекулой $O_2$. Полностью насыщенный кислородом гемоглобин содержит 1,39 мл. О2 на 1 г $Hb$ (в некоторых источниках указывается 1,34 мл), если $Fе$ + окислен до Fе +, то такое соединение утрачивает способность переносить $O_2$.
Полностью насыщенный кислородом гемоглобин ($HbO_2$) обладает более сильными кислотными свойствами, чем восстановленный гемоглобин (Hb). В результате в растворе, имеющем рН 7,25, освобождение 1мМ О2 из HbО2 делает возможным усвоение О,7 мМ Н+ без изменения рН; таким образом, выделение $O_2$ оказывает буферное действие.

\subsection{Насыщение  тканей  кислородом}

Транспорт  $O_2$  из  крови  в  те  участки  ткани,  где  он  используется,  происходит  путем  простой  диффузии. 
Поскольку  кислород  используется  главным  образом  в  митохондриях,  расстояния,  на  которые  происходит  диффузия в тканях,  представляются  большими  по  сравнению  с  обменом  в  легких.  В  мышечной  ткани  присутствие  миоглобина,  как  полагают,  облегчает  диффузию  $O_2$.  Для  вычисления  тканевого  $Po_2$  созданы  теоретически  модели,  которые  предусматривают  факторы,  влияющие на поступление  и  потребление  $O_2$,  а  именно  расстояние  между  капиллярами,  кроваток  в  капиллярах  и  тканевой  метаболизм.  

Самое  низкое $Po_2$  установлено  в  венозном  конце  и  на  полпути  между  капиллярами,  если  принять, что кроваток  в  капиллярах  одинаковый  и  что  они  параллельны.

\chapter{Гигиена дыхания}

Физиологии  наиболее  важны газы - $O_2$, $CO_2$, $N_2$.   С  точки  зрения медицины  при недостаточном  снабжении  тканей кислородом  возникает  гипоксия.  Краткое изложение разных причин гипоксии может служить и сокращенным обзором всех дыхательных процессов. Ниже в каждом пункте указаны нарушения одного или более процессов.

Систематизация их позволяет рассматривать все эти явления одновременно.

\begin{enumerate}
    \item { Недостаточный транспорт $O_2$ кровью (аноксемическая гипоксия) (содержание $O_2$ в артериальной крови большого круга понижено)
        \begin{itemize}
        \item {Сниженное $PO_2$
        \begin{enumerate}
            \item недостаток $O_2$ во вдыхаемом воздухе
            \item снижение легочной вентиляции
            \item снижение газообмена между альвеолами и кровью
            \item смешивание крови большого и малого круга
        \end{enumerate}
        }
        \item {Нормальное $РO_2$
        \begin{enumerate}
            \item снижение содержания гемоглобина (анемия)
            \item нарушение способности гемоглобина присоединять $O_2$
        \end{enumerate}
        }
        \end{itemize}
    }
    \item { Недостаточный транспорт крови (гипокинетическая гипоксия)
        \begin{itemize}
        \item {Недостаточное кровоснабжение
        \begin{enumerate}
            \item во всей сердечно-сосудистой системе (сердечная недостаточность) 
            \item местное (закупорка отдельных артерий)
        \end{enumerate}
        }
        \item { Нарушение оттока крови
        \begin{enumerate}
            \item закупорка определенных вен
        \end{enumerate}
        }
        \item { Недостаточное снабжение кровью при возросшей потребности.
    Неспособность ткани использовать  поступающий $O_2$ (гистотоксическая  гипоксия).
        }        
        \end{itemize}    
    }
\end{enumerate}


\chapter{Введение в легочные заболевания.}

Повсеместно, особенно в индустриально развитых странах, наблюдается значительный рост заболеваний дыхательной  системы, которые вышли уже на 3-4-е место среди причин смертности населения. Что же касается, например, рака легких, то это патология по ее распространенности опережает у мужчин все остальные злокачественные новообразования. Такой подъем заболеваемости связан в первую очередь с постоянно увеличивающийся загрязненностью окружающего воздуха, курением, растущей аллергизацией населения (прежде всего за счет продукции бытовой химии). Все это в настоящее время обуславливает актуальность своевременной диагностики, эффективного лечения и профилактики болезней органов дыхания. Решением этой задачи занимается пульмонология (от лат. Pulmois - легкое, греч. - logos - учение), являющаяся одним из разделов внутренней медицины.

В своей повседневной практике врачу приходится сталкиваться с различными заболеваниями дыхательной системы. В амбулаторно-поликлинических условиях, особенно в весенне-осенний период, часто встречается такие заболевания, как острый ларингит, острый трахеит, острый и хронический бронхит. В отделениях стационара терапевтического профиля нередко находятся на лечении больные с острой и хронической пневмонией, бронхиальной астмой, сухим и экссудативным плевритом, эмфиземой легких и легочно-сердечной недостаточностью. В хирургические отделения поступают для обследования и лечения больные с бронхоэктатической болезнью, абсцессами и опухолями легких.

Современный арсенал диагностических и лечебных средств, применяемых при обследовании и лечении больных с заболеваниями органов дыхания, является весьма обширным. Сюда относятся различные лабораторные методы исследования (биохимические, иммунологические, бактериологические и др.), функциональные способы диагностики - спирография и спирометрия (определение и графическая регистрация тех или иных параметров, характеризующих функцию внешнего дыхания), вневмотахография и пневмотахометрия (исследование максимальной объемной скорости форсированного вдохы и выдоха), исследование содержания (парциального давления) кислорода и углекислого газа в крови и др.

Весьма информативными являются различные рентгенологические методы исследования дыхательной системы: рентгеноскопия и рентгенография органов грудной клетки, флюорография (рентгенологические исследование с помощью специального аппарата, позволяющего делать снимки размером 70X70 мм, применяющееся при массовых профилактических обследованиях населения),томография (метод прослойного рентгенологического исследования легких, точнее оценивающий характер опухолевидных образований), бронгография, дающая возможность с помощью введения в бронхи через катетер контрастных веществ получить четкое изображение бронхиального дерева.

Важное место в диагностике заболеваний органов дыхания занимают эндоскопические методы исследования, представляющий собой визуальный осмотр слизистой оболочки трахеи и бронхов и помощью введения в них специального оптического инструмента - бронхоскопа.  

Бронхоскопия позволяет установить характер поражения слизистой оболочки бронхов (например, при бронхитах и бронхоэктатической болезни), выявить опухоль бронха и взять с помощью щипцов кусочек ее ткани (провести биопсию) с последующим морфологическим исследованием, получить промывание воды бронхов для бактериологического или цитологического исследования. Во многих случаях бронхоскопию проводят и с лечебной целью. Например, при бронхоэктатической болезни, тяжелым течении бронхиальной астмы можно осуществить санацию бронхиального дерева с последующим отсасыванием вязкой или гнойной мокроты и введением лекарственных средств.    

Уход за больными с заболеваниями органов дыхания обычно включает в себе и ряд общих мероприятий, проводимых при многих заболеваниях других органов и систем организма. 

Так, при крупозной пневмонии необходимо строго придерживаться всех правил и требований ухода за лихорадящими больными (регулярное измерение температуры тела и ведение температурного листа, наблюдение за состоянием сердечно сосудистой и центральной нервной систем, уход за полостью рта, подача судна и мочеприемника, своевременная смена нательного белья и т.д.) При длительном пребывании больного и в постели уделяют особое внимание тщательному уходу за кожными покровами и профилактике пролежней. Вместе с тем уход за больными с заболеваниями органов дыхания предполагает и выполнение целого ряда дополнительных мероприятий, связанных с наличием кашля, кровохарканье, одышки и других симптомов.       

\section{Кашель}
Кашель  представляет собой сложнорефлекторный акт, в котором участвует ряд механизмов (повышение внутригрудного давления за счет напряжения дыхательной мускулатуры, изменения просвета голосовой щели т.д.) и который при заболеваниях органов дыхания обусловлен обычно раздражением рецепторов дыхательных путей и плевры. Кашель встречается при различных заболеваниях дыхательной системы - ларингитах, трахеитах, острых и хронических бронхитах, пневмониях и др. Он может быть связан также с застоем крови в малом кругу кровообращения (при пороках сердца), а иногда имеет центральное происхождение.

Кашель бывает сухим или влажным и выполняет часто защитную роль, способствуя удалению ссодержимого из бронхов (например, мокроты). Однако сухой, особенной мучительный кашель, утомляет больных и требует применения отхаркивающих (препараты термопсиса, и пекакуаны) и противокашлевых средств (либексин, глауцин и др.). В таких случаях больным целесообразно рекомендовать теплое щелочное тепло (горячее молоко с боржомом или с добавлением ? чайной ложки соды), банки, горчичники).

Нередко кашель сопровождается выделением мокроты: слизистой, бесцветной, вязкой (например, при бронхиальной астме), слизисто-гнойной (при бронхопневмонии), гнойной (при прорыве абсцесса легкого в просвет бронха).

Очень важно добиться свободного отхождения мокроты, поскольку ее задержка (например, при бронхоэктатичской болезни, абсцессе легкого) усиливает интоксикацию организма. Поэтому больному помогают найти положение (так называемое дренажное, на том или ином боку, на спине), при котором мокрота отходит наиболее полно, т.е. осуществляется эффективный дренаж бронхиального дерева. Указанное положение больной должен принимать раз в день в течении 20-30 минут.

\section{Кровохаркание и легочное кровотечение}
Кровохарканье представляет собой выделение мокроты с примесью крови, примешанной равномерно(например, <<ржавая>> мокрота при крупозной пневмонии, мокрота в виде <<малинового желе>> при раке легкого) или расположенной отдельными прожилками).
Выделения через дыхательных пути значительного количества крови (с кашлевыми толчками, реже - непрерывной струей) носит название легочного кровотечения.

Кровохарканье и легочное кровотечение встречается чаще всего при злокачественных опухолях, гангрене, инфаркте легкого, туберкулезе, бронхоэктатической болезни, травмах и ранениях легкого, а также при митральных пороках сердца.
При наличии легочного кровотечения его иногда приходится дифференцировать с желудочно-кишечным кровотечением, проявляющимися рвотой с примесью крови. 

В таких случаях необходимо помнить, что легочное кровотечение характеризуется выделением пенистой, алой крови, имеющей щелочную реакцию и свертывающиеся, тогда как при желудочно-кишеном кровотечении (правда, не всегда) чаще выделяются сгустки темной крови, по типу <<кофейной гущи>> смешанные с кусочками пищи, с кислой реакцией.

Кровохарканье и особенно легочное кровотечение являются весьма серьезными симптомами, требующими срочного установления их причины - проведения рентгенологического исследования органов грудной клетки, с томографией, бронхоскопией, бронхографией, иногда - ангиографии.

Кровохарканье и легочное кровотечение, как правило не сопровождаются явлениями шока или коллапса. Угроза для жизни в таких случаях обычно бывает связанна с нарушением вентиляционной функции легких, в результате попадания крови в дыхательные пути. Больным назначают полный покой. Им следует придать полусидячее положение с наклоном в сторону пораженного легкого во избежание  попадание крови в здоровое легкое. На эту же половину грудной клетки кладут пузырь со льдом. При интенсивном кашле, способствующим усилению кровотечения применяют противокашлевые средства.

Для остановки кровотечения внутримышечно вводят викасол, внутривенно - хлористый кальций, эпсилон аминокапроновую кислоту. Иногда при срочной бронхоскопии удается тампонировать кровоточащий сосуд специальной кровоостанавливающей губкой. 

В ряде случаев встает вопрос о срочном хирургическом вмешательстве.

\section{Отдышка}

Одним из наиболее частых заболеваний дыхательной системы является одышка, характеризующаяся изменением частоты, глубины и ритма дыхания. Одышка может сопровождаться как резким учащением дыхания, так и его урежением, вплоть до его остановки. В зависимости от того, какая фаза дыхания оказывает затрудненной, различают инспираторную одышку (проявляется затруднением вдоха, например, при сужение трахеи и крупных бронхов), экспираторную одышку (характеризуются затруднением выдоха, в частности, при спазме мелких бронхов и скопление в их просвете вязкого секрета) и смешанную.

Одышка встречается при многих острых и хронических заболеваниях дыхательной системы. Причина ее возникновения в большинстве случаев возникает с изменением газового состава крови - повышением содержания углекислого газа и снижением содержания кислорода, сопровождающимся сдвигом pH крови в кислую сторону, последующим раздражением центральных и периферических хеморецепторов, возбуждение дыхательного центра и изменения частоты и глубины дыхания.

Одышка является ведущим проявлением дыхательной недостаточности - состояние, при котором система внешнего дыхания человека не может  обеспечить нормальный газовый состав крови или когда этот состав поддерживается лишь благодаря чрезмерному напряжения всей системы внешнего дыхания. Дыхательная недостаточность может возникать остро (например, при закрытие дыхательных путей инородным телом) или протекать хронически, постепенно нарастая в течение длительного времени (например, при эмфиземе легких).

Внезапно возникающий приступ сильной одышки носит название удушья (астмы). Удушье, которое является следствием острого нарушения бронхиальной проходимости - спазма бронхов, отека их слизистой оболочки, накопления в просвете вязкой мокроты, называется приступом бронхиальной астмы. В тех случаях, когда обращение вследствие слабости левого желудочка принято говорить о сердечной астме, иногда переходящей в отек легких.

Уход за больными, страдающими одышкой, предусматривает постоянный контроль за частотой, ритмом и глубиной дыхания. Определения частоты дыхания (по движению грудной клетки или брюшной стенки) проводят незаметно для больного (в этот момент положением руки можно имитировать определенные частоты пульса). У здорового человека частота дыхания колеблется от 16 до 20 в 1 минуту, уменьшаясь во время сна и увеличиваясь при физической нагрузке. При различных заболеваниях бронхов и легких частота дыхания может достигать 30-40 и более в 1 минуту. Полученные результаты подсчета частоты дыхания ежедневно вносят в температурный лист. Соответствующие точки соединяют синим карандашом, образую графическую кривую частоты дыхания.При появление одышки больному придают возвышенное (полусидячее) положение освобождая его от стесняющей одежды, обеспечивают приток свежего воздуха за счет регулярного проветривания. При выраженной степени дыхательной недостаточности проводят оксигенотерапию.

Под оксигенотерапией понимают применение кислорода в лечебных целях. При заболеваниях органов дыхания кислородную терапия применяют в случаю острой и хронической дыхательной недостаточности сопровождающейся цианозом (синюшность кожных покровов), учащением сердечных сокращений (тахикардия), снижением парциального давления кислорода в тканях, мене 70 мм рт.ст. 

Выдыхание чистого кислорода может оказать токсического действие на организм человека проявляющееся в возникновение сухости во рту, чувство жжения за грудиной, болей в грудной клетке, судорог и т.д., поэтому для лечения используют обычно газовую смесь содержащую до 80\% кислорода (чаще всего 40-60\%). Современный устройства, позволяющие подавать больному не чистый кислород, а обогащенную кислородом смесь. Лишь при отравление окисью углерода (угарным газом) допускается применение карбогено содержащего 95\% кислорода и 5\% углекислого газа. В некоторых случаях при лечение дыхательной недостаточности используют ингаляции гелио-кислородные смеси состоящие из 60-70 гелей и 30-40\% кислорода. 
При отеке легких, которые сопровождается пенистой жидкости из дыхательных путей, применяют смесь, содержащую 50\% кислорода и 50\% этилового спирта, в которой спирт играет роль пеногасителя.

Оксигенотерапия может осуществляться как при естественном дыхание так и при использование аппаратов искусственной вентиляции легких. В домашних условиях с целью оксигенотерапии применяют кислородные подушки. При этом больной вдыхает кислород через трубку или мундштук подушки, который он плотно обхватывает губами.

 С целью уменьшения потери кислорода в момент выдоха, его подача временно прекращается с помощью пережатия трубки пальцами или поворотом специального крана

В больничных учреждениях оксигенотерапию проводят с использованием баллонов со сжатым кислородом или системы централизованной подачи кислорода в палаты. Наиболее распространенным  способом кислородотерапии является его ингаляция через носовые катетеры, которые вводят в носовые ходы на глубину примерно равную расстоянию от крыльев носа до мочки уха, реже используют носовые и ротовые маски, интубационные и трахеостомические трубки, кислородные тенты-палатки.
Ингаляции кислородной смеси проводят непрерывно или сеансами по 30-60 мин. несколько раз в день. При этом необходимо, чтобы подаваемый кислород был обязательно увлажнен. Увлажнение кислорода достигается его пропусканием через сосуд с водой, или применением специальных ингаляторов, образующих в газовой смеси взвесь мелких капель воды.

\chapter{Основы методики лечебной физической культуры при заболеваниях органов дыхания}

В занятиях лечебной физической культурой при заболеваниях органов дыхания применяются общетонизирующие и специальные (в том числе дыхательные) упражнения.

Общетонизирующие упражнения, улучшая функцию всех органов и систем, оказывают активизирующее влияние и на дыхание. Для стимуляции функции дыхательного аппарата используются упражнения умеренной и большой интенсивности. В случаях, когда эта стимуляция не показана, применяются упражнения малой интенсивности. Следует учесть, что выполнение необычных по координации физических упражнений может вызвать нарушение ритмичности дыхания; правильное сочетание ритма движений и дыхания при этом установится лишь после многократных повторений движений. Выполнение упражнений в быстром темпе приводит к увеличению частоты дыхания и легочной вентиляции, сопровождается усиленным вымыванием углекислоты (гипокапнией) и отрицательно влияет на работоспособность.

Специальные упражнения  укрепляют дыхательную мускулатуру, увеличивают подвижность грудной клетки и диафрагмы, способствуют растягиванию плевральных спаек, выведению мокроты, уменьшению застойных явлений в легких, совершенствуют механизм дыхания и. координации дыхания и движений. Подбираются упражнения соответственно требованиям, предъявляемым клиническими данными. Например, для растягивания плевродиафрагмальных спаек в нижних отделах грудной' клетки применяются наклоны туловища в здоровую сторону в сочетании с глубоким вдохом; для растягивания спаек в боковых отделах грудной клетки - наклоны туловища в здоровую сторону в сочетании с глубоким выдохом. Толчкообразный выдох и дренажные исходные положения способствуют выведению из дыхательных путей скопившейся мокроты и гноя. При снижении эластичности легочной ткани для улучшения легочной вентиляции применяются упражнения с удлиненным выдохом и способствующие увеличению подвижности грудной клетки и диафрагмы.

При выполнении специальных упражнений во время вдоха под воздействием дыхательных мышц происходит расширение грудной клетки в передне-заднем, фронтальном и вертикальном направлениях. Поскольку вентиляция осуществляется неравномерно, больше всего воздуха поступает в части легкого, прилегающие к наиболее подвижным участкам грудной клетки и диафрагмы, хуже вентилируются верхушки легких и отделы около корня легкого. При выполнении упражнений в исходном положении лежа на спине ухудшается вентиляция в задних отделах легких, а в исходном положении лежа на боку почти исключаются движения нижних ребер.

Учитывая, что неравномерность вентиляции легких особенно проявляется при заболеваниях органов дыхания, специальные дыхательные упражнения следует применять при необходимости улучшить вентиляцию в различных участках легких. Увеличение вентиляции верхушек легких достигается за счет углубленного дыхания без дополнительных движений руками в исходном положении руки на пояс. Улучшение вентиляции задних отделов легких обеспечивается усилением диафрагмального дыхания. Увеличению поступления воздуха в нижние отделы легких способствуют упражнения в диафрагмальном дыхании, сопровождающиеся подъемом головы, разведением плеч, подъемом рук в стороны или вверх, разгибанием туловища. Дыхательные упражнения, увеличивающие вентиляцию легких, незначительно повышают потребление кислорода.

При лечебном применении дыхательных упражнений необходимо учитывать ряд закономерностей. Обычный выдох осуществляется при расслаблении мышц, производящих вдох, под действием силы тяжести грудной клетки. Замедленный выдох происходит при динамической уступающей работе этих мышц. Выведение воздуха из легких в обоих случаях обеспечивается в основном за счет эластических сил легочной ткани. Форсированный выдох происходит при сокращении мышц, производящих выдох. Усиление выдоха достигается наклоном головы вперед, сведением плеч, опусканием рук, сгибанием туловища, подъемом ног вперед и т. п. При необходимости щадить пораженное легкое дыхательные упражнения проводятся в исходных положениях, ограничивающих подвижность грудной клетки с больной стороны (например, лежа на больном боку). При помощи дыхательных упражнений можно произвольно изменять частоту дыхания. Больше других применяются упражнения в произвольном замедлении частоты дыхания (для лучшего эффекта в этих случаях рекомендуется вести подсчет <<про себя>>), Оно уменьшает скорость движения воздуха и снижает сопротивление его прохождению через дыхательные пути. Учащение дыхания увеличивает скорость движения воздуха, но при этом увеличивается сопротивление и напряжение дыхательных мышц. При показаниях к усилению вдоха или выдоха следует во время выполнения дыхательных упражнений произвольно изменять соотношение по времени между вдохом и выдохом (так, при усилении выдоха - увеличивать его продолжительность).

Лечебная физическая культура противопоказана в острой стадии большинства заболеваний, при тяжелом течении хронических заболеваний, при злокачественных опухолях мышц.

\chapter*{Заключение}
\addcontentsline{toc}{chapter}{Заключение}%\tabularnewline

Из всего вышесказанного и осмыслив роль дыхательной системы в нашей жизни можно сделать  вывод  о ее важности в нашем существовании.

От   процесса дыхания зависят все  процессы жизнедеятельности организма.  Болезни дыхательной системы  очень опасны и требуют серьезного подхода и по возможности полного выздоровления  больного. Запускание таких болезней может привести к тяжелым последствиям вплоть до летального исхода.

\addcontentsline{toc}{chapter}{Литература}
\begin{thebibliography}{99}

\bibitem{nodejs} Алексеенко Н. Ю. <<Основы  Физиологии>>

\bibitem{ajax} Гребнев А.Л., Шептулин А.А. <<Основы общего ухода за больными>>

\bibitem{igit} Баешко А.А., Гайдук Ф.М. <<Неотложные состояния>>

\bibitem{sss} Машков В.В. <<Основы лечебной физической культуры>>

\end{thebibliography}


\end{document}

биологич
физика
истрос
алгеьра
оитератуо
алг
фыизра