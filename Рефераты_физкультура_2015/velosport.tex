\documentclass[a4paper,14pt]{extreport}
 
\usepackage{extsizes}
\usepackage{cmap}
\usepackage[T2A]{fontenc}
\usepackage[utf8x]{inputenc}
\usepackage[russian]{babel}
\usepackage{cyrtimes}

%%%%%%%%%%%%%%%%%%%%%%%%%%%%%%%%%%%%%%%%%%%%%%%%%%%%%%%%%%%%%%%%%%%%%%%%%%%%%%%%%%  
\usepackage{graphicx} % для вставки картинок
\graphicspath{{img/}}

\usepackage{amssymb,amsfonts,amsmath,amsthm} % математические дополнения от АМС
\usepackage{indentfirst} % отделять первую строку раздела абзацным отступом тоже
\usepackage[usenames,dvipsnames]{color} % названия цветов
\usepackage{makecell}
\usepackage{multirow} % улучшенное форматирование таблиц
\usepackage{ulem} % подчеркивания
\linespread{1.3} % полуторный интервал
\renewcommand{\rmdefault}{ftm} % Times New Roman
\frenchspacing
\usepackage{geometry}
\geometry{left=3cm,right=1cm,top=2cm,bottom=2cm,bindingoffset=0cm}
\usepackage{titlesec}
% \definecolor{black}{rgb}{0,0,0}
% \usepackage[colorlinks, unicode, pagecolor=black]{hyperref}
\usepackage[unicode]{hyperref}


\begin{document}

% -*- root: project.tex -*-
\begin{titlepage}
\newpage

\begin{center}

Муниципальное автономное образовательное учреждение \\
лицей № 180 \\
г. Нижнего Новгорода \\

% \hrulefill
\end{center}
 
% \flushright{КАФЕДРА № ХХХ}

\vspace{14em}

\begin{center}
\large{Реферат}
\end{center}

% \vspace{.5em}
 
\begin{center}
Велосипедный спорт
\end{center}

\vspace{4.5em}
 
\begin{flushright}
Выполнил: Сарафанов Фёдор, \\
ученик 10 <<А>> класса \\
Научный руководитель: \\
Яшков Александр Николаевич, \\
учитель физической культуры \\ 
первой квалификационной категории
\end{flushright}
 
\vspace{\fill}

\begin{center}
Нижний Новгород \\
2015
\end{center}

\end{titlepage}
\addtocounter{page}{1}

\tableofcontents
%\large

\chapter*{Введение}
\addcontentsline{toc}{chapter}{Введение}

Велосипедный спорт — одна из самых популярных форм двигательной активности. Являясь полностью аэробным упражнением, он укрепляет сердце и легкие, мышцы ног, хотя занятия им и не способствуют развитию мышц туловища и плечевого пояса. В этом велоспорт схож со скоростным бегом на коньках или скоростным спуском (горные лыжи), поскольку в той же степени укрепляет мышцы ног. Таким образом, летом можно заниматься велосипедным спортом, чтобы сохранить хорошую форму, а зимой — горнолыжным или конькобежным. Бегуньи и теннисистки обнаружили, что занятия велосипедным спортом помогают им улучшить общую выносливость.

<<Занятия велосипедным спортом прекрасно тренируют сердечно-сосудистую систему, — считает доктор медицины А. Дж. Райен, редактор журнала <<Физишн энд спортсмедисн>>. — Наиболее большие по размеру сердца наблюдались у велосипедистов. Конкуренцию им могут составить только лыжники и гребцы>>. У велосипедистов экстра-класса отмечаются также очень высокие показатели потребления кислорода и одна из самых низких ЧСС.

Согласно данным физиолога У. Хескелла, занятия велосипедным спортом очень полезны для снижения веса, увеличения количества липопротеинов высокой плотности и снижения количества липопротеинов низкой плотности в крови. <<Они улучшают биохимические процессы, протекающие в организме, что помогает снизить шансы заболевания диабетом или появления невосприимчивости к глюкозе. Они не предупреждают появления диабета, но могут свести к минимуму его последствия>>.

Занятия велоспортом укрепляют мышцы бедер и в меньшей степени голени, но не влияют на развитие гибкости. Мышцы стопы, колена и бедра получают гораздо меньшую нагрузку, чем при беге или прыжках.

\chapter{Велосипед и оздоровительная физическая культура}
По степени влияния на организм все виды оздоровительной физической культуры (в зависимости от структуры движений) можно разделить на две большие группы: упражнения циклического и ациклического характера.
Циклические упражнения - это такие двигательные акты, в которых длительное время постоянно повторяется один и тот же законченный двигательный цикл. Езда на велосипеде, равно как и ходьба, бег, ходьба на лыжах, езда не велосипеде, плавание, гребля, относятся к таким упражнениям. Но сначала необходимо рассмотреть историю создания самого велосипеда и историю велоспорта.

\section{История велоспорта}
В отличие от многих олимпийских видов спорта, история которых исчисляется тысячелетиями, велосипедный спорт возник сравнительно недавно - в конце XIX века. Но прежде, чем рассказать о зарождение и развитии велосипедного спорта, уместно будет вспомнить историю возникновения самого велосипеда. Идея передвижения на колесах за счет мускульной силы человека родилась очень давно. Различные коляски, повозки, приводимые в движение человеком на 4, 3 и 2 колесах, появились почти одновременно в Германии, Франции, Англии и других странах. Не остались в стороне и российские умельцы. Первый год XIX столетия ознаменовался созданием первого двухколесного цельнометаллического велосипеда. Сотворил его на нижнетагильском заводе крепостной мастеровой Ефим Михеевич Артамонов. Он в день Ильи Пророка года 1800 ездил на диковинном велосипеде по улицам Екатеринбурга, а в 1801г. добрался на своем самокате в Москву, преодолев по бездорожью более 5тыс. км. В столице он показал свое детище во время коронации царя, за что был освобожден от крепостной зависимости. Весил <<новорожденный>> больше 40кг. Но мастеровому не помогли. Патент на изобретение не выдали. Летописцы велосипедной истории считают родоначальником современной веломашины изобретение немца лесничего Карла фон Драйса из Мангейма. В 1814г. он построил свой деревянный двухколесный велосипед, который имел управляемое переднее колесо и мог двигаться в любую сторону.
В 1817 г. Карл фон Драйс получил патент в Германии на изобретение велосипеда. Дальнейшее усовершенствование велосипеда происходило в Западной Европе и Америке.

Официальной датой начала проведения соревнований по велосипедному спорту принято считать 31 мая 1868 года, когда на аллеях парка парижского пригорода Сен-Клу была организована гонка на 2 000м. Ее победителем стал англичанин Дж. Мур, который в следующем году триумфально финишировал во впервые проводившейся шоссейной гонке на велосипеде Париж-Руан на 120км. Победитель англичанин Мур окончил дистанцию за 10 час 45 мин, т.е. со скоростью спортсмена-пешехода. В дальнейшем скорость велосипеда была доведена до 30км/ч. Изобретатели увеличили переднее колесо и уменьшили заднее. Эти машины получили название <<пауков>>. Однако езда на них была не безопасна. При малейшем толчке велосипед опрокидывался и седок совершал прыжок через руль. В 1885г. на таком велосипеде Томас Стивенс совершил кругосветное путешествие, передвигаясь со скоростью 60км в день. Одновременно с <<пауками>> стала распространяться другая модель велосипеда, названная <<кенгуру>>. Впервые была применена цепная передача, а увеличение скорости достигалось соотношением шестерен. Дальнейшее совершенствование велосипеда шло очень быстро. Настоящий переворот в этом деле произошел в 1885г., когда шотландским ветеринаром Денлопом была изобретена и применена полая пневматическая шина. Пневматические шины были тем новшеством, которого нахватало велосипеду для окончательного признания его удобным способом передвижения.

Начиная с 1870 года в различных городах Франции, Италии, Великобритании и других странах начинается строительство треков. В 1890г. в велосипедном спорте наметилось несколько категорий гонщиков: профессионалы, любители и независимые. С появлением пневматических шин велосипедные гонки получили широкое распространение не только на треке, но и на шоссе. В 1891г. было положено начало традиционной дорожной гонке Бордо-Париж (600км). Первый чемпион мира по треку для спортсменов любителей был организован в Чикаго в 1893г. С 1895г. начинают разыгрываться чемпионаты мира для спринтеров-профессионалов. В конце XIX века особой популярностью пользовались шестидневные гонки на треке. Первая такая гонка была устроена в 1896г. в Америке. А первые чемпионаты мира по шоссе для любителей стали проводиться в 1921г. на 190км; для гонщиков-профессионалов - в 1927г. на 185км; среди женщин в 1958г. Наиболее значительной дорожной гонкой является гонка для профессионалов вокруг Франции - <<Тур де Франс>>. Впервые она состоялась в 1903г., общая протяженность - 5.000км. Условия этой сверхдальней гонки меняются каждый год, подвергаются изменению и этапы.

Велосипедный спорт - одна из немногих дисциплин, которая была представлена на всех Олимпийских играх современности. Причем для участников Игр I Олимпиады в Афинах был построен трек, во многом отвечающий современным стандартам.8 апреля 1896г. были даны первые олимпийские старты, в которых приняли участие велосипедисты из 5 стран Европы. В программу соревнований было включено 5 видов гонок на треке и один на шоссе. Уникально достижение французского спортсмена П. Массона, ставшего на одной Олимпиаде трехкратным чемпионом. Во время проведения этой I Олимпиады еще не было Международного Союза велосипедистов (УСИ), он возникнет лишь в 1900г., но завидную настойчивую инициативу по включению велогонок в программу проявили представители Международного союза рабочих велосипедистов <<Солидарность>>. Долгое время организаторы игр составляли программу соревнований по своему усмотрению иногда устраивая гонки только на треке, как это было в 1900г., 1904г. (результаты состязаний 1904г., в которых участвовали лишь спортсмены США, не вошли в официальные протоколы олимпийских игр), или только на шоссе, как в 1912г. В 1908-1972г. г. проводились трековые гонки на тандемах. Современный регламент соревнований в общих чертах стал определяться с 1928 года. Женщины впервые участвовали в Олимпиаде 1984г. На первых Олимпийских играх лидировали велосипедисты Франции и Великобритании, затем к ним присоединились спортсмены Дании, Италии, Германии, СССР. На Олимпийских играх в Атланте в 1996г. велосипедисты крутили педали в погоне за медалями на шоссе, на треке и по дорогам Атланты. Впервые в истории Олимпийских игр присоединился горный велосипед, для которого характерны широкие колеса.

Горный велосипед изобрели около 20 лет назад хиппи из северной Калифорнии, и в прошлом американцы были лучшими в этом виде спорта. Гонка по пересеченной местности прошла по холмам и рощам парка для верховой езды. Впервые к участию в Играх допущены профессионалы. В каждом виде соревнований используются разные типы велосипедов, а цена моделей для трековых и шоссейных гонок может достигать 4 500 долларов.

Шоссейный велосипед может иметь до 14 передач и снабжены первоклассными тормозами. Трековый велосипед имеет только одну передачу и не имеет тормозов. Скорость велосипедистов на треке на закругленных бортиках может достигать 64,5км в час. В Атланте построен первый переносной трек <<Стоун маунтин>>. Борьба за первое место всегда очень напряженная: в 1964 году, например, только 0,16 секунд отделяло победителя от спортсмена, занявшего 51 место; а в 1976 году команда из Западной Германии выиграла гонку - преследование на 4км, потому что спортсмены заполнили шины своих велосипедов более легким гелием, а не воздухом. Одним из наиболее заметных явлений олимпийского движения в последние десятилетия является постоянное расширение видов программы. К началу 90-х годов приобрели массовую популярность новые виды велосипедного спорта - маунтенбайк (горный велосипед), триатлон, ряд дисциплин трековой программы соревнований. Эта тенденция затрагивает и программу женского велосипедного спорта. На XXVII Олимпийских играх в Сиднее в 2000г. программа соревнований по велосипедному спорту расширена до 18 видов, из них 7 комплектов медалей разыгрывались между женщинами.

\chapter{История Tour de France}
Малоизвестно, что знаменитая велогонка Tour de France появилась на свет в результате крупного политического скандала, разделившего всю Францию. Корни этой истории уходят в 1894г. Офицер французской армии Альфред Дрейфус, был приговорен к пожизненному заключению за передачу секретных данных немецкому военному атташе в Париже. Либеральная публика считала дело сфабрикованным и требовала немедленного освобождения Дрейфуса, с чем многие были не согласны. В поддерживавшем Дрейфуса лагере оказался Пьер Жиффар, владелец спортивного журнала Le Velo. В другом граф Дион, промышленник, имевший фабрику по производству велосипедов. Рекламу своей продукции Дион размещал главным образом в Le Velo. Политические разногласия сделали Жиффара и Диона врагами. Впоследствии эта вражда и вылилась в создание <<Тур де Франс>>, поскольку их ссора закончилась разрывом коммерческих контактов.

Дион в ответ решил создать свой собственный журнал, также посвященный велоспорту. Новому журналу необходима реклама, и помощник Диона - Лефевр - предложил утроить велосипедный пробег по всей Франции. Вся гонка должна будет состоять из шести этапов общей протяженностью около 2428 км.

1 июля 1903 года 60 велосипедистов (21 профессионал и 39 любителей) отправились колесить по дорогам Франции.

\chapter{Отечественный велоспорт}
Велосипедный первенец родился в нашей стране. Но изобретение крепостного мастера Артамонова не нашло применения. И потому первые велосипеды, появившиеся в России в конце 60-х годов XIX в., были продукцией заводов Англии и Германии. В 1880г. Петербургская городская управа зарегистрировала около 100 велосипедов, а через 2 года они появились в Москве. Для булыжной мостовой они были мало пригодны. К тому же езда на них в городе была строжайше запрещена.

Первое официальное состязание было проведено в Москве 24 июля 1883г. на двух дистанциях 1,5 и 7,5 верст. Фактически они носили международный характер. В состязаниях участвовали американский, австрийский и английские спортсмены.1883 год и отмечается как дата рождения велосипедного спорта в России. Вторым по значению для развития отечественного велоспорта было соревнование, состоявшееся 23 сентября 1884г. на Царском Лугу (Марсовом поле) в Петербурге. Эти выступления ускорили создание Московского и Петербургского обществ велосипедистов-любителей.

Уже в 1882г. в Петербурге было создано первое русское велосипедное общество. Вскоре возникло и Московское общество велосипедистов-любителей и Московский клуб велосипедистов. Уже к концу 80-х г. г. велосипедные общества созданы в других городах России, а велосипедные кружки были во многих губернских и уездных городах. В 1896г. среди любителей велосипедной езды москвичи увидели Л.Н. Толстого.70-летний писатель прекрасно владел машиной. Почитатели подарили ему велосипед с серебряными спицами.

В 1886г. появилась первая конструкция современного велосипеда с колесами одинакового диаметра и цепной передачей на заднем колесе. Центрами велосипедной жизни стали города Рига, Киев, Одесса. Вначале состязания проводили на ипподромах и шоссе. В дальнейшем на средства общества и крупных предпринимателей были выстроены циклодромы (треки). В 1891г. в Москве был построен трек, покрытый цементом. В Петербурге действовал деревянный трек длиной 250м, который собирался за полчаса. В Одессе в 1894г. был построен первый велотрек с асфальтовым покрытием длиной 360м.

Важнейшим событием в спортивной жизни России явился розыгрыш в 1891г. звания <<Первый ездок России>>. Его назвали первым <<Всероссийским чемпионатом>>. В Москву съехались сильнейшие гонщики Петербурга, Киева, Одессы. В программе была гонка на дистанции 7,5 версты, считавшейся тогда классической. Подобные соревнования были проведены потом в 1892-1894г. г. в Москве. В 1894г. впервые был проведен интереснейший марафон от Москвы до Нижнего Новгорода, но шоссе оказалось таким разбитым, что только двое спортсменов доехали до Волги.

В 1895г. зарождается самая тяжелая и длительная гонка Петербург - Москва. Победителем ее стал волевой и выносливый гонщик М. Дзевочко. Так, с первенства мира 1896 г., состоявшегося в Копенгагене, он вернулся вторым призером в лидерской гонке на 100км и занял первые места в гонках на 1 и 10миль.А. Панкратову принадлежит приоритет первого русского велосипедиста, свершившего в 1911-1913г. г. кругосветное путешествие по маршруту, который официально утвердил Международный Союз велосипедистов. А. Бутылкин, Г. Вашкевич, П. Ипполитов, С. Уточкин - все эти спортсмены стали гордостью нашего велоспорта.
Велосипедная лихорадка гонок, охватившая в конце XIX века Западную Европу и Америку, перекинулась в Россию. Зритель охотно шел на велогонки. Велосипедный спорт был в расцвете славы. Но тут в прибыльное дело вмешались крупные торговые фирмы. Они скупали лучших гонщиков и превратили велосипедные соревнования в чисто коммерческое предприятие, торговые махинации целиком переносятся на спортивную арену. Пропадает спортивный интерес, гонки перестают привлекать зрителей.

После Октябрьского переворота (1917г) Тульский трек оставался единственным в стране действующим <<велосипедным островком>>. 

В 1918г. Тульский губернский олимпийский комитет совместно с Московским кружком велосипедистов организовал в Туле розыгрыш первого чемпионата Советской России. Этим было положено начало развития велосипедного спорта в РСФСР. В 20-х годах прошли: первая гонка по Московскому Садовому кольцу, тогда еще вымощенному крупным булыжником, первая встреча гонщиков Москвы и Петрограда, первый чемпионат Сибири, всеукраинская олимпиада в Харькове, первый чемпионат страны, проведенный в 1923г. на Московском ипподроме: соревнования на треке, а на шоссе - в 1928г. Резкий толчок развитию советского велосипедного спорта дала Всесоюзная спартакиада 1928г. (Победитель спринтерской гонки А. Куприянов в течение многих лет занимал пост вице-президента УСИ и ФИАК).12 августа 1937г. в Москве стартовала первая советская многодневная гонка. На олимпийской трассе впервые русские велосипедисты выступили в Стокгольме в 1912г. Из 12 гонщиков только один велосипедист закончил 320-километровую гонку, он был 60-м. В 1926г. российские гонщики впервые участвовали в матчах с зарубежными спортсменами - членами рабочих спортивных организаций. Тогда наши велосипедисты убедительно победили на состязаниях в Париже и Бермене. 

Однако официальный счет выступлениям на международной арене был открыт лишь в 50-е годы, после признания наших гонщиков Международным союзом велосипедистов в 1952г. А в Международную любительскую федерацию велоспорта СССР вошел в 1965г. 

Первым советским рекордсменом мира в гонке на треке на 1км с ходу был Р. Варгашкин. Олимпийский дебют советских гонщиков состоялся в 1952г., но был неудачным. За горечью первых неудач пришла радость первых побед. Первым советским олимпийским чемпионом по велосипедному спорту стал В. Капитонов. В 1976 и 1980г. г. сборная СССР стала олимпийским чемпионом в шоссейной командной гонке, а в групповой гонке на 189км победу праздновал С. Сухорученков. На Олимпийских играх 1988г. советские велосипедисты завоевали 4 золотые медали, причем А. Кириченко победил в гите - номере программы, в котором наши мастера никогда прежде не поднимались на верхнюю ступень олимпийского пьедестала почета. Надо отметить и победу Эрики Салумяэ из Таллинна, которая выступала за команду СССР, в спринтерской гонке для женщин, впервые включенной в олимпийскую программу. В 1996г. золото России завоевала Зульфия Забирова (фото) в индивидуальной гонке на время.

В настоящее время, чтобы на равных бороться с ведущими велосипедными странами необходимо, по мнению специалистов, делегировать как можно больше российских велосипедистов в зарубежные профессиональные команды или создавать собственные профессиональные команды, решая вопросы финансирования на государственном уровне. Те, кто выходит на трассы, никогда не скажут друг о друге <<велосипедист>>, даже не скажут <<спортсмен>>. Непременно произнесут слово <<гонщик>>, ибо для них гонщик - синоним слова <<боец>>. Этот титул - большая честь, его надо заслужить.

\chapter{Развитие велосипеда}

История развития велосипеда прошла несколько этапов. Сначала был так называемый низкий велосипед, его сменил высокий, чтобы затем вновь уступить место низкому. По настоящему велосипед начал совершенствоваться с начала XIX века. Однако конструкции с колесами, предназначенные для самостоятельного перемещения их человеком, упоминается уже в XV столетии. Так, Майнингенская хроника 1447 года повествует о перемещающемся устройстве, приводимым в движение водителем.
В 1761 году тележник Михаэль Каслер “проскакал” 2км из Браусдорфа (округ Магдебург) в поселок Бедру (нынешнее название Браусбедра). Его машина представляла собой два обитых стальными обручами деревянных колеса, которые соединялись скамеечкой для сиденья. Вес ее составлял, должно быть, приблизительно 125кг.

Путь к современному велосипеду был проложен только в 1817 году Людвигом Драйсом. На своем самокате длиной 2,4м с 30 дюймовыми колесами он ввел новшество - управляемое переднее колесо. На этой машине Драйс сумел преодолеть расстояние от Лейпцига до Дрездена (111км) за 7 часов.

В 1860 году Пьер Мишо, каретник из Парижа, ремонтируя старый самокат, установил на передние колеса две педали. Уже через два года такие машины начали пускаться серийно под названием “велосипед” (“вело” - быстро, “пед” - нога).
Открытие Мишо окончательно утвердило велосипед. Начали появляться новые усовершенствования. Если до того велосипеды изготавливались преимущественно из дерева, то в последующие 10 лет колеса оделись плотной резиной, а для рам и полых вилок начали использовать трубки.

В 1870 году англичанин Хилман начинает продавать первые полностью металлические велосипеды с высокими колесами. Величина переднего колеса, как правило, составляла 54дюйма (современных колес - 27дюймов, ровно в два раза меньше).
В 1885 году англичанин Старлей изготовил так называемый “ровер” - первый низкий велосипед с цепным приводом. Он весил около 20кг. В 1888 году ирландец Дэнлоп изобрел и выпустил в продажу шины, наполняемые воздухом. С этого момента наступает настоящий расцвет двухколесного велосипеда. В конце XIX - начале XX века в мире насчитывалось около миллиона велосипедистов.

Первые соревнования велосипедистов состоялись в Париже в 1868 году. В 1875 году устанавливается первый мировой рекорд на одну английскую милю (1660м) для высокого велосипеда (2 мин 55 сек). В 1876 году был установлен первый неофициальный мировой рекорд на дальность поездки (за 1ч - 25,508км). Официальный рекорд в этом виде, установленный в 1893 году французом Дегранжем, составлял уже 32,325км. В 1893 году на шоссейных гонках Вена - Берлин за 32 часа 22 минуты было преодолено расстояние 591км.

Зубчато-цепная передача начала применяться на рубеже веков. Она состояла из ведущей шестерни, соединенной с рычагами, и ведомой, расположенной на ступице заднего колеса. Соединенные цепью, они и составляли передачу. Вначале такая передача на велосипедах делалась жесткой. Надо было крутить педали на каждом преодолеваемом метре, даже при спуске с горы. Холостой ход дал возможность не двигать ногами, если во время перемещения не требовалось дополнительного усилия. Посредством замены ведущей или ведомой шестерни можно было регулировать передаточное соотношение, что открывало путь к изменению скоростей.

В дальнейшем ступицу заднего колеса стали изготавливать так, чтобы слева и справа на нее можно было навинчивать звездочки-шестерни различной величины. Благодаря этому появилась возможность, подкручивать заднее колесо, осуществлять замену передачи. В 30-е годы разработали первый переключатель скоростей, который, однако, не обеспечивал достаточной надежности (цепь слишком натягивалась или, наоборот, спадала). Позднее этот узел был переработан и появился современный переключатель скоростей параллельного типа. Ведущую шестерню сталь изготавливать двух - или даже трех дисковой. Специальный переключатель позволял на ходу перебрасывать цепь с одного диска на другой. В результате современный велосипедист при двух ведущих дисках и пяти ведомых может располагать десятью скоростями. За последние 30 лет велосипедные узлы и детали значительно усовершенствовались. Многие части велосипеда такие, как руль, рулевая колонка, обода, педали, шестерни, тормоза, изготавливаются сегодня из легких металлов. Недавно и раму начали делать из легких металлов и целиком склеенной. Гибкие велосипедные шины (или трубки) благодаря значительному улучшению дорожных покрытий стали легче. Нынешние шоссейные трубки весят от 250 до 330г. Собранная из таких частей гоночная машина весит от 9 до 9,5кг, то есть она на 2-3кг легче обычных гоночных машин.

\chapter{Cовременный спортивный велосипед}
Гоночный велосипед - это сложная машина, состоящая из многих узлов. Очень важно, чтобы все они работали нормально и, прежде всего, чтобы велосипед, предназначенный для езды по улице, был соответственно оборудован. Каждая деталь имеет свое назначение, и велосипедист должен обязательно ее знать.

Шоссейный и трековый гоночные велосипеды заметно отличаются друг от друга по конструкции и технологии производства. Трековый, на котором никогда не ездят по улицам, не имеет тормозов, холостого хода, переключателя передач и осветительного устройства. Он оснащен специальными трековыми колесами и шинами. Спицы трековых колес в точках пересечения обмотаны проволокой и пропаяны. От этого повышается прочность колес. Есть отличия и в прочности и конструкции рамы, крутизне вилки.

Правильная посадка на велосипеде важна для гонщика. От того, под каким углом будут работать коленные суставы, бедра и стопы, от положения туловища и рук зависит способность мышц прилагать максимальное усилие. Положение тела влияет и на положение внутренних органов (легких, органов пищеварения). Слишком сильное или слишком слабое сгибание и разгибание суставов не позволит развивать максимальную мощность. Правильная посадка определяется взаимным расположением голеней, бедер и рук, величиной рамы, высотой седла и руля.

Для некоторых видов шоссейных и трековых гонок, а также в определенных ситуациях на тренировках и соревнованиях используются специальные виды посадок.

Прежде чем начать ездить на велосипеде, надо научиться правильной посадке, соответствующей особенностям телосложения: росту, длине ног, рук и туловища. Время от времени, по мере увеличения роста, посадку необходимо проверять и корректировать. Правильность посадки зависит от размера рамы, установки седла и руля, подбора шатунов педалей.
Седло можно перемещать вертикально, вперед и назад и придавать ему желаемый наклон. Высота седла измеряется от средней точки оси каретки до середины поверхности седла. Обычно седло устанавливается параллельно раме. Его острие не должно быть направленно вверх или вниз. Есть три варианта установки высоты седла.

Первый вариант. Велосипедист сидит на седле прямо, пяткой вытянутой ноги опирается на педаль, находящуюся в нижнем положении.

Для гонок на треке седло рекомендуется устанавливать на 1 - 1,5см выше этой нормы; для езды по пересеченной местности на такую же величину ниже.

Второй вариант. Одна из педалей устанавливается в крайнее нижнее положение. Сидя на седле прямо, поставьте носок ноги под педаль так, чтобы вся стопа располагалась параллельно земле.

Третий вариант. Длина рук обычно соответствует длине ног и наоборот, следовательно, положение седла может определяться и длиной рук. Этот вариант используется когда седло нужно установить очень быстро. Наклонитесь над рамой так, чтобы вытянутая рука составляла с продольной осью велосипеда угол 90°. Колени сгибать до тех пор, пока седло не окажется в подмышечной впадине вытянутой руки. При этом седло устанавливается на такой высоте, чтобы средний палец руки оказался на уровне оси каретки. Не следует перегибаться плечом через седло. Оно должно попасть точно под мышку.

Высота руля устанавливается в зависимости от высоты седла. При переустановке руля по высоте болт крепления отворачивается на 3-4 оборота. Руль всегда следует располагать немного ниже поверхности седла: для езды на шоссе - на 1-2см, для гонок на треке - на 2-4см. Расстояние от седла до руля соответствует длине руки от локтя до кончиков пальцев. Стоя рядом с велосипедом, приложить локоть к острию седла. Вытянутые пальцы должны касаться поперечной трубы руля рядом с выносом руля.
Тормоза, их безукоризненная работа имеют решающее значение для безопасности самого велосипедиста и других гонщиков. Тормозные носики, укрепленные в тормозных ручках, передают усилие руки на тормозные скобы и колодкодержатели. Колодки из резины давят на обода с двух сторон и тормозят колеса. Тормозное усилие регулируется нажатием руки. Тормозные носики всегда должны быть в порядке, а винты скоб надежно затянуты. В свободном состоянии резина тормозных колодок должна находиться на расстоянии 2 - 3мм от обода. В притянутом состоянии резина тормозных колодок должна прилегать к ободу всей поверхностью.

Руль состоит из передней крепи и речек, которые можно поворачивать при установке. Концы руля (с заглушками, чтобы не получить повреждений при падении) устанавливаются приблизительно параллельно земле. Труба, ручки и концы руля, за исключением тормозных рычагов, обматывают клейкой лентой, а иногда покрывают бесцветным лаком.

\chapter{Влияние велоспорта на организм}

Влияние занятием велоспортом на организм очень положительно и широко направлено. Во-первых, укрепляется опорно-двигательный аппарат, во-вторых, - сердечно-сосудистая система, в-третьих, способствует поправки фигуры и пополнение мышечной массы.

Велосипед на сегодняшний день можно считать не только самым эффективным способом укрепления здоровья, но и самым доступным. Какой бы вид спорта ни выбрать для себя в качестве альтернативы тренажёрам, всё равно придётся попотеть - и в прямом, и в переносном смысле этого слова. Результат в виде подтянутой спортивной фигуры. Как у любого вида физической нагрузки велосипедные прогулки, как самый популярный на сегодня из способов поддержания формы, имеют свои плюсы и минусы.

Плюсы: можно считать упражнение на выносливость, в основном развивающим нижнюю часть, а точнее - бёдра и ягодицы. Также подтягивает заднюю поверхность бёдер, практически не задействованную во многих других видах физической активности. Антицеллюлитный эффект от занятия велосипедом.

Минусы: они как продолжение достоинств. Велосипед не рекомендован тем, кто никак не заинтересован в увеличении объёма бёдер. Фигура велосипедиста отличается некоторой гипертрофией в этой области. Если слишком упорно крутить педали, можно <<перекачать>> и икры. 

\chapter*{Заключение}
\addcontentsline{toc}{chapter}{Заключение}%\tabularnewline

Велосипед на сегодняшний день можно считать не только самым эффективным способом укрепления здоровья, но и самым доступным. К тому же с его помощью можно и укрепить опорно-двигательный аппарат, и сердечно-сосудистую систему, и поправить фигуру и пополнить мышечную массу. Вдобавок к этому и чисто эстетическое чувство при поездках как по городу и его паркам, так и на лоне природы.

Для любого вида физической культуры важна системность и постоянство. Именно поэтому велосипед целесообразно сочетать по окончании летнего периода с лыжами, или занятиями на велотренажере.

\addcontentsline{toc}{chapter}{Литература}
\begin{thebibliography}{99}

\bibitem{nodejs} Машков А.В. <<Основы лечебной физической культуры>>

\bibitem{ajax} Васильев В.Е. <<Лечебная физическая культура>>

\bibitem{igit} Аксельрод С.Л.  <<Спорт и здоровье>>

\bibitem{iii} \href{http://www.gpsies.com/mapUser.do?username=osabio}{http://www.gpsies.com/mapUser.do?username=osabio}

\end{thebibliography}
\end{document}